\documentclass[12pt,a4paper]{article}
\usepackage{amsmath}
\usepackage{geometry}
\usepackage{graphicx}
\usepackage{hyperref}

\geometry{margin=1in}

\title{Volumes of Solids}
\author{Tutoring Centre Ferndale\\
\includegraphics[width=4em]{ApS_logo.png}}
\date{}

\begin{document}

\maketitle

Volume measures the amount of space that a solid object occupies. It is expressed in cubic units, such as cubic centimeters (cm\(^3\)) or cubic meters (m\(^3\)).


Understanding the volumes of solids helps in real-world applications such as calculating the capacity of containers or the amount of material needed to construct objects.


Here are some important terms that will help you understand solids and their volumes:

\begin{itemize}
    \item \textbf{Face:} A flat surface of a three-dimensional solid.
    \item \textbf{Edge:} The line segment where two faces of a solid meet.
    \item \textbf{Cross Section:} The two-dimensional shape you see when you cut through a solid along a plane.
    \item \textbf{Solid of Uniform Cross Section:} A solid where every cross section taken parallel to its base is identical in shape and size, such as a prism or a cylinder.
    \item \textbf{Apex:} The single point where the triangular faces of a pyramid or cone meet.
    \item \textbf{Polygon:} A closed two-dimensional shape made of straight-line segments.
    \item \textbf{Regular Polygon:} A polygon with all sides and angles equal (e.g., an equilateral triangle or a square).
    \item \textbf{Polyhedron:} A three-dimensional solid with flat polygonal faces, straight edges, and vertices.
\end{itemize}

\section*{Volume Formulas for Solids}

\subsection*{1. Rectangular Prism (Cuboid)}
\[
V = l \times w \times h
\]
where \(l\), \(w\), and \(h\) are the length, width, and height, respectively.

\subsection*{2. Cube}
\[
V = s^3
\]
where \(s\) is the length of a side.

\subsection*{3. Triangular Prism}
\[
V = \frac{1}{2} \times b \times h \times l
\]
where \(b\) is the base of the triangular face, \(h\) is the height of the triangular face, and \(l\) is the length of the prism.

\subsection*{4. Cylinder}
\[
V = \pi r^2 h
\]
where \(r\) is the radius of the circular base and \(h\) is the height.

\subsection*{5. Pyramid}
\[
V = \frac{1}{3} \times B \times h
\]
where \(B\) is the area of the base and \(h\) is the perpendicular height.

\subsection*{6. Cone}
\[
V = \frac{1}{3} \pi r^2 h
\]
where \(r\) is the radius of the circular base and \(h\) is the height.

\subsection*{7. Sphere}
\[
V = \frac{4}{3} \pi r^3
\]
where \(r\) is the radius.

\section*{Examples}

\subsection*{Example 1: Volume of a Rectangular Prism}
A box has a length of 10 cm, a width of 5 cm, and a height of 4 cm.  
\[
V = l \times w \times h = 10 \times 5 \times 4 = 200 \, \text{cm}^3
\]

\subsection*{Example 2: Volume of a Cylinder}
A water tank has a radius of 7 m and a height of 10 m.  
\[
V = \pi r^2 h = \pi (7)^2 (10) = \pi \times 49 \times 10 = 490 \pi \, \text{m}^3
\]  
Approximately \( V = 1539.38 \, \text{m}^3 \) when \(\pi \approx 3.1416\).

\subsection*{Example 3: Volume of a Sphere}
A ball has a radius of 3 cm.  
\[
V = \frac{4}{3} \pi r^3 = \frac{4}{3} \pi (3)^3 = \frac{4}{3} \pi \times 27 = 36 \pi \, \text{cm}^3 \approx 113.1 \text{cm}^3
\]  

\section*{Practice Problems}

\begin{enumerate}
    \item Find the volume of a cube with a side length of 6 cm.
    \item A cone has a base radius of 5 cm and a height of 12 cm. Find its volume.
    \item A rectangular prism has dimensions 8 cm \(\times\) 3 cm \(\times\) 2 cm. What is its volume?
    \item Find the volume of a cylinder with a radius of 4 m and height of 15 m.  
    \item A triangular prism has a base width of 3 cm, a triangle height of 4 cm, and a prism length of 10 cm. Find its volume.
\end{enumerate}

\end{document}
