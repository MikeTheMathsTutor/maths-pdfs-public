\documentclass[12pt]{article}
\usepackage{amsmath}
\usepackage{amssymb}
\usepackage{multicol}
\usepackage{graphicx}

\title{\textbf{The Remainder \& Factor Theorems}}
\author{Tutoring Centre Ferndale\\
\includegraphics[width=4em]{ApS_logo.png}}
\date{}

\begin{document}

\maketitle

The \textbf{Remainder Theorem} and the \textbf{Factor Theorem} are useful in determining whether a polynomial has a specific factor and in finding the remainder when a polynomial is divided by a linear divisor.

\section*{The Remainder Theorem}

The Remainder Theorem states that if a polynomial $f(x)$ is divided by $(x - c)$, the remainder of this division is $f(c)$. This theorem is very useful for quickly finding the remainder without performing long division.

\vfill

\subsection*{Example}

Consider the polynomial $f(x) = 2x^3 - 3x^2 + 4x - 5$. Find the remainder when $f(x)$ is divided by $x - 2$.

\textbf{Solution:}

According to the Remainder Theorem, the remainder is simply $f(2)$.
\[
f(2) = 2(2)^3 - 3(2)^2 + 4(2) - 5 = 2(8) - 3(4) + 8 - 5 = 16 - 12 + 8 - 5 = 7
\]
Therefore, the remainder is $7$.

\vfill

\newpage

\subsection*{Example}

Let $f(x) = x^4 + 2x^3 - 5x^2 + x - 3$. Determine the remainder when $f(x)$ is divided by $x + 1$.

\textbf{Solution:}

We need to evaluate $f(-1)$:
\[
f(-1) = (-1)^4 + 2(-1)^3 - 5(-1)^2 + (-1) - 3 = 1 - 2 - 5 - 1 - 3 = -10
\]
So, the remainder is $-10$.

\vfill

\section*{The Factor Theorem}

The Factor Theorem is a special case of the Remainder Theorem. It states that $(x - c)$ is a factor of the polynomial $f(x)$ if and only if $f(c) = 0$. This theorem is particularly useful in factorizing polynomials.

\subsection*{Example}

Determine if $x - 3$ is a factor of $f(x) = x^3 - 7x^2 + 14x - 6$.

\textbf{Solution:}

To use the Factor Theorem, evaluate $f(3)$:
\[
f(3) = (3)^3 - 7(3)^2 + 14(3) - 6 = 27 - 63 + 42 - 6 = 0
\]
Since $f(3) = 0$, $x - 3$ is indeed a factor of $f(x)$.

\subsection*{Example}

Check whether $x + 2$ is a factor of $f(x) = 2x^3 + x^2 - 8x + 4$.

\textbf{Solution:}

Evaluate $f(-2)$:
\[
f(-2) = 2(-2)^3 + (-2)^2 - 8(-2) + 4 = -16 + 4 + 16 + 4 = 8
\]
Since $f(-2) \neq 0$, $x + 2$ is not a factor of $f(x)$.

\vfill

\newpage

\section*{Practice Questions}

Answer the following questions using the Remainder and Factor Theorems.

\begin{enumerate}
    \item Find the remainder when $f(x) = 3x^4 - 2x^3 + x - 5$ is divided by $x - 1$.
    \item Determine if $x + 1$ is a factor of $f(x) = x^3 + x^2 - x - 1$.
    \item Evaluate the remainder when $f(x) = 2x^5 - 3x^4 + 6x^2 - 4x + 7$ is divided by $x - 2$.
    \item Verify whether $x - 4$ is a factor of $f(x) = x^3 - 12x + 16$.
    \item Find the remainder when $f(x) = 4x^4 - 5x^2 + 2x - 3$ is divided by $x + 2$.
\end{enumerate}

\vfill

\section*{Answers}

\begin{enumerate}
    \item The remainder is $f(1) = 3(1)^4 - 2(1)^3 + 1 - 5 = 3 - 2 + 1 - 5 = -3$.
    \item $f(-1) = (-1)^3 + (-1)^2 - (-1) - 1 = -1 + 1 + 1 - 1 = 0$. Yes, $x + 1$ is a factor.
    \item The remainder is $f(2) = 2(2)^5 - 3(2)^4 + 6(2)^2 - 4(2) + 7 = 64 - 48 + 24 - 8 + 7 = 39$.
    \item $f(4) = (4)^3 - 12(4) + 16 = 64 - 48 + 16 = 32$. No, $x - 4$ is not a factor.
    \item The remainder is $f(-2) = 4(-2)^4 - 5(-2)^2 + 2(-2) - 3 = 256 - 20 - 4 - 3 = 229$.
\end{enumerate}

\vfill

\end{document}
