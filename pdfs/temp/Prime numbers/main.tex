\documentclass{article}
\usepackage[fontsize=16pt]{fontsize}
\usepackage[a4paper,margin=3cm]{geometry}
\usepackage{setspace}
\usepackage{pifont}
\usepackage{caption}
\usepackage{tikz}

\author{}
\date{}
\title{Prime\\Numbers\\
\vspace{28pt}
\begin{normalsize}Applied Scholastics, Ferndale WA \end{normalsize}}

\begin{document}
\maketitle
\newpage
\Large

\section*{Factors}
In English, a factor means a part of something. People might talk about the factors of some situation, meaning the different parts of it.\\

In maths, a factor is a part of a number. Multiplying the factors of a number will equal that number.\\

You can see this when you arrange different numbers of objects into rectangles. Most numbers can form exact rectangles or squares.\\

8 things can be made into a line of 8 things in 1 line $\otimes\otimes\otimes\otimes\otimes\otimes\otimes\otimes$, or a line of 4 things and a line of 2 things.

\begin{center}
\otimes\otimes\otimes\otimes\\
\otimes\otimes\otimes\otimes\\
\end{center}

The factors of 8 are 1, 2, 4 and 8, because 1 x 8, 2 x 4, 4 x 2, and 8 x 1 all equal 8.

\pagebreak

\section*{Composite}
Composite means made of different parts, like white light is a composite of rainbow colours.\\

In maths, numbers that can be broken up into smaller factors are called composite numbers.

\begin{center}
\otimes\otimes\otimes\otimes\otimes\\
\otimes\otimes\otimes\otimes\otimes\\
\otimes\otimes\otimes\otimes\otimes\\
\vspace{16pt}
$3 \times 5$\\
\vspace{16pt}
\otimes\otimes\otimes\otimes\otimes
\otimes\otimes\otimes\otimes\otimes
\otimes\otimes\otimes\otimes\otimes\\
\vspace{16pt}
$1 \times 15$\\
\end{center}

For example, 15 is a composite number, made up of the factors 1, 3, 5 and 15, because 1 x 15, 3 x 5, 5 x 3, and 15 x 1 all equal 15.\\

\pagebreak

\section*{Prime}
Prime means the first or the most important. In Canberra there are ministers in charge of different things, but the leader of them all is called the prime minister.\\

In maths, prime means a number that is not a composite number because it’s only factors are 1 and itself. It isn’t made up of any smaller parts. It can’t broken down because it can’t be evenly divided by any other numbers.\\

Any number is either a prime number or a composite number.\\

The prime numbers between 1 and 100 are 2, 3, 5, 7, 11, 13, 17, 19, 23, 29, 31, 37, 41, 43, 47, 53, 59, 61, 67, 71, 73, 79, 83, 89, and 97.

\pagebreak

\section*{Prime Factors}
A prime factor means a factor that is a prime number so that it can’t be broken down into any smaller factors.\\

Product means the result of multiplying. You can take any composite number and write it as a product of prime factors, which is as far as the composite number can be broken down.\\

For example, $24 = 2 \times 2 \times 2 \times 3$, is 24 written as a product of its prime factors.

\pagebreak

\section*{Prime Factor Trees}

To find the prime factors of a number, you start with the smallest prime and keep dividing your number by that prime until it won’t divide evenly any more. Then you try dividing it by the next biggest prime, and so on.\\

Its easiest to write this in the form of a tree.

\begin{center}
\begin{tikzpicture}
  [level distance=1.5cm,
  level 1/.style={sibling distance=2cm},
  level 2/.style={sibling distance=2cm}]
  \node {24}
    child {node {2}}
    child {node {12}
      child {node {2}}
      child {node {6}
        child {node {2}}
        child {node {3}}}
    };
\end{tikzpicture}
\end{center}

From this tree you can see that $24 = 2 \times 2 \times 2 \times 3$ so the prime factors of 24 are 2 and 3.

\pagebreak

You can also see other factors in the tree, 6 and 12, but they aren’t prime factors. 1 is also a factor of 12, of course, because 1 is a factor of any number.\\

You can use a prime factor tree to find all of the factors of a number. All of the factors of 24 are 1, 2, 3, 6, 12, and 24, but the prime factors of 12 are only the ones along the left of the tree, 2 and 3.\\

Being able to work out the prime factors of a number is useful in all sorts of maths that you will be learning.


\pagebreak

\section*{Finding how many factors}

How many factors are there of some number?\\

Do the prime factor tree for the number and write the number, not just as a product of its prime factors, but written as a product of the powers of its prime factors.\\

For example, we saw that $24=2\times2\times2\times3$.\\

Using powers of numbers, and remembering that a number to the first power is equal to itself, this can be written as $24=2^3\times3^1$.\\

Now add 1 to each power, and multiply them to get the number of factors.\\

So 24 has $(3+1)\times(1+1)=4$ factors.\\

\pagebreak

Another example, finding the number of factors of 280:\\

\begin{center}
\begin{tikzpicture}[level distance=1.5cm,
                    level 1/.style={sibling distance=2cm},
                    level 2/.style={sibling distance=2cm}]
\node {280} child {node {2}} child {node {140}
             child {node {2}} child {node {70}
              child {node {2}} child {node {35}
               child {node {5}} child {node {7} }}}};
\end{tikzpicture}
\end{center}
$280=2\times2\times2\times5\times7=2^3\cdot5^1\cdot7^1$
$$\longrightarrow\underbrace{(3+1)\times(1+1)\times(1+1)}_{\textrm{\small{product of each power + 1}}$$
$$=4\times2\times2=16\textrm{ factors of }280.$$

\pagebreak

\doublespacing

\begin{center}

Enquiries

\textbf{Applied Scholastics Ferndale}

Principal: Paula McLennan

mobile phone: 0431 683 306

email address: apsferndale@gmail.com

website: apsferndale.webs.com
\end{center}

\end{spacing}

\end{document}
