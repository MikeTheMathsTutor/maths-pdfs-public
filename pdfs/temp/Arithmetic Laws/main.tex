\documentclass{article}
\usepackage[fontsize=16pt]{fontsize}
\usepackage{setspace}
 
\author{}
\date{}
\title{Arithmetic Laws\\
\vspace{28pt}
\begin{normalsize}Applied Scholastics, Ferndale \end{normalsize}}

\begin{document}

\maketitle
\pagebreak

\section*{Arithmetic Laws}

There are some useful facts about addition, subtraction, multiplication and division. They are used all the time in working out math problems.\\

\subsection*{Commutative Law}
To commute means to go back and forth between to places, like when people commute to work each day. Commutative means that something goes back and forth.\\

Addition and multiplication are commutative because the numbers can be in either order and the result is the same.\\

$$a + b  =  b + a$$
$$a \times b  =  b \times a$$\\

$$3 + 2 = 2 + 3$$
$$22 \times 42 = 42 \times 22$$

\newpage

\subsection*{Associative Law}

To associate means to be in a group with others. An associate is another word for someone you spend time with. Associative means someone or something that likes to associate.\\

Addition and multiplication are associative because the numbers can be put into different groups and the result is the same. It doesn't matter which ones are calculated first. That means that problems can be rearranged to make them easier to work out.\\

$$(a + b) + c  =  a + (b + c)$$
$$(a \times b) \times c  =  a \times (b \times c)$$\\

$$(1+2)+3=1+(2+3)$$
$$(1\times2)\times3=1\times(2\times3)$$

\newpage

\subsection*{Distributive Law}
To distribute means to share something out. Distributive means something that shares.\\

Multiplication is distributive because when a group of terms are multiplied, each term can be multiplied separately.\\

$$a × (b + c)  =  a × b  +  a × c$$\\

$$2 \times (3 \times 4) = 2 \times 3 + 2 \times 4$$

\newpage

\section*{Identities}
Identity means what something is. It also means that someone something is the same person or thing.\\

\subsection*{Additive Identity}
0 is called the additive identity because a number doesn't change when 0 is added.\\

\subsection*{Multiplicative Identity}
1 is called the multiplicative identity because a number doesn't change when multiplied by 1.\\

\newpage
\
\newpage
\
\newpage

\doublespacing
\large

\begin{center}

Enquiries\\

\textbf{Applied Scholastics Ferndale}\\

Principal: Paula McLennan\\

mobile phone: 0431 683 306\\

email address: apsferndale@gmail.com\\

website: apsferndale.webs.com\\

\end{center}

\end{spacing}

\end{document}