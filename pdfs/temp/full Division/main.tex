\documentclass{article}
\usepackage{caption}
\usepackage{cancel}
\usepackage[fontsize=16pt]{fontsize}
\usepackage{amsmath}
\usepackage{polynom}
\usepackage{longdivision}
\usepackage{tikz}
\usepackage[raggedrightboxes]{ragged2e}
\usepackage{tabularx}
\usepackage{bbding}

\author{Mike McLennan}
\date{}
\title{Division\\
\vspace{28pt}
\begin{normalsize}Applied Scholastics, Ferndale WA \end{normalsize}}

\newcommand\mylongdiv[2]{%
$\strut#1$\kern.25em\smash{\raise.3ex\hbox{$\big)$}}$\mkern-8mu
        \overline{\quad\strut#2}$}

\setcounter{secnumdepth}{0}
\begin{document}
\maketitle
\pagebreak
\tableofcontents

\pagebreak

\section{Division}

Division means separating a thing into two or more equal parts.\\

The symbol for division "$\div$"is two dots separated by a line. That is because division is like a fraction. $12 \div 3 = 4$ is the same as $\frac{12}{3}=4$.\\

The $\div$ symbol is only used in beginning arithmetic. Division is more usually written with a "/" symbol or it is written as a fraction. $3 \div 4$ can be written as $3/4$ or as $\frac{3}{4}$ and it means the same thing. The $\div$ symbol can mean other things in other countries, and other symbols are used for division.\\

The number being divided is called the dividend. You will also hear the word dividend in business where it means the amount of profit that is to be divided between owners of a company.\\

The number it is being divided by is called the divisor.\\

The result of division is called the quotient. Quotient is Latin for "how many times?" It comes from asking how many times you can subtract some number from a larger multiple of that number, or how many groups of a certain size can be made.\\

$12 \div 3 = 4$ means there are 4 times that you can subtract 3 from 12, or that you can make 4 groups of 3 from 12 things.\\

You can also think of the quotient as how many times some number "goes into" some larger multiple of that number. $12 \div 3 = 4$ means that 3 goes into 12 4 times.\\

Sometimes a quotient is not a whole number. Any amount remaining after division is called the remainder. $13 \div 4 = 3$, with a remainder of 1. The remainder can be written as itself or, more often, it is written as a fraction, as in $13 \div 4 = 3 \frac{1}{4}$.

$$\textrm{dividend} \div \textrm{divisor} = \textrm{quotient} + \textrm{remainder}$$
$$\textrm{or}$$
$$\textrm{dividend} \div \textrm{divisor} = \textrm{quotient} \frac{\textrm{remainder}}{\textrm{divisor}}$$\\

\section{Two sorts of Division}

There are two ways of thinking about division. One is used when sharing things equally and the other is used when working out how many things can be made of certain size.\\

\subsection*{Partition Division}

Partition means to divide something into parts. A famous example of partitioning is the partitioning of India when India was split into two to create India and Pakistan.\\

Partition division divides an amount into some number of equal sized parts and counts the size of each part.\\

You could say that $20 \div 5 = 4$ because the size of each part, when you divide 20 into 5 equal parts, is 4.\\

The answer to partition division can be any number, not just a whole number.\\

\subsection*{Quotition Division}

Quotition means "how many parts?" Quotition division divides an amount into parts of a certain size and counts how many parts.\\

You could say that $20 \div 5 = 4$ because there are 4 parts when you divide 20 into groups of 5.\\

Quotition division is repeated subtraction. It is how many times a number "goes into" another number. $20 \div 5 = 4$ means $20 \underbrace{- 5 - 5 - 5 - 5}_{4 \times 5} = 0$.\\

The result of quotition division will be a whole number plus any remainder.

\paragraph{Division is the opposite of Multiplication}

You can see here that division, repeated subtraction, is the opposite of multiplication, which is repeated addition.\\

\pagebreak

\section{Testing for Divisibility}

There are some simple tests that can be done first to see if a divisor goes evenly into a dividend, without working out the full division.\\

\subsubsection*{Prime Factors of the Dividend}

One method is to make a prime factor tree of the dividend which will be evenly divisible by any of the prime factors or any product of those prime factors, but not evenly divisible by any other number.\\

For example,\\

\begin{center}
\begin{tikzpicture}
  [level distance=1cm,
  level 1/.style={sibling distance=2cm},
  level 2/.style={sibling distance=2cm}]
  \node {24}
    child {node {2}}
    child {node {12}
      child {node {2}}
      child {node {6}
        child {node {2}}
        child {node {3}}}
    };
\end{tikzpicture}
\end{center}

$24 = 2 \times 2 \times 2 \times 3$, so 24 is only divisible by 1 and 24, by 2, by 3, by $2 \times 2 = 4$, by $2 \times 2 \times 2 = 8$, by $2 \times 2 \times 3 = 12$, and by $2 \times 3 = 6$.\\

\pagebreak

\subsubsection*{Divisibility Rules}

\renewcommand{\arraystretch}{1.1}
\begin{tabular}{p{1em}p{\textwidth}}

1.& \RaggedRight{all whole numbers are divisible by 1.}\\

2.& \RaggedRight{is the last digit even?}\\

3.& \RaggedRight{is the sum of its digits divisible by 3?}\\

4.& \RaggedRight{are the last two digits divisible by 4?}\\

   &\RaggedRight{is the ones digit plus two times the tens digit divisible by 4?}\\

   &\RaggedLeft{e.g. $1,036: 6 + 2 \times 3 = 12: 12 = 3 \times 4$ \Checkmark}\\

5.& \RaggedRight{is the last digit 0 or 5?}\\

6.& \RaggedRight{is it divisible by both 2 and 3?}\\

   &\RaggedRight{is it an even number with the sum of its digits being 0, 3 or 6?}\\

7.& \RaggedRight{is 5 times the ones digit plus the rest of the number a multiple of 7?}\\

   &\RaggedLeft{e.g. $18,123: (5 \times 3) + 1,312 = 1827$ \newline
   $1827: (5 \times 7) + 182 = 217$ \newline
   $217: (5 \times 7) + 21 = 56 = 5 \times 7$ \Checkmark}\\

8.& \RaggedRight{is the ones digit plus two times the rest of the number divisible by 8?}\\

   &\RaggedLeft{e.g. $4,496: 6 + 2 \times 449 = 904 =\newline 800 + 80 + 24 = 113 \times 8$ \Checkmark}\\

9.& \RaggedRight{is the sum of its digits divisible by 9?}\\

10.& \RaggedRight{is the last digit 0?}\\

11.& \RaggedRight{is the sum of pairs of its digits divisible by 11?}\\

   &\RaggedLeft{e.g. $98,615: 9 + 86 + 15 = 110 = 10 \times 11$ \Checkmark}\\

12.& \RaggedRight{is it divisible by both 3 and 4?}\\

\end{tabular}

\section{Special Rules for Division}

There are some special rules about some divisions:\\

Any number divided by 1 is unchanged.\\

$$22 \div 1 = 22$$.

Any number divided by itself equals 1.\\

$$22 \div 22 = 1$$.

A number cannot be divided by 0.\\

$$22 \div 0 = \ ???$$

\pagebreak

\section{Checking Division}

\subsubsection*{Rearranging to Multiplication}

To check your answers, rearrange the terms into multiplication instead of division. Say you have worked out that $1,131 \div 87 = 13$. Check that by doing $87 \times 13$, which should equal 1,131.\\

If there is a  remainder in your answer, first subtract the remainder from the dividend to make it easily divisible. $12 \div 5 = 2$, with a remainder of 2, so $(12 - 2) \times 2 = 10$ means that your answer was correct.\\

\subsubsection*{Casting Out 9s}

You can check by casting out 9s, as you can for other operations, but for division the results are hard to interpret. You can work around that by converting your division to multiplication.\\

$1081 \div 23 = 47$ so $47 \times 23 = 1081 \rightarrow$ digit sums $2 \times 5 = 1 + 9 = \ $correct.\\

You can cast out 9s of remainders as well or you can subtract the remainders first.\\

\addtocontents{toc}{\protect\pagebreak}

\section{Methods of Division}

\subsection{Division by Multiplication Table}

For smaller numbers it is enough to use the multiplication table to look up the answer for a division.\\

If $7 \times 6 = 42$ then it's easy to look up both $42 \div 6 = 7$ and $42 \div 7 = 6$.\\

\subsection{Division by Grouping}

For larger numbers, with a single-digit divisor, division can be done digit by digit.
$$369 \div 3 = 123$$

It is also possible to do division by groups of digits.
\begin{align*}
276,390 \div 3 & = 27\ 63\ 90 \div 3\\
               & = 9\ 12\ 30 = 91,230
\end{align*}
\begin{align*}
294,216,369 \div 3 & = 294\ 216\ 369 \div 3\\
                   & = 98\ 072\ 123 = 98,072,123
\end{align*}

\pagebreak

\subsection{Division by Adding or Subtracting\\into Evenly Divisible Parts}

A hard division problem can be made into an easy one by writing a dividend or divisor as a sum or difference with terms that are now evenly divisible.

$$894 \div 3 = (900 - 6) \div 3 = 300 - 2 = 298$$
$$237 \div 9 = (240 - 3) \div 4 = 60 - \frac{3}{4} = 59 \frac{1}{4}$$
$$407 \div 6 = (360 + 47) \div 6 = 60 + 7 \frac{5}{6} = 67 \frac{5}{6}$$\\

Also, you can use facts like $25 = 100 \div 4$ so you can do $232 \times 25 = 23200 \div 4$ which, by easy short division, = 5800\\

\newpage
\subsection{Division by Factors}

Division can be done by doing it in stages, dividing by factors of the divisor.

\begin{align*}
322 \div 36 & = 322 \div (6 \times 6) = 322 \div 6 \div 6\\
            & = (300 + 22) \div 6 \div 6\\
            & = (50 + \frac{22}{6}) \div 6 = 8 \frac{2}{6} + \frac{22}{36}\\
            & = 8 \frac{12}{36} + \frac{22}{36} = 8 \frac{17}{18}
\end{align*}

\begin{align*}
665 \div 156 & = 665 \div (3 \times 4 \times 13)\\
             & = 665 \div 3 \div 4 \div 13\\
             & = (660 + 5) \div 3 \div 4 \div 13\\
             & = (220 + \frac{5}{3}) \div 4 \div 13\\
             & = (55 + \frac{5}{12}) \div 13\\
             & = \frac{55}{13} + \frac{5}{156} = \frac{8580}{2028} + \frac{65}{2028} = \frac{8645}{2028} = 4 \frac{41}{156}\\
\end{align*}

\newpage
\subsection{Division by Repeated Subtraction}
As you have seen, division is just repeated subtraction, and that can be used to do division without having a total grasp of times tables and multiples. You keep subtracting the divisor, or some multiple of the divisor, until nothing is left.\\

There is a special way of writing division when working with larger numbers. Write the divisor, followed by a right bracket, followed by the dividend with a line over it, and with the quotient written above the line.

\begin{center}
\begin{tabular}{rrrrr}
 & &5&\\
 &28&\overline{) 140}&\\
 &-&28&\ 1 \\
\cline{2-3}
 &&112& \\
 &-&28&\ 1 \\
\cline{2-3}
 & &84& \\
 &-&28&\ 1 \\
\cline{2-3}
 & &56& \\
 &-&28&\ 1 \\
\cline{2-3}
 & &28& \\
 &-&28&\ \underline{+ 1}\\
\cline{2-3}
 & & 0&\ = 5\\
\end{tabular}
\end{center}

\newpage
Using multiples of the divisor is a bit more streamlined. Say we want to know $324 \div 36$. You can fairly easily work out that twice 36 is 72, and twice 72 is 144, and twice 144 is 288. Doubling 288 is a bit hard so lets stay with using 288.\\

\begin{center}
\begin{tabular}{rrrr}
 & &9&\\
 &36 &\overline{)324}&\\
 &-&288& \leftarrow (8 \times 36)\\
\hline
 & &36&\\
 &-&36& \leftarrow (1 \times 36)\\
\hline
 & &0&
\end{tabular}
\end{center}

36 was subtracted from 324 a total of 9 times, in other words $324 \div 36 = 9$.\\

\newpage
\subsection{Short Division}
Short division is a way of dividing any large number by a single-digit divisor. It requires no more than a knowledge of the times table.\\

Short division is also written with the divisor, a right round bracket, the dividend with a line over it, and the quotient written above the dividend. It is called short division because it is all done on one line.\\

Starting at the left digit of the divisor, if the divisor is less than the dividend, divide that digit by the divisor and write the quotient above that digit. If the divisor is greater than the dividend then include the next digit of the dividend, do that division, and write the quotient above that digit.\\

\newpage
\begin{center}
\mylongdiv{7}{2,296}\\
\end{center}

In this example, 7 is greater than 2 so include the next digit of the dividend to get 22. $22 \div 7 = 3$ with a remainder of 1. Write the 3 above the 22.
\begin{center}
\hspace{3.5ex}3\\
\mylongdiv{7}{2,296}\\
\end{center}

Any remainder is written, small, as a tens digit to the left of the next digit of the divisor.
\begin{center}
\hspace{3ex}3\\
\mylongdiv{7}{2,2{^1}96}\\
\end{center}

The whole dividend is treated this way until the final quotient is calculated.
\begin{center}
\hspace{5.8ex}3\hspace{0.8ex}2\hspace{0.8ex}8\\
\mylongdiv{7}{2,2{^1}9{^5}6}\\
\end{center}

\subsubsection{Remainders}
If there is still a remainder, write a decimal point after the units digits of the quotient and the dividend, pad the dividend with as many extras zeroes as needed, and continue the division to get a decimal fraction.

\begin{center}
\hspace*{4.2ex}1\hspace{1ex}1\hspace{0.9ex}2\hspace{0.3ex}.\hspace{0.8ex}5\\
\mylongdiv{6}{6\hspace{1.1ex}7\hspace{0.4ex}{^1}5\hspace{0.3ex}.^30}\\
\end{center}

\newpage
\subsection{Long Division}
Long division is the general purpose method to use for dividing numbers of any length. It is written with the divisor, a right round bracket, the dividend with a line over it, and the quotient written above the dividend. The procedure is similar to short division where multiples of the divisor are subtracted from parts of the dividend, working from left to right.
   
\begin{center}
\mylongdiv{27}{9,855}\\
\end{center}

27 won't go into 9 but 27 will go 3 times into 98, with a remainder of 17. This is written as a subtraction, with the 3 written above the 98.

\begin{center}
\begin{tabular}{cccccccccc}
 & & & & & &3& & &\\
\cline{4-9}
2&7& &)& &9&8&5&5& \\
 & & & &-&8&1& & & \\\cline{5-7}
 & & & & &1&7& & & 
\end{tabular}
\end{center}

Next "bring down" the next digit of the dividend to get a new partial dividend that can now be subtracted from.

\begin{center}
\begin{tabular}{cccccccccc}
 & & & & & &3& & &\\
\cline{4-9}
2&7& &)& &9&8&5&5& \\
 & & & &-&8&1&\downarrow& & \\\cline{5-7}
 & & & & &1&7&5& & 
\end{tabular}
\end{center}

It can be useful to work out the first ten multiples of the divisor. Then you can easily see what is the greatest multiple of the divisor that you can subtract.\\

\begin{tabular}{c|c|c|c|c|c|c|c|c|c}
 1& 2& 3&  4&  5&  6&  7&  8&  9& 10\\
27&54&81&108&135&162&189&216&243&270
\end{tabular}
 
\begin{center}
\begin{tabular}{cccccccccc}
 & & & & & &3&6& & \\
\cline{4-9}
2&7& &)& &9&8&5&5& \\
 & & & &-&8&1& & & \\\cline{5-8}
 & & & & &1&7&5& & \\
 & & & &-&1&6&2& & \\\cline{5-8}
 & & & & & &1&3& & 
\end{tabular}
\end{center}

Bring down the next digit, subtract the greatest multiple of the divisor that will fit, and write the next digit of the quotient on the line above.

\begin{center}
\begin{tabular}{cccccccccc}
 & & & & & &3&6&5& \\
\cline{4-9}
2&7& &)& &9&8&5&5& \\
 & & & &-&8&1& & & \\\cline{5-8}
 & & & & &1&7&5& & \\
 & & & &-&1&6&2&\downarrow& \\\cline{5-9}
 & & & & & &1&3&5& \\
 & & & & &-&1&3&5& \\\cline{6-9}
  & & & & & & & &0&
\end{tabular}
\end{center}

There is no remainder so $9,855 \div 27 = 365$ exactly.

\subsubsection{Remainders}
Remainders in long division can be expressed just as a remainder, or as a fraction, or as a decimal fraction (a fraction expressed in decimal numbers to the right of the decimal point.)\\

\begin{center}
\longdivision{675}{12}
\end{center}

As you can see in this example, when you reach the last digit of the dividend and there is still a remainder, add a decimal point to both the dividend and the quotient and simply continue the procedure.\\

\pagebreak

The decimal fraction part of the quotient will either end with no further remainder, or it will start to repeat itself, or it may continue forever. That is why it can be better to write a remainder as a normal fraction rather than as a decimal fraction.\\

$\frac{22}{7}$ is easy to write as a fraction but, written as a decimal fraction, it goes on forever and can't be written exactly.\\

When a decimal fraction starts to repeat, that is shown by drawing a line above the repeating part of the fraction.You don't have to keep working out a division past that point.\\

\begin{center}
\longdivision{571}{99}
\longdivision{23}{30}
\end{center}

\newpage
\subsection{Division by Repeated\\ Short Division}

This is a version of division by factors.\\

$638 \div 18 = 638 \div (2 \times 3 \times 3) = 638 \div 2 \div 3 \div 3$

\begin{center}
\hspace{3em}35.\overline{4}\\
\begin{tabular}{ll}
&\mylongdiv{3}{106.\overline{3}}\\
&\mylongdiv{3}{319}\\
&\mylongdiv{2}{638}
\end{tabular}
\end{center}

This can be shorter to do than long division, even for fairly large numbers. Here is the same thing done by long division:

\begin{center}
\begin{tabular}{r|r|r|r|r|r|r|r|r|r}
 1& 2& 3& 4& 5&  6&  7&  8&  9& 10\\
18&36&54&72&90&108&126&144&162&180
\end{tabular}
\end{center}

\longdivision{638}{18}

\newpage
\subsection{Division of Fractions}
Fractional divisors or divisors with a fractional part can be solved by converting to decimal fractions and moving the decimal place to the right in both numbers until the divisor is a whole number. The quotient will still be the same.
$$2 \frac{3}{4} \div 4 = 2.75 \div 4 = 275 \div 400 = \frac{11}{16}$$

$$4 \div 2 \frac{3}{4} = 4 \div 2.75 = 400 \div 275 = \frac{16}{11}$$

Of course you can still do long and short division with fractions by remembering to place the decimal point correctly in your answer, but this way can be much simpler.\\

\newpage
\subsection{Galley Division}

Galley division is a very old method. It is called galley division because it looks like a galley, which was a type of ship. It is also called scratch division because it was originally written in sand. It is more compact than long division but it's harder to follow how the answer was reached. It fell out of use because printers couldn't print the cancelled numbers that are used, and long division became the preferred method.\\

There are three versions of galley division: One where the digits are rubbed out and replaced, as you would do if writing in sand, one where digits are crossed out and replaced, as you would do with pen and ink, and one where digits are rewritten without being cancelled, known as the printer's method.\\

\newpage
Here is the procedure:\\

Write the dividend and a vertical bar. Write the divisor under the dividend with the left-most digits aligned. The quotient will be written to the right of the bar.\\

Multiply each digit of the divisor by the quotient digit one at a time and subtract it from the dividend, crossing out digits and replacing them above with the result of the subtraction.\\

Rewrite the divisor one digit to the right, and repeat the same procedure. Keep doing this for all digits of the dividend.\\

The quotient is the number to the right of the bar, and the remainder is any remaining uncancelled numbers along the top to the left of the bar.\\

\newpage
Here is an example:\\

$6284 \div 32$ :\\

Lay out the dividend and divisor:\\

\begin{tabular}{llll|l}
6&2&8&4&\\
3&2&&\\
\end{tabular}\\

\vspace{16pt}
$62 \div 32 = 1$ is the first digit of the quotient.\\

\begin{tabular}{llll|l}
6&2&8&4&1\\
3&2&&\\
\end{tabular}\\

\vspace{16pt}
Now, starting from the left, multiply each digit of the divisor by the quotient digit and subtract that product from the left-most digits (those above the divisor) of the dividend. Cancel used digits and write new digits directly above.\\

$6 - (3 \times 1) = 3$:\\

\begin{tabular}{llll|l}
3&&&&\\
\cancel{6}&2&8&4&1\\
\cancel{3}&2&&&\\
\end{tabular}\\

\newpage
$2 - (2 \times 1) = 0$:\\

\begin{tabular}{llll|l}
3&0&&&\\
\cancel{6}&\cancel{2}&8&4&1\\
\cancel{3}&\cancel{2}&&&\\
\end{tabular}\\

\vspace{16pt}
Now rewrite the divisor one place to the left:\\

\begin{tabular}{llll|l}
         3&         0& & & \\
\cancel{6}&\cancel{2}&8&4&1\\
\cancel{3}&\cancel{2}&2& & \\
          &         3& & & \\
\end{tabular}\\

\vspace{32pt}
\begin{tabular}{c|c|c|c|c|c|c|c|c|c}
  1&  2&  3&  4&  5&  6&  7&  8&  9& 10 \\
 32& 64& 96&128&160&192&224&256&288&320 \\
\end{tabular}\\

32 goes 9 times into 308, which is the next quotient digit.\\

\begin{tabular}{llll|l}
         3&         0& & & \\
\cancel{6}&\cancel{2}&8&4&19\\
\cancel{3}&\cancel{2}&2& & \\
          &         3& & & \\
\end{tabular}\\

\newpage
$30 - (3 \times 9) = 3$:\\

\begin{tabular}{llll|l}
          &         3& & & \\
\cancel{3}&\cancel{0}& & & \\
\cancel{6}&\cancel{2}&8&4&19\\
\cancel{3}&\cancel{2}&2& & \\
          &\cancel{3}& & & \\
\end{tabular}\\

 \vspace{16pt}
 $38 - (2 \times 9) = 20$:\\
 
\begin{tabular}{llll|l}
          &         2& & & \\
          &\cancel{3}& & & \\
\cancel{3}&\cancel{0}&0& & \\
\cancel{6}&\cancel{2}&\cancel{8}&4&19\\
\cancel{3}&\cancel{2}&\cancel{2}& & \\
          &\cancel{3}& & & \\
\end{tabular}\\

\vspace{16pt}
Again, rewrite the divisor one place to the right.\\

\begin{tabular}{llll|l}
          &         2&          & & \\
          &\cancel{3}&          & & \\
\cancel{3}&\cancel{0}&         0& & \\
\cancel{6}&\cancel{2}&\cancel{8}&4&19\\
\cancel{3}&\cancel{2}&\cancel{2}&2& \\
          &\cancel{3}&         3& & \\
\end{tabular}\\

\newpage
32 goes 6 times into 204, the next digit of the quotient:\\

\begin{tabular}{llll|l}
          &         2&          & & \\
          &\cancel{3}&          & & \\
\cancel{3}&\cancel{0}&         0& & \\
\cancel{6}&\cancel{2}&\cancel{8}&4&196\\
\cancel{3}&\cancel{2}&\cancel{2}&2& \\
          &\cancel{3}&         3& & \\
\end{tabular}\\

\vspace{16pt}
$20 - (3 \times 6) = 2$:\\

\begin{tabular}{llll|l}
          &\cancel{2}&          & & \\
          &\cancel{3}&         2& & \\
\cancel{3}&\cancel{0}&\cancel{0}& & \\
\cancel{6}&\cancel{2}&\cancel{8}&4&196\\
\cancel{3}&\cancel{2}&\cancel{2}&2& \\
          &\cancel{3}&\cancel{3}& & \\
\end{tabular}\\

$24 - (2 \times 6) = 12$:\\

\begin{tabular}{llll|l}
          &\cancel{2}&         1&          & \\
          &\cancel{3}&\cancel{2}&          & \\
\cancel{3}&\cancel{0}&\cancel{0}&         2& \\
\cancel{6}&\cancel{2}&\cancel{8}&\cancel{4}&196\\
\cancel{3}&\cancel{2}&\cancel{2}&\cancel{2}& \\
          &\cancel{3}&\cancel{3}&          & \\
\end{tabular}\\

So $6284 \div 32 = 196$, and the remainder is 12.

\newpage
\subsection{Alison's Method}

Students have said that they don't like "bringing down" numbers in long division. Here is a method using repeated subtraction that is done all in one line. It is like short division but explicitly writes out the multiple of the divisor that is being subtracted from each digit of the  dividend.\\

\begin{tabular}{r{2ex}r{2ex}r{2ex}r{2ex}r{2ex}r{2ex}r{2ex}}
 & & 1&   2& 1&   9&   6\\
\cline{2-7}
5&)& 6&^1 0& 9&^4 8&^3 0\\
 & &-5& -10&-5& -45& -30\\
\end{tabular}

\vspace{32pt}
\begin{tabular}{r{2ex}r{2ex}r{2ex}r{2ex}r{2ex}r{2ex}r{2ex}r{2ex}}
  & &   &  4&  0&  6&  5&    \\
\cline{2-7}
15&)&  6&  0&  9&  8&^80&  \\
  & &   &-60&   &-90&-75&   \\
\end{tabular}

\newpage
\subsection{Short Division\\with a multi-digit Divisor}

I have seen that students are no longer being taught long division, but instead are trained to use short division in all cases. The interim steps are not written and must be done mentally. It is only workable for relatively small numbers.\\

Here is an example:\\

\begin{center}
\hspace{6.5ex}1\hspace{0.8ex}3\hspace{0.8ex}5\\
\mylongdiv{17}{2,2{^5}9{^8}6}\\
remainder 1
\end{center}

\newpage
\

\begin{center}
\linespread{2}\large

Enquiries

\textbf{Applied Scholastics Ferndale}

Principal: Paula McLennan

mobile phone: 0431 683 306

email address: apsferndale@gmail.com

website: apsferndale.webs.com
\end{center}

\end{document}
