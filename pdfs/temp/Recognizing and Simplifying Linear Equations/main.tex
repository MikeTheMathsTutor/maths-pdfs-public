\documentclass{article}
\usepackage{amsmath}
\usepackage{multicol}
\usepackage{tikz}

\title{Recognizing and Simplifying Linear Equations}\\
\author{Tutoring Centre Ferndale\\
\includegraphics[width=4em]{ApS_logo.png}}
\date{}

\begin{document}

\maketitle

\section*{Forms of Linear Equations}
A linear equation is an equation that forms a straight line when its solutions are plotted on a graph.

\begin{itemize}
    \item The standard form of a linear equation is:
\[
Ax + By = C
\]
where \(A\), \(B\), and \(C\) are constants, and \(x\) and \(y\) are variables.

\item Sometimes linear equations are in the slope-intercept form:
\[
y=mx+b
\]
where $m$ is the slope of the line and $b$ is the $y$-intercept, the point where the line intercepts the $y$-axis.

\item Another common form is the point-slope form, which is used when the slope of the line and the coordinates of a point on the line are known:
\[
y - y_1 = m(x - x_1)
\]
where \(m\) is the slope of the line and \((x_1, y_1)\) is a point on the line.
\end{itemize}

Any linear equation can be converted to any one of these forms.

\newpage

\section*{Recognizing Linear Equations}
Linear equations can sometimes appear in nonstandard forms. The key to recognizing them is to check if they can be simplified to either the standard form or the slope-intercept form or the point-slope form.

\subsection*{Identifying Linear Equations}
An equation is linear if:
\begin{itemize}
    \item The highest power of any variable is 1.
    \item There are no products of variables (e.g., \(xy\)).
    \item There are no variables inside fractions, squares, or other complicated expressions.
\end{itemize}

\subsection*{Examples}
Let's look at some examples to identify and simplify linear equations.

\subsubsection*{Example 1}
\[
2x + 3y = 6
\]
This is already in standard form. Therefore, it is a linear equation.

\subsubsection*{Example 2}
\[
y = 4x - 5
\]
This is in the slope-intecept form $y=mx+b$ so it is a linear equation.

This can be rewritten as:
\[
4x - y = 5
\]
Now, it is in standard form \(Ax + By = C\).

\subsubsection*{Example 3}
\[
2(x - 3) = y + 1
\]
First, simplify the equation:
\[
2x - 6 = y + 1
\]
Rearrange to standard form:
\[
2x - y = 7
\]
Thus, it is a linear equation.

You could also simply note that it was already in the point-slope form \[y - y_1 = m(x - x_1)\] and is therefore a linear equation.

\subsubsection*{Example 4}
\[
y = \frac{2x + 3}{4}
\]
Multiply both sides by 4 to clear the fraction:
\[
4y = 2x + 3
\]
Rearrange to standard form:
\[
2x - 4y = -3
\]
So, it is a linear equation.

\section*{Exercises}
Determine if the following equations are linear. If they are, convert them to standard form \(Ax + By = C\).

\begin{multicols}{2}
\begin{enumerate}
    \item \(3x + 4y = 12\)
    \item \(y = 5x - 7\)
    \item \(2(x + 1) = y - 3\)
    \item \(y = \frac{3x + 2}{2}\)
    \item \(x^2 + y = 1\)
    \item \(y = \frac{1}{x} + 2\)
    \item \(6 = 2x - 3y\)
    \item \(\frac{x - 2y}{3} = 1\)
\end{enumerate}
\end{multicols}

\newpage

\section*{Answers to Exercises}
\begin{enumerate}
    \item \(3x + 4y = 12\) (Already in standard form)
    \item \(y = 5x - 7 \Rightarrow 5x - y = 7\)
    \item \(2(x + 1) = y - 3 \Rightarrow 2x + 2 = y - 3 \Rightarrow 2x - y = -5\)
    \item \(y = \frac{3x + 2}{2} \Rightarrow 2y = 3x + 2 \Rightarrow 3x - 2y = -2\)
    \item \(x^2 + y = 1\) (Not a linear equation)
    \item \(y = \frac{1}{x} + 2\) (Not a linear equation)
    \item \(6 = 2x - 3y \Rightarrow 2x - 3y = 6\) (Already in standard form)
    \item \(\frac{x - 2y}{3} = 1 \Rightarrow x - 2y = 3\)
\end{enumerate}

\end{document}
