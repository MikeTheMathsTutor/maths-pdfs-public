\documentclass{article}
\usepackage{amsmath,amsfonts}
\usepackage[fontsize=14pt]{fontsize}

\begin{document}

\title{Fractions Worksheet}
\author{}
\date{}}

\begin{center}
\begin{Large}
\textbf{Fractions Worksheet}\\
\end{Large}
\end{center}

\section*{Fractions}
\begin{enumerate}
    \item Define 'Fraction'
    \item Define 'Numerator'
    \item Define 'Denominator'
\end{enumerate}

\section*{Equivalent Fractions}

\begin{enumerate}
    \item Define 'Equivalent'
    \item Define 'Express'
    \item What are equivalent fractions?
    \item What does 'multiplying by 1' mean?
    \item How do you make equivalent fractions?
    \item Are $\frac{3}{4}$ and $\frac{6}{9}$ equivalent fractions?
    \item Write 3 fractions that are equivalent to $\frac{2}{3}$
\end{enumerate}

\section*{Simplifying Fractions}

\begin{enumerate}
    \item Define 'Simplify'
    \item What is a simplified fraction?
    \item Define 'Factor'
    \item What is a 'Greatest Common Factor'?
    \item What is 'cancelling'?
    \item How are Greatest Common Factors used?
    \item How does knowing the times table help in finding GCFs?
    \item How can you use any common factor that you spot to simplify a fraction?
    \item How is listing out all factors used in simplifying fractions?
    \item Define 'Prime Number'
    \item Define 'Composite Number'
    \item Define 'Prime Factor'
    \item Write a prime factor tree for 120
    \item Write 120 as a product of primes
    \item How can finding prime factors be used in simplifying fractions?

    \item Simplify: $\frac{12}{18}$
    \item Simplify: $\frac{25}{50}$
    \item Simplify: $\frac{48}{64}$
    \item Simplify: $\frac{16}{24}$
    \item Simplify: $\frac{35}{70}$
\end{enumerate}

\section*{Proper, Improper,\\and Mixed Fractions}

\begin{enumerate}
    \item Define 'Proper Fraction'
    \item Define 'Improper Fraction'
    \item How do you change a whole number into an improper fraction?
    \item Define 'Mixed Fraction'
    \item What should you do if the result of your calculation is an improper fraction?
    \item How do you change an improper fraction into a mixed fraction?
    \item How do you change a mixed fraction into an improper fraction?
    \item What is $\frac{11}{5}$ as a mixed fraction?
    \item What is $2 \frac{7}{8}$ as an improper fraction?
    \item What is $\frac{56}{9}$ as a mixed fraction?
    \item Write 7 as an improper fraction
\end{enumerate}

\section*{Multiplying Fractions}

\begin{enumerate}
    \item How do you multiply fractions?
    \item Write the appropriate maths statements to solve the following: "How many males are in a class of twenty students where two thirds of the class are male?"
    \item Why simplify fractions before multiplying?
    \item What is cross-cancelling?
    \item How do you multiply mixed fractions?

    \item $\frac{9}{15} \times \frac{3}{9}$
    \item $\frac{2}{3} \times \frac{4}{5}$
    \item $\frac{5}{8} \times \frac{2}{3}$
    \item $\frac{6}{8} \times \frac{1}{6}$
    \item $\frac{2}{5} \times \frac{5}{7}$
    \item $\frac{5}{10} \times \frac{3}{5}$
    \item $3\frac{3}{4} \times 4\frac{2}{3}$
\end{enumerate}

\section*{Dividing Fractions}

\begin{enumerate}
    \item How do you divide fractions?
    \item How do you divide mixed fractions?
    
    \item $\frac{3}{4} \div \frac{1}{2}$
    \item $\frac{5}{6} \div \frac{2}{3}$
    \item $\frac{7}{8} \div \frac{1}{4}$
    \item $\frac{2}{5} \div \frac{4}{7}$
    \item $\frac{3}{4} \div \frac{5}{6}$
    \item $3\frac{3}{4} \div 4\frac{2}{3}$
\end{enumerate}

\section*{Comparing Fractions}

\begin{enumerate}
    \item Why can't you compare fractions with different denominators?
    \item What is cross-multiplying?
    \item Compare: $\frac{3}{5} \text{ and } \frac{7}{10}$
    \item Compare: $\frac{4}{9} \text{ and } \frac{5}{12}$
    \item Compare: $\frac{2}{7} \text{ and } \frac{3}{8}$
    \item Compare: $\frac{1}{3} \text{ and } \frac{2}{5}$
    \item Compare: $\frac{5}{6} \text{ and } \frac{5}{8}$
\end{enumerate}

\section*{Adding Fractions}

\begin{enumerate}
    \item Define 'Common Denominator'
    \item Define 'Lowest Common Denominator'
    \item Define 'Lowest Common Multiple'
    \item Why can't you add or subtract fractions that have different denominators?
    \item How is 1 expressed a a fraction used to make equivalent fractions with the same denominators?
    \item What are some different ways of finding the lowest common denominator?
    \item How do you cross-multiply to add and subtract fractions?
    \item $\frac{2}{3} + \frac{5}{6}$
    \item $\frac{3}{4} + \frac{1}{8}$
    \item $\frac{7}{12} + \frac{2}{5}$
    \item $\frac{5}{16} + \frac{9}{16}$
    \item $\frac{2}{9} + \frac{4}{9}$
\end{enumerate}

\section*{Subtracting Fractions}

\begin{enumerate}
    \item $\frac{5}{8} - \frac{1}{4}$
    \item $\frac{7}{10} - \frac{3}{5}$
    \item $\frac{9}{16} - \frac{3}{8}$
    \item $\frac{11}{15} - \frac{2}{5}$
    \item $\frac{4}{7} - \frac{2}{7}$
\end{enumerate}

\end{document}
