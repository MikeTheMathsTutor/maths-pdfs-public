\documentclass{article}

\usepackage{amsmath}
\usepackage[a4paper,margin=2cm]{geometry}
\usepackage{textcomp}
\usepackage{xspace}
\usepackage{verbatim}
\usepackage{cancel}
\usepackage{setspace}

\parskip 4pt
\begin{document}
	
	\author{Mike McLennan\\
		Applied Scholastics Ferndale\\
		Perth, Western Australia 6148\\}
	\date{\today}
	\title{Powers, Exponents, Indices, Roots, Surds, Logarithms}
	\maketitle

\begin{abstract}
	This document is an overview of Powers of Numbers and other topics that derive from that subject. It includes notes on the history of some topics as well as derivations of some of the words involved. It is intended for use in instruction and it is planned to be be expanded on with more examples and exercises into a checksheet for students to work through. I am working on a document on earlier arithmetic that will lead into this one.
\end{abstract}

\section*{Notation}

In most mathematics beyond arithmetic, instead of using the ‘ × ‘ multiplication symbol, a dot ‘ $\cdot$ ‘ is often used to mean multiplication, or if no symbol is used at all then it is implied by default that multiplication is intended.

e.g.	{abc and ${a\cdot b \cdot c}$ means the same as a $\times$ b $\times$ c.}

\tableofcontents
\newpage

\section{Powers}
Powers are numbers multiplied by themselves a given number of times. The number being multiplied is called the \textit{base}, and the number of times that it is multiplied by itself is called the \textit{power} (or the \textit{index}, or the \textit{exponent}.)

The power or index or exponent is written as a small raised number to the right of the base.

The word \textit{power} is from Latin \textit{potentias} which is a mistranslation of the Greek word \textit{dynamis} meaning \textit{amplification} used by the mathematician Euclid around 300BC for the square of a line.

The word \textit{exponent} was coined in a book from 1544 called \textit{Arithmetica Integra}. It is from Latin \textit{expo}, out of, and \textit{ponere}, place, with the idea of laying something out to view it's parts.

\textit{Exponent} is the preferred term in the US but the rest of the world prefers the terms \textit{index} or \textit{power}.

\textit{Index} is a Latin word meaning pointer, and the plural of \textit{index} is \textit{indices}. The use of the term index derived from 'pointing' to which of the powers of a number was meant. For example, the powers of 3 are $3^1=3, 3^2=9, 3^3=27, 3^4=81 and 3^5=273$ so you could say that the $4^{th}$ power of 3 is 81, with 4 being the pointer to that particular power.

The word power is also used to mean the result of raising a number to a power, such as when we say that 8 is a power of 2 because $2^3=8$.

If we use \textit{a} as the base and \textit{n} as the exponent, it is read as the \textit{$n^{th}$ power of a}, or as \textit{a to the power of n}, or as \textit{a the the $n^{th}$ power}.

e.g.	a to the power of n is a times a, n times.
	${a^n = a\cdot a\cdot a \cdot ... \cdot a}$ \{for n factors\}
	${3^5 = 3 \cdot 3 \cdot 3 \cdot 3 \cdot 3 = 243}$

In writing computer programs, and in some maths programs, because they use only standard text characters and can’t write the small raised n, the “caret” symbol, \^\, is used to write the power of a number.

e.g.	$3 \string^ 5 = 3 \cdot 3 \cdot 3 \cdot 3 \cdot 3 = 243$

\section{Roots}

The \textit{base} can also be called the \textit{root}, particularly when given some number and you are looking for the number which when multiplied by itself a given number of times will result in that number.

The Latin word for \textit{root} is \textit{radical} and that word is also sometimes used to mean the root. It is also thought to be where the radical symbol \textsurd \enspace comes from, being a sort of stretched out letter \textit{r}.

The number to the left of the radical symbol that indicates which root is to be taken and it is called the \textit{index} or the \textit{order} or the \textit{degree} of the root.

The value inside the radical symbol is known as the \textit{radicand}.

e.g.	$\sqrt[n]{a}$ is read as ‘the $n^{th}$ root of a’

e.g.	$\sqrt[5]{243}=3$ is read as ‘the $5^{th}$ root of 243 equals 3.’
(notice that $3^5 = 243$)

The second root of a number is also called its ‘square’ root, because the area of a square is given by the product of the lengths of its sides. The number is usually not written. If a radical symbol has no number given, then it is implied to be a square root that is meant.

e.g.	$\sqrt{a}$ is read as ‘the square root of a.’

The third root of a number is also called its ‘cube’ root, because the volume of a cube is given by the product of the lengths of its three dimensions.

e.g.	$\sqrt[3]{a}$ is read as ‘the cube root of a.’

\newpage

\section{Power Laws}

The powers of numbers follow some useful laws:

\subsection{Unit Power Law}
\begin{center}
\begin{Large}
$$a^1=a$$
\end{Large}
\text{e.g. }$3^1=1$
\end{center}

\subsection{Product Law}
\begin{Large}
$$a^m\times a^n=a^{m+n}$$
\end{Large}
\begin{align*}
\text{e.g. }{3^2 \times 3^3}&={3^{2+3} = 3^5}\\
&={3 \cdot 3 \times 3 \cdot 3 \cdot 3}\\
&={3 \cdot 3 \cdot 3 \cdot 3 \cdot 3} = 3^5
\end{align*}

\subsection{Quotient Law}
\begin{Large}
$$\frac{a^m}{a^n}=a^{m-n}$$
\end{Large}
\begin{align*}
\text{e.g. }\frac{3^5}{3^3}&=3^{5-3}=3^2\\
&=(3 \times 3 \times 3 \times 3 \times 3) \div (3\times 3 \times 3 )\\
&=\frac{3 \cdot 3 \cdot \cancel{3 \cdot 3 \cdot 3}}{\cancel{3 \cdot 3 \cdot 3}}=3 \cdot 3=3^2
\end{align*}

\subsection{Power of a Power Law}
\begin{Large}
$$(a^m)^n=a^{m \times n}$$
\end{Large}
\begin{align*}
\text{e.g. }(3^2)^3&=3^{2 \times 3}=3^6\\
&=3 \cdot 3 \times 3 \cdot 3 \times 3\cdot 3=3 \cdot 3 \cdot 3 \cdot 3 \cdot 3 \cdot 3=3^6
\end{align*}

\subsection{Power of a Product Law}
\begin{Large}
$$(a \times b)^m=a^m \times b^m$$
\end{Large}
\begin{center}
\text{e.g. }$(3 \cdot 3)^2=3^2 \cdot 3^2=3^{2+2}=3^4$
\end{center}

\subsection{Power of a Quotient Law}
\begin{Large}
$$\left(\frac{a}{b}\right)^m=\frac{a^m}{b^m}$$
\end{Large}
\begin{center}
\text{e.g. }$\left(\frac{3}{9}\right)^2=\frac{3^2}{9^2}=\frac{9}{81}=\frac{1}{9}$
\end{center}

\newpage

\subsection{Zero Power Law}
\begin{Large}
$$a^0=1$$
\end{Large}
\begin{center}
\text{e.g. }$3^0=1$
\end{center}
Anything to the power of 0 is equal to 1.\\
\\
Here are a few explanations of the Zero Power Law:
\subsubsection{by the Quotient Law}
\begin{align*}
\frac{n^x}{n^x}&=\frac{\cancel{n^x}}{\cancel{n^x}}  =1\\
&=n^{x-x}=n^0
\end{align*}
Anything divided by itself equals 1 but, by the quotient law, a number raised to a power and divided by itself is equal to that number raised to the power 0. (For any number other than $0^0$.)

\subsubsection{by the Multiplicative Identity}
\begin{align*}
n^x&=1 \cdot \underbrace{n \cdot \ldots \cdot n}_{\text{n times}}\\
\end{align*}
The number 1 is known as the ‘multiplicative identity’ because anything multiplied by 1 results in the same (identical) number. Zero is the ‘additive identity’ because 0 added to anything results in the same number. The ‘1×’ or ‘0+’ is always there but not usually written.

If you apply the fact that 1 times any number equals the same number then you see that $n^0$ is just 1 multiplied by n, zero times, which is 1.

\subsubsection{by Dividing by the Base}
A number to a power, divided by its base, equals that number
	to one less power:
\begin{align*}
x^{n-1}&=\frac{x^n}{x}\\
\\
\text{e.g. }2^3&=\frac{2^4}{2}=\frac{2 \times 2 \times 2 \times \cancel{2}}{\cancel{2}}=2 \times 2 \times 2=2^3\\
\\
\text{So, replacing n with 0 in this formula:}\\
x^0&=x^{1-1}=\frac{x^1}{x}=\frac{x}{x}=1
\end{align*}

\newpage

\subsection{Negative Power Law}
\begin{Large}
$$a^{-n}=\frac{1}{a^n}$$
\begin{center}
also
\end{center}
$$\frac{1}{a^{-n}}=a^n$$
\end{Large}
\begin{center}
\text{e.g. }$3^{-2}=\frac{1}{3^2}=\frac{1}{9}$\\
\end{center}

\noindent
Here are some explanations of the Negative Exponent Law:

\subsubsection{by the Quotient Law}
\begin{align*}
a^{-n}=a^{0-n}=\frac{a^0}{a^n}=\frac{1}{a^n}
\end{align*}

\subsubsection{by Arithmetic}
\begin{center}
\begin{tabular}{rr}
&\text{e.g. }$\frac{a^2}{a^5}=a^{2-5}=a^3$\\\\
&$\frac{a \cdot a}{a \cdot a \cdot a \cdot a \cdot a}
=\frac{\cancel{a \cdot a}}{\cancel{a \cdot a} \cdot a \cdot a \cdot a}
=\frac{1}{a \cdot a \cdot a}=\frac{1}{a^3}$
\end{tabular}
\end{center}

\subsubsection{by the Multiplicative Identity}
A positive exponent is how many times to multiply by a number. The multiplicative identity, 1×, is always there but not usually written.
\begin{align*}
a^n&=1 \underbrace{\times a \times \ldots \times a}_{\text{n times}}\\
\text{e.g. }2^3&=1 \underbrace{\times 2 \times 2 \times 2}_{\text{3 times}}
\end{align*}

If you include the multiplicative identity then you can see that a negative exponent is just how many times to divide by a number, the opposite of multiplication, starting at 1.

\begin{align*}
a^{-n}&=1 \underbrace{\div a \div \ldots \div a}_{\text{n times}}\\
\text{e.g. }2^{-3}&=1 \underbrace{\div 2 \div 2 \div 2}_{\text{3 times}}=\frac{1}{2^3}=\frac{1}{8}
\end{align*}

\newpage

\subsection{Reciprocal Power law}
\begin{Large}
$$a^{\frac{1}{n}}=\sqrt[n]{a}$$
\end{Large}
\begin{center}
\begin{align*}
\text{e.g. }4^{\frac{1}{2}}&=\sqrt[2]{4}=2\\
\text{e.g. }32^{\frac{1}{5}}&=\sqrt[5]{32}=2
\end{align*}
\end{center}

\subsubsection{Definition of Reciprocal}
A reciprocal means the inverse of a fraction or a fraction that has been flipped the other way around.
For example, $\frac{3}{4}$ and $\frac{4}{3}$ are reciprocal fractions.
A fraction multiplied by it’s reciprocal equals 1.
For example, $\frac{1}{2} \times \frac{2}{1}=1$.\\

\noindent
Here is an explanation of the Negative Exponent Law:

\subsubsection{by the Power of a Power Law}
\begin{align*}
(a^{\frac{1}{n}})^{\frac{n}{1}}=a^{\frac{1}{n} \times \frac{n}{1}}=a^{\frac{n}{n}}=a^1=a
\end{align*}
A number raised to the reciprocal of a power, that is then raised to that power, gives the original number.

Expanding this out, you can see that $a^{\frac{1}{n}}$ must be multiplied by itself $n$ times to equal $a$:
\begin{align*}
&(a^{\frac{1}{n}})^n
=\underbrace{a^{\frac{1}{n}} \times a^{\frac{1}{n}} \times \ldots \times a^{\frac{1}{n}}}_{\text{n times}}
=a^{\underbrace{\frac{1}{n}+\frac{1}{n}+\ldots+\frac{1}{n}}_{\text{n times}}}
=a^{\frac{n}{n}}=a^1=a\\&
\end{align*}

and that $a^{\frac{1}{n}}$ is actually the $n^{th}$ root of $a$: $a^{\frac{1}{n}}=\sqrt[n]{a}$.

\subsection{Fractional Power Law}
(Also known as the ‘rational’ power law because ‘rational’ means to do with ratios, i.e. fractions.)

\begin{Large}
$$a^{\frac{m}{n}}=\sqrt[n]{a^m}=(\sqrt[n]{a})^m$$
\end{Large}
\begin{center}
\text{e.g. }$3^{\frac{2}{3}}=\sqrt[3]{3^2}\text{  }(=\sqrt[3]{9} ) =(\sqrt[3]{3})^2$
\end{center}
Here is an explanation of the Fractional Power Law:
\subsubsection{by using the Power of a Power Law and the Reciprocal Power Law}
Separating the fraction into numerator and denominator, so that $\frac{m}{n}=m \times \frac{1}{n}$, you can use the Power of a Power Law and the Reciprocal Power Law:
\begin{align*}
a^{\frac{m}{n}}&=a^{m \times \frac{1}{n}}=(a^m)^{\frac{1}{n}}=\sqrt[n]{a^m}\\
\text{and }a^{\frac{m}{n}}&=a^{\frac{1}{n} \times m}=(a^{\frac{1}{n}})^m=(\sqrt[n]{a})^m\\
\end{align*}

\newpage

\subsubsection{Negative Fractional Powers}
\begin{Large}
$$a^{\frac{-m}{n}}=1/a^{\frac{m}{n}}=\frac{a}{(\sqrt[m]{a})^n}$$
\end{Large}
\begin{center}
\text{e.g. } $3^{\frac{-2}{3}}=1/3^{\frac{2}{3}}=\frac{1}{(\sqrt[2]{3})^3}\approx \frac{1}{1.73^3}\approx\frac{1}{5.20}\approx0.19$
\end{center}

\subsubsection{Fractions with Negative Powers}
\begin{Large}
$$(\frac{a}{b})^{-n}=\frac{1}{({\frac{a}{b}})^n}=\frac{1}{\frac{a^n}{b^n}}=\frac{b^n}{a^n}$$
\end{Large}
\begin{center}
\text{e.g. }
$(\frac{2}{3})^{-2}=\frac{1}{({\frac{2}{3}})^2}=\frac{1}{\frac{2^2}{3^2}}=\frac{3^2}{2^2}=\frac{9}{4}$
\end{center}

\subsubsection{Multiplication of Negative Powers}
different powers:
\begin{Large}
$$a^{-m} \times a^{-n}=a^{-(m+n)}=\frac{1}{a^{m+n}}$$
\end{Large}
\begin{center}
\text{e.g. }
$2^{-2} \times 2^{-3}=2^{-(2+3)}=2^{-5}=\frac{1}{2^5}=\frac{1}{32}$
\end{center}

different bases:
\begin{Large}
$$a^{-n}\times b^{-n}=(a\times b)^{-n}$$
\end{Large}
\begin{center}
\text{e.g. }
$2^{-3}\times 3^{-3}=(2\times 3)^{-3}=6^{-3}= \frac{1}{6^3}=\frac{1}{216}$
\end{center}

different bases and powers:
\begin{Large}
$$a^{-m} \times b^{-n}\text{\normalsize (calculate separately)}$$
\end{Large}
\begin{center}
\text{e.g. }
$2^{-3} \times 3^{-2}=\frac{1}{2^3} \times \frac{1}{3^2}=\frac{1}{8} \times \frac{1}{9}=\frac{1}{72}$
\end{center}

division of negative powers:
\begin{Large}
$$a^{-m} \div a^{-n}=a^{-m-(-n)}=a^{-m+n}$$
\end{Large}
\begin{center}
\text{e.g. }
$3^{-3} \div 3^{-2}=3^{-3-(-2)}=3^{-3+2}=3^{-1}=\frac{1}{3^1}=\frac{1}{3}$
\end{center}

\newpage

\subsection{Powers of a Negative Number}
A negative number multiplied by a negative number results in a positive product. When that positive product is multiplied by a negative number, the result is a negative product. And so the sign alternates depending on whether the power is an odd or an even number.\\

\noindent
A negative number taken to an even power gives a positive result.\\
\indent\text{e.g. }$(-4)^4 = -4 \times -4 \times -4 \times -4 = 256$

\noindent
A negative number taken to an odd power gives a negative result.

\text{e.g. }$(-4)^5 = -4 \times -4 \times -4 \times -4 \times -4 = 1024$

\noindent
This also applies when looking for the roots of a number:

\subsubsection{Roots of an Even Power of a Negative Number}

There is no $n^{th}$ root of an even power of a negative number.

e.g.	No number can be multiplied by itself to find $\sqrt{-16}$.

\subsubsection{Roots of an Odd Power of a Negative Number}
You can, however, find the $n^{th}$ root of an odd power.

e.g.	$-3 \times -3 \times -3 = -27, so \sqrt[3]{-27}= -3$.

\subsubsection{Brackets}
Be careful with brackets, e.g.
$-3^2 = -(3 \times 3)=-1 \times 3 \times 3 = -9$
, but$(-3)^2 = -3 \times -3 = 9.$

\section{Scientific Notation}
(also called Exponential Notation, or Standard Form)

Values in science can range from very large to very small. To make these numbers shorter they are usually expressed as multiples of some power of ten.

For example, the speed of light is approximately 300,000,000 meters per second, but it is usually written more briefly as $3x10^8$ m/s, or 3E+8 m/s.

Similarly for very small values, the weight of an electron has been measured as\\0.0000000000000000000000000009109 kilograms but that is much more briefly written as $9.109x10^{-31}$ kg, or 9.109E-31 kg.

\newpage

\section{Surds}

A rational number is a number that can be written as a ratio of two whole numbers. $1, \frac{2}{3}, -4, \frac{273}{8}$ are all rational numbers. Numbers that can’t be written as the ratio of two whole numbers are called irrational numbers. For example, the relationship between the circumference of a circle and its diameter, known by the Greek letter $\pi$ (pi), which stands for ‘perimeter,’ cannot be written exactly as a ratio of any two whole numbers so it is called an irrational number.

Surds are numbers that can only be written as the root of an integer because they are not a rational number. ‘Surd’ is a Latin word meaning ‘deaf, mute’ because a surd is a value that cannot be expressed as a number.

e.g. $\sqrt{9}$ is not a surd because $\sqrt{9}=3$ exactly.\\
\indent e.g. $\sqrt{2}$ is a surd because $\sqrt{2}\approx 1.41421\ldots$ and can only be written exactly as $\sqrt{2}$.

\subsection{Surd Laws}
Surds follow laws similar to the Power Laws:
\begin{Large}
\begin{align*}
x^{\frac{m}{n}}&=\sqrt[n]{x^m}=(\sqrt[n]{x})^m\\
x^{\frac{1}{n}}&=\sqrt[n]{a}\\
x^{\frac{m}{n}}&=(a^{\frac{1}{n}})^m\\
\sqrt{ mn}&=\sqrt{m} \cdot \sqrt{n}\\
\sqrt{\frac{m}{n}}&=\frac{\sqrt{m}}{\sqrt{n}}\\
(\sqrt{n})^2&=n
\end{align*}
\end{Large}

\subsection{Simplification of Surds}
Surds often appear in trigonometry and algebra and calculus, and to simplify a surd to its simplest possible form when it appears makes the rest of the problem easier.

To simplify a surd expression, list the factors of the radicand, choose two of them, one of which must be a square number, and then use the rule $\sqrt{mn}=\sqrt{m}\sqrt{n}$.

(Remember that ‘radical’ means ‘root,’ and is the name of the $\surd$ symbol and ‘radicand’ is the word for the value in the radical symbol.)
\begin{align*}
\text{e.g. }
\sqrt{18}
&=\sqrt{2 \times 9}\\
&=\sqrt{2} \times \sqrt{9}\\
&=\sqrt{2} \times 3\\
&=3\sqrt{2}\\
\\
\text{e.g. }
\sqrt{147}-2\sqrt{12}
&=\sqrt{49 \cdot 3}-2\sqrt{4 \cdot 3}\\
&=\sqrt{49} \cdot \sqrt{3}-2\sqrt{4} \cdot \sqrt{3}\\
&=7\sqrt{3}-2 \cdot 2\sqrt{3}\\
&=7\sqrt{3}-4\sqrt{3}\\
&=3\sqrt{3}
\end{align*}

\newpage

\subsection{Rationalizing the Denominator}
Rationalizing means turning something into a ratio, such as a fraction. The denominator means the number on the bottom of a fraction which says what sort of fraction it is. So rationalizing the denominator just means getting rid of the irrational surd on the bottom of a fraction.

That makes it easier to calculate a decimal number value for a surd, and it is easier generally to have the surds in the numerator rather than in the denominator of expressions.

e.g. Working out $\frac{3}{\sqrt{2}}$ requires long division of $\frac{3}{1.4142\ldots}$.

Instead, multiply the numerator and denominator by $\sqrt{2}$:

\begin{align*}
\frac{3}{\sqrt{2}}\cdot
\frac{\sqrt{2}}{\sqrt{2}}
&=\frac{3\sqrt{2}}{2}\\
&\approx\frac{3 \times 1.4142}{2}
\approx\frac{4.2426}{2}
\approx2.1213
\end{align*}

To rationalize $\frac{a}{\sqrt{b}}$ multiply the numerator and denominator by $\sqrt{b}$.

\begin{align*}
\text{e.g. }
\frac{1}{\sqrt{3}}
&=\frac{1}{\sqrt{3}}\cdot \frac{\sqrt{3}}{\sqrt{3}}\\
&=\frac{1\sqrt{3}}{\sqrt{3}\sqrt{3}}=\frac{\sqrt{3}}{3}
\end{align*}

\begin{align*}
\text{e.g. }
\frac{1}{\sqrt{3}}
&=\frac{1}{\sqrt{3}}\cdot\frac{\sqrt{3}}{\sqrt{3}}\\
&=\frac{1\sqrt{3}}{\sqrt{3}\sqrt{3}}=\frac{\sqrt{3}}{3}
\end{align*}

\begin{align*}
\text{e.g. }
\frac{\sqrt{3}}{\sqrt{2}}+\frac{2}{\sqrt{6}}
&=\frac{\sqrt{3}}{\sqrt{2}}\cdot
\frac{\sqrt{3}}{\sqrt{3}}
+\frac{2}{\sqrt{6}}\\
&=\frac{3}{\sqrt{6}}+\frac{2}{\sqrt{6}}
=\frac{5}{\sqrt{6}}\\
&=\frac{5}{\sqrt{6}}\cdot\frac{\sqrt{6}}{\sqrt{6}}\\
&=\frac{5\sqrt{6}}{6}
\end{align*}

\newpage

\subsection{Using the Difference of Squares Formula}

It is possible to rationalizing the denominator of a quadratic surd using the difference of squares formula.

Quadratic means ‘to do with a square’ and is from the Latin word for four. A quadratic surd is a surd or an expression containing a surd that has an index of 2, or is a ‘square root.’

e.g. $\sqrt{2}$, $\sqrt{5}$, and $3\sqrt{10}$ are quadratic surds since the indices of their roots are 2.

The difference of squares formula is that the product of the sum and difference of two values is equal to the difference of their squares: $(a+b)(a-b)=a^2-b^2$.

A conjugate means a pair of joined or related opposites, such as the sum and difference $(a+b)(a-b)$. For surds, the sum and difference of $n\sqrt{a}$ and $n\sqrt{b}$ are $n\sqrt{a}+m\sqrt{b}$ and $n\sqrt{a}-m\sqrt{b}$ which are said to be conjugate to each other. Conjugates such as these are useful because if you multiply a quadratic surd with its conjugate you get a rational number.

e.g. $(3+\sqrt{2})(3-\sqrt{2})=3^2-{\sqrt{2}}^2=9-2=7$

e.g. $(\sqrt{2}+\sqrt{3})(\sqrt{2}-\sqrt{3})={\sqrt{2}}^2-{\sqrt{3} }^2=2-3=-1$

Multiplying the numerator and denominator of a quadratic surd by its conjugate allows use of the difference of squares formula to rationalize the denominator:

To rationalize $\frac{a}{b+\sqrt{c}}$ multiply the numerator and denominator by $b-\sqrt{c}$.
\begin{align*}
\text{e.g. }
\frac{1}{\sqrt{7}+\sqrt{5}}
&=\frac{1}{\sqrt{7}+\sqrt{5}}
\cdot\frac{\sqrt{7}-\sqrt{5}}{\sqrt{7}-\sqrt{5}}\\
&=\frac{\sqrt{7}-\sqrt{5}}{(\sqrt{7})^2-(\sqrt{5})^2}\\
&=\frac{\sqrt{7}-\sqrt{5}}{7-5}\\
&=\frac{\sqrt{7}-\sqrt{5}}{-2}
\end{align*}

\begin{align*}
\text{e.g. }
\frac{3}{2+\sqrt{5}}
&=\frac{3}{2+\sqrt5}\cdot\frac{2-\sqrt{5}}{2-\sqrt{5}}\\
&=\frac{3(2-\sqrt{5})}{(2+\sqrt{5})(2-\sqrt{5})}\\
&=\frac{6-3\sqrt{5}}{4+2\sqrt{5}-2\sqrt{5}-5}\\
&=\frac{6-3\sqrt{5}}{-1}\\
&=3\sqrt{5}-6
\end{align*}


\begin{align*}
\text{e.g. }
\frac{\sqrt{3}+\sqrt{2}}
{3\sqrt{2}+2\sqrt{3}}
&=\frac{\sqrt{3}+\sqrt{2}}
{3\sqrt{2}+2\sqrt{3}}\cdot
\frac{3\sqrt{2}-2\sqrt{3}}
{3\sqrt{2}-2\sqrt{3}}\\
&=\frac{3\sqrt{2}\sqrt{3}
-2{\sqrt{3}}^2
+3{\sqrt{2}}^2
-2{\sqrt{3}}^2}
{(3\sqrt{2})^2-(2\sqrt{3})^2}\\
&=\frac{2\sqrt{3}-2{\sqrt{3}}^2+3{\sqrt{2}}^2}
{3^2{\sqrt{2}}^2-2^2{\sqrt{3}}^2}\\
&=\frac{\sqrt{6}-(2\cdot3)+(3\cdot2)}{(9\cdot2)-(4\cdot3)}\\
&=\frac{\sqrt{6}}{6}
\end{align*}

\section{Logarithms}
Logarithms were developed in the 1600s by mathematicians John Napier and John Briggs. The word was coined by Napier from the Latin words logos and arithmos and means ‘ratio-number.’

The product of two numbers can be found by looking up the logarithm for each number in a table of logarithms, adding the logarithms together, and then looking in the table for the number with that logarithm, known as that number’s antilogarithm.

For example, 123 and 234 can be multiplied by looking up their logarithms in a table ($2.09$ and $2.37$), adding them together ($2.09+2.37=4.46$), and looking through the values of the table to find the number that has that logarithm ($\log {28,782}=4.46$, so $123 \times 234=28,782$).

This was much faster and simpler than long tedious and error-prone calculations done by earlier methods, particularly for much longer numbers. In the same way, division problems became simpler subtraction problems, and the calculation of powers and roots were also simplified. This was an important problem to solve, and particularly helped in the field of navigation as sailors began to venture further around the world.

Logarithms were commonly used for calculations up until the invention of electronic calculators. They involved either the use of a book of values or the use of a slide rule that was based on a logarithmic scale rather than a linear scale. They are still used in logarithmic scales such as in measurement of pH, sound, and earthquake intensity, and they are used in mathematical modeling and in other fields.

\subsection{Definition of a Logarithm}
A logarithm is the opposite of an exponent. It is the power to which a base number must be raised to equal a given number. The base is written as a subscript next to the word ‘log.’

\begin{Large}
\begin{align*}
number &= base^{exponent}\\
log_{base} number &= exponent
\end{align*}
\end{Large}

For a number $n$ written as a power with a base $b$ and an exponent $e$ such that $n = be$, then $log_b n = e$. (Given $b>0$, $b\neq1$, and $n>0$.)

For example, for a base of $10$, the log of $100$ is $2$, because $100 = 10^2$.

Napier used a base of $e$ for his logarithms. This is ‘Euler’s number,’ roughly equal to 2.7183. It is a constant that often turns up in studies of the natural world and it is used in many branches of mathematics such as in problems involving growth and decay. It was first discovered in solving a problem to do with compound interest. Euler was a famous mathematician and $e$ was named after him. Logarithms with a base of $e$ are called natural logarithms and are written as ln $n$ or log\textsubscript{e} $n$.

Briggs later introduced the base of 10 as being easier to work with decimal numbers. Logarithms with a base of 10 are called common logarithms and are written simply as log n without indicating the base. You can assume that the base of a logarithm is 10 unless it is indicated otherwise.

\begin{center}
e.g.
\begin{tabular}{ l l }
$\log{25}=\log_{10}25\approx1.3979$ & $10^{1.3979}\approx25$\\
$\ln 25=\log_{e}25\approx3.2189$ & $e^{3.2189}\approx25$\\
\end{tabular}
\end{center}

\begin{center}
e.g.
\begin{tabular}{ l l }
$16 = 4^2$ & $\log_4{16}=2$\\
$4 = 8^{\frac{2}{3}}$ & $\log_8{4}=\frac{2}{3}$\\
$1000 = 10^3$ & $\log{1000}=3$
\end{tabular}
\end{center}

\subsection{Antilogarithms}
Given the logarithm of an unknown number, how do you find the number? 

Say $\log{x}\approx0.4771$.

The base is not written so it is implied to have a base of 10, so actually its $\log_10{x}\approx0.4771$.

The definition of a log is that given $n = b^e$ then $\log_b{n}=e$,
so $\log_10{x}\approx0.4771$ becomes $x\approx10^{0.4771}$.

It is the same for a base other than 10, such as $\log_5{x}\approx0.6065$, which becomes $x\approx5^{0.6065}$.

\subsection{Properties of Logarithms}

\begin{Large}
$$\log_a{1}=0$$
\end{Large}
\begin{center}
\text{(because }$n^0=0$
\end{center}
\begin{align*}
\text{also: }\ln{1}&=0\\
\text{and: }\log{1}&=0
\end{align*}

\begin{Large}
$$\log_a{a}=1$$
\end{Large}
\begin{center}
\text{(because }$n^1=n$\\
\end{center}

\begin{align*}
\text{also: }\ln{e}&=1\\
\text{and: }\log{10}&=1
\end{align*}

\begin{Large}
$$log_a{a^n}=n$$
\end{Large}
\begin{align*}
\text{also: }\ln{e^n}&=n\\
\text{and: }\log_10{n}&=n
\end{align*}

\begin{Large}
$$a^{log_b{n}}=n$$
\end{Large}
\begin{align*}
\text{also: }e^{\ln{x}}&=x\\
\text{and: }10^{\log{x}}&=x
\end{align*}

\subsection{Log Rules}

\subsubsection{Product Rule}
\begin{Large}
$$\log_b{A}+\log_b{B}=\log_b{AB}$$
\end{Large}

\begin{center}
\begin{tabular}{ l l l }
\text{proof:}
&$\text{let }log_b{A}=x$ & $A = b^x$\\
&\text{let }$\log_b{B}=y$ & $B = b^y$\\
&\text{so	}$AB=b^x b^y=b^{x+y}$ & $log_b{AB}=x+y=log_b{A}+log_b{B}$
\end{tabular}
\end{center}

\begin{align*}
\text{e.g. }
\log_2{5}+log_2{4}
&=\log_2{(5\cdot4)}=\log_2{20}\\
\cr
\text{e.g. given }\log{2}&=0.3010\text
{ and }\log{3}=0.4771\text
{ and }\log{5}=0.6990\\
\text{then }
\log{30}&=\log{(2\cdot3\cdot5)}
=\log{2}+\log{3}+\log{5}=1.4771
\end{align*}

\subsubsection{Quotient Rule}
\begin{Large}
$$\log_b{A}-\log_b{B}=\log_b\frac{A}{B}$$
\end{Large}

\begin{center}
\begin{tabular}{ l l l }
\text{proof:}&\text{let }$\log_b{A}=x$ & $A = b^x$\\
&\text{let }$\log_b{B}=y$ & $B = b^y$\\
&\text{so	}$\frac{A}{B}=\frac{b^x}{b^y}=b^{x-y}$ & $\log_b{\frac{A}{B}}=x-y=\log_b{A}-\log_b{B}$
\end{tabular}
\end{center}

\begin{align*}
\text{e.g. }\log_2{20}+log_2{4}&
=\log_2{(\frac{20}{4})}=\log_2{5}\\
\cr
\text{e.g. given }\log{2}&=0.3010\\
\text{then }
\log{50}&=\log{(\frac{100}{2})}\\
&=\log{100}-\log{2}
\text{  }(\log_10{100}=2)\text{ (because }100=10^2)\\
&=2-0.3010=1.699
\end{align*}

\subsubsection{Power Rule}
\begin{Large}
$$log_a{x^n}=n\log_a{x}$$
\end{Large}

\begin{center}
\begin{tabular}{ l l l l }
\text{proof:}&\text{let }$m=\log_a{x}=x$ & $x = a^m$\\
&\text{so	}$x^n=({a^m})^n=a^{mn}$ & $\log{x^n}=mn=nm=n\log_a{x}$
\end{tabular}
\end{center}

\begin{align*}
\text{e.g. }\log_3{\frac{1}{3}}&=log_3{1}-log_3{3}=0-1=-1\\
\text{(from properties of logs:   }&
\log_a{1}=0\text{ and }\log_a{a}=1\text{)}\\
\text{e.g. }\log{x^4}&=4\log{x}\\
\text{e.g. }\log{\sqrt{10}}&=\log10^{\frac{1}{2}}
=\frac{1}{2}\log{10}=\frac{1}{2}\cdot1=\frac{1}{2}\\
\end{align*}

\subsubsection{Root Rule}
\begin{Large}
$$\log_a{(\sqrt[n]{x})}=\frac{\log_a{x}}{n}$$
\end{Large}

\begin{align*}
\text{e.g.}
\log_2{\sqrt{8}}&=\frac{\log_2{8}}{2}=\frac{3}{2}\\
\text{e.g.}
\log_3{\sqrt{9}}&=\frac{\log_3{9}}{2}=\frac{2}{2}=1
\end{align*}

\newpage

\subsubsection{Change of Base Law}

It is unusual to have to find a log for a number with a base other than 10 or $e$,\\
but there is a formula that can be used:

\begin{Large}
$$\log_a{n}=\frac{\log_b{n}}{\log_b{a}}$$
\end{Large}

\begin{align*}
\text{for base 10}
\log_a{n}&=\frac{\log_{10}{n}}{\log_{10}{a}}=\frac{\log{n}}{\log{a}}\\
\text{for base e}
\log_a{n}&=\frac{\log_{e}{n}}{\log_{e}{a}}=\frac{\log{n}}{\log{a}}
\end{align*}

\begin{center}
\begin{tabular}{lll}
proof: & $\log_a{n}=m n=a^m$ & (by definition)\\
&$\log_b{n}=\log_b{a^m}$ & (taking log of both sides)\\
&$\log_b{n}=m\log_b{a}$ & (using power rule)\\
&$\frac{\log_b{n}}{\log_b{a}}=m={\log_a{n}}$ & (rearranging)
\end{tabular}
\end{center}

\begin{center}
\begin{align*}
\text{e.g. }\log_4{40}
&=\frac{\log{40}}{\log{4}}=\frac{16.0206}{0.60206}=2.661\\
\text{e.g. }\log_2{5}+log_3{7}
&=(\frac{\log{5}}{\log{2}})+(\frac{\log{7}}{\log{3}})=4.0932\\
\text{e.g. }
\log{x}+3\log{y}-4\log{z}
&=\log{x}+\log{y^3}-\log{z^4}=\log\frac{x{y^3}}{z^4}\\\\
\text{e.g. }
2+4\log_3{x}
&=\log_3{9}+\log_3{x^4}=log_3{9x^4}
\end{align*}
\text{(express 2 as a log with base 3: }
$2=2\log_3{3}=\log_3{3^2}=\log_3{9})$
\end{center}

\section{Logarithmic Equations}

\subsubsection{Using the Definition of a Logarithm}

\onehalfspacing
\begin{center}
\begin{tabular}{llll}
\text{e.g. }&$\log_10{x}=2$ & $\longrightarrow x=10^2$ & $\longrightarrow x=100$\\
\text{e.g. }&$\log_x{25}=2$ & $\longrightarrow25=x^2$ & $\longrightarrow x=\pm5=5$\\
\text{e.g. }&$\log_x{22}=2.1$ & $\longrightarrow22=x^{2.1}$ & $\longrightarrow x=\sqrt[2.1]{22}=4.358$\\
\text{e.g. }&$\log_4{5}=x$ & $\longrightarrow\frac{\log{5}}{\log{4}}=1.161$\\
\end{tabular}
\end{center}
\singlespacing

\subsubsection{Using Equivalence}
(equivalence: if $log_b{x}=log_b{y}$ then $x=y$.)
\begin{align*}
\text{e.g. }\log{x}+\log{5}&=\log{20}\\
\log{5x}&=\log{20}\text{ (product rule)}\\
5x&=20\\
x&=4
\end{align*}

\newpage

\subsubsection{Grouping Log Terms to one side}

(Rearranging to get log terms on on side and numbers on the other)
\begin{align*}
\text{e.g. }\log_4{(2x+4)}-2&=\log_4{3}\\
\log_4{(2x+4)}-\log_4{3}&=2\\
\log_4{\frac{2x+4}{3}}&=2\\
\frac{2x+4}{3}&=4^2\\
\frac{2x+4}{3}&=16\\
2x+4&=48\\
2x&=44\\
x&=22
\end{align*}

\subsubsection{Replacing a Number with the Equivalent Log}
\begin{align*}
\text{e.g. }\log_4{(2x+4)}-2&=\log_4{3}\\
\log_4{(2x+4)}-\log_4{16}&=\log_4{3}\\
\log_4{(2x+4)}-\log_4{4^2}&=\log_4{3}\\
\log_4\frac{2x+4}{16}&=log_4{3}\\
\frac{2x+4}{16}&=3\\
2x+4&=48\\
2x&=44\\
x&=22
\end{align*}

\section{Exponential Equations}
e.g. How long will it take investor who deposits \$20,000 compounding at 5 \% p.a. to increase this amount to \$30,000?

Compound Interest Formula:		$A=P(1+i)^n$
\begin{align*}
30,000&=20,000(1.05)^n\\
\frac{30,000}{20,000}&=1.05^n\\
1.5&=1.05^n\\
\log{1.5}&=\log{1.05^n}\\
\log{1.5}&=n\log{1.05}\\
\frac{\log{1.5}}{\log{1.05}}&=n\\
8.31&=n\\
\text{check: }1.05^{8.31}&=1.5
\end{align*}

\end{document}