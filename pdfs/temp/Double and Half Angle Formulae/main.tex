\documentclass{article}
\usepackage{amsmath, amssymb, tikz}
\begin{document}

\title{Double and Half Angle Formulae}
\author{Tutoring Centre Ferndale\\
\includegraphics[width=4em]{ApS_logo.png}}
\date{}
\begin{document}
\maketitle

There are special formulae for evaluating double and half of known angles.

\subsection*{Double Angle Formulae}

The double angle formulae express trigonometric functions of $2\theta$ in terms of trigonometric functions of $\theta$.

\begin{itemize}
    \item \textbf{Sine:}
    \begin{align*}
        \sin 2 \theta &= \sin(\theta+\theta)\\
        &=\sin \theta \cos \theta + \sin \theta \cos \theta\\
        &= 2\sin(\theta)\cos(\theta)
    \end{align*}

    \item \textbf{Cosine:}
    \begin{align*}
        \cos 2 \theta &= \cos(\theta+\theta)\\
        &=\cos \theta \cos \theta - \sin \theta \sin \theta\\
        &=\cos^2 \theta - \sin^2 \theta
    \end{align*}

    \begin{align*}
        \cos 2 \theta &= \cos^2 \theta - \sin^2 \theta\\
        &=1 - 2 \sin^2 \theta\\
        &=2\cos^2(\theta) - 1
    \end{align*}

    \item \textbf{Tangent:}
    \begin{align*}
        \tan(2\theta) &= \tan (\theta + \theta)\\
        &=\frac{\tan \theta + \tan \theta}{1 - \tan \theta \tan \theta}\\
        &=\frac{2\tan \theta}{1 - \tan^2 \theta}
    \end{align*}
\end{itemize}

\newpage

\subsection*{Half Angle Formulae}

The half angle formulae, derived by using the double angle formulae, express trigonometric functions of $\frac{\theta}{2}$ in terms of trigonometric functions of $\theta$.

\subsubsection*{Sine}

We begin with the cosine double-angle formula:

\begin{align*}
    \cos(2\alpha) &= 1 - 2\sin^2(\alpha)
\end{align*}

Let $\theta = 2\alpha$. Then $\alpha = \frac{\theta}{2}$. Substituting these into the double-angle formula, we get:

\begin{align*}
    \cos \theta = 1 - 2\sin^2\left(\frac{\theta}{2}\right)
\end{align*}

Now, solve for $\sin\left(\frac{\theta}{2}\right)$:

\begin{align*}
    2\sin^2\left(\frac{\theta}{2}\right) &= 1 - \cos \theta \\
    \sin^2\left(\frac{\theta}{2}\right) &= \frac{1 - \cos \theta}{2} \\
    \sin\left(\frac{\theta}{2}\right) &= \pm\sqrt{\frac{1 - \cos \theta}{2}}
\end{align*}

The $\pm$ sign indicates that the sign of $\sin\left(\frac{\theta}{2}\right)$ depends on the quadrant of $\frac{\theta}{2}$.

\subsubsection*{Cosine}

We use another form of the cosine double-angle formula:

\begin{align*}
    \cos(2\alpha) = 2\cos^2(\alpha) - 1
\end{align*}

Again, let $\theta = 2\alpha$, so $\alpha = \frac{\theta}{2}$. Substituting:

\begin{align*}
    \cos \theta = 2\cos^2\left(\frac{\theta}{2}\right) - 1
\end{align*}

Now, solve for $\cos\left(\frac{\theta}{2}\right)$:

\begin{align*}
    2\cos^2\left(\frac{\theta}{2}\right) &= 1 + \cos \theta \\
    \cos^2\left(\frac{\theta}{2}\right) &= \frac{1 + \cos \theta}{2} \\
    \cos\left(\frac{\theta}{2}\right) &= \pm\sqrt{\frac{1 + \cos \theta}{2}}
\end{align*}

The $\pm$ sign, as before, depends on the quadrant of $\frac{\theta}{2}$.

\subsubsection*{Tangent}

We can derive the tangent half-angle formula using the sine and cosine half-angle formulas:

\begin{align*}
    \tan\left(\frac{\theta}{2}\right) = \frac{\sin\left(\frac{\theta}{2}\right)}{\cos\left(\frac{\theta}{2}\right)}
\end{align*}

Substituting the derived formulas for sine and cosine:

\begin{align*}
    \tan\left(\frac{\theta}{2}\right) = \pm\sqrt{\frac{1 - \cos \theta}{1 + \cos \theta}}
\end{align*}

This is a valid form. We can derive more useful forms by multiplying the numerator and denominator by a strategic term:

\subsubsection*{Form 1}

\begin{align*}
    \tan\left(\frac{\theta}{2}\right) &= \pm\sqrt{\frac{1 - \cos \theta}{1 + \cos \theta}} \cdot \sqrt{\frac{1 - \cos \theta}{1 - \cos \theta}} \\
    &= \pm\frac{1 - \cos(\theta)}{\sqrt{1 - \cos^2 \theta}} \\
    &= \pm\frac{1 - \cos(\theta)}{|\sin \theta|}
\end{align*}
If we restrict $\theta$ to an interval where $\sin(\theta)$ is positive:
\begin{align*}
    \tan\left(\frac{\theta}{2}\right) = \frac{1 - \cos \theta}{\sin \theta}
\end{align*}

\subsubsection*{Form 2}

\begin{align*}
    \tan\left(\frac{\theta}{2}\right) &= \pm\sqrt{\frac{1 - \cos \theta}{1 + \cos \theta}} \cdot \sqrt{\frac{1 + \cos\theta}{1 + \cos \theta}} \\
    &= \pm\frac{\sqrt{1 - \cos^2 \theta}}{1 + \cos \theta} \\
    &= \pm\frac{|\sin \theta|}{1 + \cos \theta}
\end{align*}
Similarly, if we restrict $\theta$ to an interval where $\sin \theta$ is positive:
\begin{align*}
    \tan\left(\frac{\theta}{2}\right) = \frac{\sin(\theta)}{1 + \cos \theta}
\end{align*}

These alternative forms are often preferred because they avoid the $\pm$ sign ambiguity (when considering specific intervals for $\theta$) and are sometimes easier to work with algebraically.

\section*{Summary}

\subsection*{Double Angle Formulas}

\begin{itemize}
    \item \textbf{Sine:} $\sin(2\theta) = 2\sin \theta \cos \theta$
    \item \textbf{Cosine:}
    \begin{itemize}
        \item $\cos2\theta = \cos^2 \theta - \sin^2 \theta$
        \item $\cos2\theta = 2\cos^2 \theta - 1$
        \item $\cos2\theta = 1 - 2\sin^2 \theta$
    \end{itemize}
    \item \textbf{Tangent:} $\tan 2\theta = \frac{2\tan \theta}{1 - \tan^2 \theta}$
\end{itemize}

\subsection*{Half Angle Formulas}

\begin{itemize}
    \item \textbf{Sine:} $\sin\left(\frac{\theta}{2}\right) = \pm\sqrt{\frac{1 - \cos \theta}{2}}$
    \item \textbf{Cosine:} $\cos\left(\frac{\theta}{2}\right) = \pm\sqrt{\frac{1 + \cos \theta}{2}}$
    \item \textbf{Tangent:}
    \begin{itemize}
        \item $\tan\left(\frac{\theta}{2}\right) = \frac{1 - \cos \theta}{\sin \theta}$
        \item $\tan\left(\frac{\theta}{2}\right) = \frac{\sin \theta}{1 + \cos \theta}$
    \end{itemize}
\end{itemize}

\noindent \textbf{Note:} The $\pm$ sign in the half-angle formulas depends on the quadrant of $\frac{\theta}{2}$.

\end{document}
