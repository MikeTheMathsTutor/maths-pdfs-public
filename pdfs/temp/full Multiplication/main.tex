\documentclass{article}

\usepackage[a4paper,margin=3cm]{geometry}
\usepackage{amsmath}
\usepackage{mathtools}
\usepackage{tikz}
\usepackage{pgf-pie}
\usepackage{tikz-cd}
\usetikzlibrary{shapes.geometry}
\usepackage{cancel}
\usepackage{setspace}
\usepackage[fontsize=16pt]{fontsize}
\usepackage{indentfirst}

\author{}
\date{}
\title{Full\\Multiplication\\
\vspace{28pt}
\begin{normalsize}Applied Scholastics, Ferndale \end{normalsize}}

\begin{document}

\maketitle
\pagebreak
\tableofcontents
\pagebreak
\begin{spacing}{1.25}

\section{Multiplication}
Multiply means to add a number to itself many times. It is from Latin. Multi- means many, and -ply means fold, so multiply means folded many times.\\

Multiplication is called times, because it is adding a number to itself many times.\\

The times symbol is $\times$: 
$\overbrace{8+8+8+8}^{\textrm{8 added to itself 4 times}}= 8 \times 4 = 32$\\

Sometimes a dot ( . ) or a raised dot ($\text{ }\cdot$\text{ })is used for times because it can be confused for the letter x, and because it is shorter.
Sometimes the times symbol is just left out so if two things are next to each other in a maths statement it is assumed that times is meant.\\

The number being multiplied is called the multiplicand, which is Latin for "to be multiplied."\\

The number it is being multiplied by is called the multiplier.\\

The result of multiplying is called the product.
$$\text{Multiplicand}\times \text{Multiplier} = \text{Product.}$$

\newpage

\section{Skip Counting}
Learning skip counting is the first step in learning to multiply. These are called the multiples of a number.\\

Practice until you can skip count all the way to 100 for any number without looking.\\

Practice counting by twos all the way to 100:\\
2, 4, 6, 8, 10, 12, 14, 16, 18, 20, 22, 24, 26, 28, 30, 32, 34, 36, 38, 40, 42, 44, 46, 48, 50, 52, 54, 56, 58, 60, 62, 64, 66, 68, 70, 72, 74, 76, 78, 80, 82, 84, 86, 88, 90, 92, 94, 96, 98, 100.

You can see this on a number line.

\begin{center}
\begin{tikzpicture}
\draw[thick, ->] (0,0) -- (9,0) node[below] {$\ $};
\foreach \n in {0,1,2,3,4,5,6,7,8} {\draw (\n,0.1) -- (\n,-0.1) node[below] {$\n$};}
\draw[->, bend left=30] (0,0.3) to node[above] {$2$} (2,0.3);
\draw[->, bend left=30] (2,0.3) to node[above] {$2$} (4,0.3);
\draw[->, bend left=30] (4,0.3) to node[above] {$2$} (6,0.3);
\draw[->, bend left=30] (6,0.3) to node[above] {$2$} (8,0.3);
\end{tikzpicture}
\end{center}

Practice counting by threes all the way to 100:\\
3, 6, 9, 12, 15, 18, 21, 24, 27, 30, 33, 36, 39, 42, 45, 48, 51, 54, 57, 60, 63, 66, 69, 72, 75, 78, 81, 84, 87, 90, 93, 96, 99.

\begin{center}
\begin{tikzpicture}
\draw[thick, ->] (0,0) -- (13,0) node[below] {$\ $};
\foreach \n in {0,1,2,3,4,5,6,7,8,9,10,11,12} {\draw (\n,0.1) -- (\n,-0.1) node[below] {$\n$};}
\draw[->, bend left=30] (0,0.3) to node[above] {$3$} (3,0.3);
\draw[->, bend left=30] (3,0.3) to node[above] {$3$} (6,0.3);
\draw[->, bend left=30] (6,0.3) to node[above] {$3$} (9,0.3);
\draw[->, bend left=30] (9,0.3) to node[above] {$3$} (12,0.3);
\end{tikzpicture}
\end{center}

\newpage

Practice counting by fours all the way to 100:\\
4, 8, 12, 16, 20, 24, 28, 32, 36, 40, 44, 48, 52, 56, 60, 64, 68, 72, 76, 80, 84, 88, 92, 96, 100.\\

\begin{center}
\begin{tikzpicture}
\draw[thick, ->] (0,0) -- (13,0) node[below] {$\ $};
\foreach \n in {0,1,2,3,4,5,6,7,8,9,10,11,12,13} {\draw (\n,0.1) -- (\n,-0.1) node[below] {$\n$};}
\draw[->, bend left=30] (0,0.3) to node[above] {$4$} (4,0.3);
\draw[->, bend left=30] (4,0.3) to node[above] {$4$} (8,0.3);
\draw[->, bend left=30] (8,0.3) to node[above] {$4$} (12,0.3);
\end{tikzpicture}
\end{center}

Practice counting by fives all the way to 100:\\
5, 10, 15, 20, 25, 30, 35, 40, 45, 50, 55, 60, 65, 70, 75, 80, 85, 90, 95, 100.\\

Practice counting by sixes all the way to 100:\\
6, 12, 18, 24, 30, 36, 42, 48, 54, 60, 66, 72, 78, 84, 90, 96.\\

Practice counting by sevens all the way to 100:\\
7, 14, 21, 28, 35, 42, 49, 56, 63, 70, 77, 84, 91, 98.\\

Practice counting by eights all the way to 100:\\
8, 16, 24, 32, 40, 48, 56, 64, 72, 80, 88, 96.\\

Practice counting by nines all the way to 100:\\
9, 18, 27, 36, 45, 54, 63, 72, 81, 90, 99.\\

Practice counting by tens all the way to 100:\\
10, 20, 30, 40, 50, 60, 70, 80, 90, 100.\\

\newpage

\section*{Times tables}
Now learn the times table from 1 to 10.

\begin{center}
\begin{tabular}{|c||*{10}{c|}}
\hline
$\times$ & 1 & 2 & 3 & 4 & 5 & 6 & 7 & 8 & 9 & 10 \\
\hline\hline
1 & 1 & 2 & 3 & 4 & 5 & 6 & 7 & 8 & 9 & 10 \\
2 & 2 & 4 & 6 & 8 & 10 & 12 & 14 & 16 & 18 & 20 \\
3 & 3 & 6 & 9 & 12 & 15 & 18 & 21 & 24 & 27 & 30 \\
4 & 4 & 8 & 12 & 16 & 20 & 24 & 28 & 32 & 36 & 40 \\
5 & 5 & 10 & 15 & 20 & 25 & 30 & 35 & 40 & 45 & 50 \\
6 & 6 & 12 & 18 & 24 & 30 & 36 & 42 & 48 & 54 & 60 \\
7 & 7 & 14 & 21 & 28 & 35 & 42 & 49 & 56 & 63 & 70 \\
8 & 8 & 16 & 24 & 32 & 40 & 48 & 56 & 64 & 72 & 80 \\
9 & 9 & 18 & 27 & 36 & 45 & 54 & 63 & 72 & 81 & 90 \\
10 & 10 & 20 & 30 & 40 & 50 & 60 & 70 & 80 & 90 & 100 \\
\hline
\end{tabular}
\end{center}

\vspace{32pt}
Do it by saying the times table out loud, one column at a time, many times, until they can all be done without looking. Again, this works well when done as a group.\\

For each column, say "1 times 4 is 4. 2 times 4 is 8. 3 times 4 is 12. 4 times 4 is 16... and so on.\\

Then ask for the answer to random pairs of numbers, and do it until there is always an instant answer.\\

Next, learn up to the 12 times table.

\newpage

\subsection*{12 times table}

\begin{center}
\begin{tabular}{|c||c|c|c|c|c|c|c|c|c|c|c|c|}
\hline
$\times$ & 1 & 2 & 3 & 4 & 5 & 6 & 7 & 8 & 9 & 10 & 11 & 12 \\
\hline\hline
1 & 1 & 2 & 3 & 4 & 5 & 6 & 7 & 8 & 9 & 10 & 11 & 12 \\
2 & 2 & 4 & 6 & 8 & 10 & 12 & 14 & 16 & 18 & 20 & 22 & 24 \\
3 & 3 & 6 & 9 & 12 & 15 & 18 & 21 & 24 & 27 & 30 & 33 & 36 \\
4 & 4 & 8 & 12 & 16 & 20 & 24 & 28 & 32 & 36 & 40 & 44 & 48 \\
5 & 5 & 10 & 15 & 20 & 25 & 30 & 35 & 40 & 45 & 50 & 55 & 60 \\
6 & 6 & 12 & 18 & 24 & 30 & 36 & 42 & 48 & 54 & 60 & 66 & 72 \\
7 & 7 & 14 & 21 & 28 & 35 & 42 & 49 & 56 & 63 & 70 & 77 & 84 \\
8 & 8 & 16 & 24 & 32 & 40 & 48 & 56 & 64 & 72 & 80 & 88 & 96 \\
9 & 9 & 18 & 27 & 36 & 45 & 54 & 63 & 72 & 81 & 90 & 99 & 108 \\
10 & 10 & 20 & 30 & 40 & 50 & 60 & 70 & 80 & 90 & 100 & 110 & 120 \\
11 & 11 & 22 & 33 & 44 & 55 & 66 & 77 & 88 & 99 & 110 & 121 & 132 \\
12 & 12 & 24 & 36 & 48 & 60 & 72 & 84 & 96 & 108 & 120 & 132 & 144 \\
\hline
\end{tabular}
\end{center}

\\
Practice counting by elevens all the way to 100:\\
11, 22, 33, 44, 55, 66, 77, 88, 99.\\

Practice counting by twelves all the way to 100:\\
12, 24, 36, 48, 60, 72, 84, 96.\\

Then say the times table out loud, as a group if you can, one column at a time, many times, until they all can be done without looking.\\

Then ask for the answer to random pairs of numbers, and do it until there is always an instant answer.\\

\newpage

\section{Multiplication Tips}

\subsection*{Multiplication by 5}
Multiplication by 5 can be more easily done by shifting the decimal point one position to the right, and halving.\\

$54 \times 5 = 5.4 \times \frac{1}{2} = 2.7$

\subsection*{Multiplying two numbers,\\each from 6 to 10,\\on your fingers}

\begin{itemize}
\item Hold your hands palms down. Number the fingers and thumb of the left hand from 10 to 6, and the thumb and fingers of the right hand from 6 to 10.
\item Bend the fingers or thumbs corresponding to each number, and the fingers or thumbs between those two fingers.
\item The number of bent fingers or thumbs gives the tens digit.
\item Add the product of the unbent fingers or thumbs on each hand.
\end{itemize}

\subsection*{Multiplying by 9}
Multiplying by 9 is the same as multiplying by 10 and then subtracting the multiplier.\\

$23 \times 9 = (23 \times 10) - 23 = 230 - 23 = 207$

\subsection*{Multiplying by 9 on your fingers}

\begin{itemize}
\item Number the fingers and thumbs from 1 to 10 from left to right.
\item Bend the finger or thumb corresponding to the number.
\item The number of fingers or thumb to the left of the bend gives the tens digit.
\item The number of fingers or thumb to the right of the bend gives the units digit.
\end{itemize}

\subsection*{Multiplying by Powers of 10}

Numbers can be multiplied by 10 or by any power of 10 by just adding the zeroes to the product.\\

$12 \times 10 = 120$\hspace{2em}$32 \times 1000 = 32,000$.\\

What you are really doing there is shifting the decimal point to the right, so this works with decimal fractions as well.
$$3824 \times 10 = 38240$$
$$38.24 \times 10 = 382.4$$

\subsection*{Multiplying Multiples of 10}

Multiples of 10 can be multiplied by adding all the zeroes to the product, as in $1,200 \times 300 = 360,000$.

\subsection*{Multiplying by 11}
Multiplying by 11 is the same as multiplying by 10 and then adding the multiplier.\\

$23 \times 11 = (23 \times 10) + 23 = 230 + 23 = 253$

\subsection*{Multiplication by 12}
Multiplication by 12 is multiplication by 10, plus doubling.\\
$22 \times 12 = 22 \times 10 + 24 = 244$\\

\section{Multiplication Strategies}

\subsection*{Doubling and Halving}

Some multiplications can be made easier by doubling one factor and halving the other, or even by using triples and thirds. It is useful to be able to double and halve numbers.\\

$38 \times 25 = (38 \div 2) \times (25 \times 2) = 19 \times 50$.\\

$19 \times 50$ is easier to work out than $38 \times 25$. In fact, double and halve again and you have your answer:\\

$19 \times 50 = (19 \div 2) \times (50 \times 2) = 9.5 \times 100 = 950$.\\

This works because the doubling of one factor is reversed by the halving of the other factor so that the product remains the same. You can see what happens in terms of objects arranged into rectangles.
\begin{center}
$5 \times 4$\\
\begin{tabular}{ c c c c c }
\diamond & \diamond & \diamond & \diamond & \diamond \\
\diamond & \diamond & \diamond & \diamond & \diamond \\
\diamond & \diamond & \diamond & \diamond & \diamond \\
\diamond & \diamond & \diamond & \diamond & \diamond \\
\end{tabular}
\end{center}

\begin{center}
is the same amount as $10 \times 2$.\\
\begin{tabular}{ c c c c c c c c c c }
\diamond & \diamond & \diamond & \diamond &\diamond & \diamond & \diamond & \diamond &\diamond & \diamond \\
\diamond & \diamond & \diamond & \diamond &\diamond & \diamond & \diamond & \diamond &\diamond & \diamond \\
\end{tabular}
\end{center}

\subsection*{Adjusting to a Multiple of 10}

Multiplication can be made simpler by adjusting one of the terms to a multiple of 10.\\

$89 \times 7 = (90 - 1) \times 7 = 630 - 7 = 623$\\

is easier to work out than\\

$89 \times 7 = (80 + 9) \times 7 = 560 + 63 = 623$.\\

\subsection*{Multiplication by Factors}

Long multiplications can be broken down into smaller easier multiplications by breaking the terms into smaller factors.
\begin{align*}
34 \times 12 &= (34 \times 10) + (34 \times 2)\\
             &= 340 + 68 = 408\\
\text{or }   &= 34 \times 4 \times 3\\
             &= 136 \times 3 = 408\\
\text{or }   &= 34 \times 2 \times 2 \times 3\\
             &= 68 \times 2 \times 3\\
             &= 136 \times 3 = 408
\end{align*}

\subsection*{Multiplication by Subtraction}

Multiplication can be made simpler writing one factor as a difference, with one term being a power of ten, and using the distributive property of multiplication ($a \times (b + c) = a \times b + a \times c$) to get a product.
\begin{align*}
23 \times 98 &= 23 \times (100 - 2)\\
             &= 2,300 - 46= 2,254
\end{align*}
This way is much easier than doing the same multiplication in columns.\\

\subsection*{Multiplying Numbers\\that Differ by 2}

$$n \times (n+2) = (n+1)^2-1$$

$$7 \times 9 = 8^2-1=63$$
$$19 \times21= 20^2-1=399$
$$99\times101=100^2-1=9,999$$

\pagebreak

\subsection*{Stick Multiplication}

Draw sets of parallel lines representing the first factor, crossed by sets of parallel lines representing the second factor, and count the intersections.\\

\begin{tikzpicture}
\path (3.0,-1.7) node[circle,draw] (43) {43};
\draw[fill] (4.0, 0.0) -- (11, -3.0);
\draw[fill] (3.8,-0.5) -- (10.8,-3.5);
\draw[fill] (3.6,-1.0) -- (10.6,-4.0);
\path (3.1,-0.4) node (43ones) {ones};
\draw[fill] (3.0,-2.5) -- (10.0,-5.5);
\draw[fill] (2.8,-3.0) -- (9.8,-6.0);
\draw[fill] (2.6,-3.5) -- (9.6,-6.5);
\draw[fill] (2.4,-4.0) -- (9.4,-7.0);
\path (2.0,-3.2) node (43tens) {tens};
\path (9.3,-1.0) node[circle,draw] (22) {22};
\draw[fill] (2.0,-6.0) -- (8.0, 0.0);
\draw[fill] (2.5,-6.4) -- (8.5,-0.4);
\path (8.8,-0.0) node (22tens) {tens};
\draw[fill] (5.0,-7.0) -- (10.0,-1.6);
\draw[fill] (5.5,-7.4) -- (10.5,-2.0);
\path (10.7,-1.7) node (22ones) {ones};
\path (6.9,-3.3) node[scale=0.8] (14tens) {14};
\draw[rotate=15] (5.6,-5.1) ellipse (125pt and 26pt);
\path (7.0,-6.4) node[scale=0.8] (8hundreds) {8};
\draw[rotate=35] (3.1,-8.4) ellipse (42pt and 18pt);
\path (6.7,-0.7) node[scale=0.8] (6ones) {6};
\draw[rotate=26] (5.2,-4.5) ellipse (36pt and 16pt);
\end{tikzpicture}

Here we see there are 8 hundreds, a total of 14 tens, and 6 ones, so:

$$43 \times 22 = $$

\begin{center}
\begin{tabular}{c@{\,}c@{\,}c}
	8&0&0\\
    1&4&0\\
   + & &6\\
	\hline
	9&4&6\\
	\hline
	\hline
\end{tabular}
\end{center}

\pagebreak

\subsection*{Cross-Multiplying}

The FOIL method is how to multiply two pairs of values. FOIL stands for First Outer Inner Last. It is the order in which you multiply out the various terms to get a product.\\

It can also be called cross-multiplying because of the pattern that it makes. It is possibly the origin of the $\times$ symbol used for multiplication.\\

You can do
\begin{align*}
72 \times 96 &= (70 + 2) \times (90 + 6)\\
             &= (70 \times 9) + (70 \times 6) + (2 \times 9) + (2 \times 6)\\
             &=1080
\end{align*}

but if you lay it out as a cross you can easily see the terms to multiply and make a list of the products to add.\\

\begin{figure}[ht]
\begin{minipage}[ht]{0.5\linewidth} \centering 
\begin{tikzpicture} 
\path ( 0, 0)   node (times)   {$\times$};
\path (-2, 1)   node (seventy) {70};
\path ( 0, 1)   node (plus)    {+};
\path ( 2, 1)   node (two)     {2};
\path (-2,-1) node (nine)    {9};
\path ( 0,-1) node (pluss)   {+};
\path ( 2,-1) node (six)     {6};
\draw [thick] (seventy) -- (times);
\draw [thick,->] (times) -- (six);
\draw [thick] (two) -- (times);
\draw [thick,->] (times) -- (nine);
\draw [thick,->] (seventy) -- (nine);
\draw [thick,->] (two) -- (six);
\end{tikzpicture}
\end{minipage}
\begin{minipage}[ht]{0.5\linewidth} \centering 
\begin{center}
\begin{tabular}{c@{\,}c@{\,}c@{\,}c@{\,}}
 &6&3&0\\
 &4&2&0\\
 & &1&8\\
+& &1&2\\
\hline
= 1&0&8&0\\
\end{tabular}
\end{center}
\end{minipage}
\end{figure}

\subsection*{Multiplication by Adding Parts}

The distributive property of multiplication 4($a \times (b + c) = a \times b + a \times c$) allows you to split numbers up and  multiply by adding the parts. This is the same as multiplying in columns but it can be used for single-line or mental multiplication as well.\\

$36 \times 3 \frac{1}{2} = 36 \times 3 + 36 \times \frac{1}{2} = 108 + 18 = 126$.\\

$83 \times 9 = (80 + 3) \times 9 = 720 + 27 = 747$

\subsection*{Box Multiplication}
Multiplication of large numbers can be done by breaking the factors down and displaying them in a grid that makes it easy to see the partial products that are needed.\\

\noindent
$123\times45 =$\\\\
\noindent
\begin{tikzpicture}
\node at (3.5,0.5) {100};
\node at (6.5,0.5) {+ 20};
\node at (9.5,0.5) {+ 3}; % multiplicand
\node at (1,-0.7) {\ 40};
\node at (1,-2.2) {+ 5}; % multiplier
\draw (2,0) -| (11,-3);
\draw (2,0) |- (11,-3);
\draw (2,-1.5) -- (11,-1.5);
\draw (5,0) -- (5,-3);
\draw (8,0) -- (8,-3); % box & gridlines
\node at (3.5,-0.7) {4000};
\node at (3.5,-2.2) {500};
\node at (6.5,-0.7) {800};
\node at (9.5,-0.7) {120};
\node at (3.5,-2.2) {500};
\node at (6.5,-2.2) {100};
\node at (9.5,-2.2) {15};
\end{tikzpicture}
\\

\begin{center}
\begin{tabular}{c@{\,}c@{\,}c@{\,}c@{\,}c}
^{1}&4&0&0&0\\
	& &8&0&0\\
  + & &1&2&0\\
    & &5&0&0\\
    & &1&0&0\\
  + & & &1&5\\
  	\hline
	&5,&5&3&5\\
	\hline
	\hline
\end{tabular}
\end{center}

\pagebreak

\section{Multiplication in Columns}

\subsection*{Multiplying a multi-digit number\\by a single-digit number}
To multiply numbers of more than one digit, it's best to do it by arranging them into columns aligned on the units. Then you can multiply each column separately into sub-products, written with the units of each sub-product aligned with the column that produced it. Then you add the sub-products to get a final product.\\

Start by learning to multiply a multi-digit number by a single-digit number.

\begin{center}
\begin{tabular}{c@{\,}c@{\,}c@{\,}c@{\,}c}
      &1,&2&3&4\\
\times & & & &6\\
\hline
       & & &2&4\\
   +& &^{1}1&8&\\
\hline
        &1&2& &\\
      + &6& & &\\
\hline
      &7,&4&0&4\\
\hline
\hline
\end{tabular}\\
\end{center}

\newpage

Filling in the blank spaces in the sub-products list with zeroes so your next sub-product starts in the right column. It also makes it easier to keep the columns lined up and it's easier to add up without errors.

\begin{center}
\begin{tabular}{c@{\,}c@{\,}c@{\,}c@{\,}c}
       &1,&2&3&4\\
\times  & & & &6\\
\hline
        & & &2&4\\
   +& &^{1}1&8&0\\
\hline
        &1&2&0&0\\
      + &6&0&0&0\\
\hline
       &7,&4&0&4\\
\hline
\hline
\end{tabular}\\
\end{center}

You make this much shorter by doing the carries for the addition at the same time as you are working out the sub-products.

\begin{center}
\begin{tabular}{c@{\,}c@{\,}c@{\,}c@{\,}c}
          &1,&2&3&4\\
\times &_1_&_2&_2&6\\
\hline
          &7,&4&0&4\\
\hline
\hline
\end{tabular}\\
\end{center}

\newpage

\subsection*{Multiplying two multi-digit numbers}
This is done by multiplying each digit of the multiplier by each digit of the multiplicand, starting with units.

\begin{center}
\begin{tabular}{c@{\,}c@{\,}c@{\,}c@{\,}c}
       & & &2&3\\
\times & & &3&4\\
\hline
       & & &1&2\\
      +& & &8& \\
\hline
       & & &9& \\
  +& &^{1}6& & \\
\hline
       & &7&8&2\\
\hline
\hline
\end{tabular}\\
\end{center}

\vspace{32pt}
Again, fill in the blank spaces with zeroes to make it easier to add up without errors.\\

\begin{center}
\begin{tabular}{c@{\,}c@{\,}c@{\,}c@{\,}c}
       & & &2&3\\
\times & & &3&4\\
\hline
       & & &1&2\\
      +& & &8&0\\
\hline
       & & &9&0\\
  +& &^{1}6&0&0\\
\hline
       & &7&8&2\\
\hline
\hline
\end{tabular}\\
\end{center}

\newpage

This can be shortened by doing some of the carrying while you're writing the partial products.

\begin{center}
\begin{tabular}{c@{\,}c@{\,}c@{\,}c@{\,}c}
       &&&2&3\\
\times &&&_{1}3&4\\
\hline
       &&&9&2\\
+ &&^{1}6&9&0\\
\hline
      &&7&8&2\\
\hline
\hline
\end{tabular}\\
\end{center}

\vspace{32pt}
And that is how you multiply.\\

\newpage

Here is longer example:

\begin{center}
\begin{tabular}{c@{\,}c@{\,}c@{\,}c@{\,}c@{\,}c@{\,}c}
       & & & &8&9&7\\
\times & & & &7&8&9\\
\hline
       & & & & &6&3\\
   & & & &^{1}8&1& \\
  +& & &^{2}7&2& & \\
\hline
       & & & &5&6& \\
       & & &7&2& & \\
  +& &^{2}6&4& & & \\
\hline
       & & &4&9& & \\
       & &6&3& & & \\
  +&^{2}5&6& & & & \\
\hline
      &7&0&7,&7&3&3 \\
\hline
\hline
\end{tabular}\\
\end{center}

\vspace{32pt}
See how there are partial products for the units, the tens, and the hundreds of the multiplier, each aligned with its digit of the multiplier.\\

\newpage

Padding with zeroes makes adding easier and prevents errors by keeping columns in line.

\begin{center}
\begin{tabular}{c@{\,}c@{\,}c@{\,}c@{\,}c@{\,}c@{\,}c}
       & & & &8&9&7\\
\times & & & &7&8&9\\
\hline
       & & & & &6&3\\
   & & & &^{1}8&1&0\\
  +& & &^{2}7&2&0&0\\
\hline
       & & & &5&6&0\\
       & & &7&2&0&0\\
  +& &^{2}6&4&0&0&0\\
\hline
       & & &4&9&0&0\\
       & &6&3&0&0&0\\
  +&^{2}5&6&0&0&0&0\\
\hline
      &7&0&7,&7&3&3\\
\hline
\hline
\end{tabular}\\
\end{center}

\vspace{32pt}
You can make this shorter by doing carries while working out the partial products. Here padding with zeroes just reminds you where to write the partial products. This is the usual way to do multiplication of large numbers by hand.

\begin{center}
\begin{tabular}{c@{\,}c@{\,}c@{\,}c@{\,}c@{\,}c@{\,}c}
           &&&&8&9&7\\
    \times &&&&7&8&9\\
  _6&_4&_7&_5&_8&_6&\\
\hline
  &&&^{1}8&^{1}0&7&3\\
     &&^{1}7&1&7&6&0\\
    &^{1}6&2&7&9&0&0\\
\hline
       &7&0&7,&7&3&3\\
\hline
\hline
\end{tabular}\\
\end{center}

\newpage

\subsection*{Multiplication with Decimal Fractions}

In doing multiplication in columns, first fill with trailing zeroes so that both numbers have the same number of digits after the decimal point, so that the columns are aligned properly.\\

The number of digits after the decimal point of the product is the total of the number of digits after the decimal point of the multiplicand and of the multiplier.\\

\begin{center}
\begin{tabular}{c@{\,}c@{\,}c@{\,}c@{\,}c@{\,}c@{\,}c@{\,}c@{\,}c@{\,}c@{\,}}
       & & & & & &2&.&3&4\\
\times & & & & & &5&.&2&0\\
\hline
       & & & & & & & & &0\\
       & & & & & & & &0& \\
+      & & & & & &0& & & \\
\hline
       & & & & & & & &8& \\
       & & & & & &6& & & \\
+      & & & &4& & & & & \\
\hline
       & & & &2& &0& & & \\
   & &^{1}1& &5& & & & & \\
+      &1&0& & & & & & & \\
\hline
       &1&2&.&1& &6& &8&0\\
\hline
\hline
\end{tabular}\\
\end{center}

Padding the partial products with zeroes makes the columns easier to see and can prevent errors, and you don't actually  need to multiply out the first zero.

\begin{center}
\begin{tabular}{c@{\,}c@{\,}c@{\,}c@{\,}c@{\,}c@{\,}c@{\,}c@{\,}c@{\,}c@{\,}}
       & & & & & &2&.&3&4\\
\times & & & & & &5&.&2&0\\
\hline
       & & & & & & & &8&0\\
       & & & & & &6& &0&0\\
+      & & & &4& &0& &0&0\\
\hline
       & & & &2& &0& &0&0\\
   & &^{1}1& &5& &0& &0&0\\
+      &1&0& &0& &0& &0&0\\
\hline
       &1&2&.&1& &6& &8&0\\
\hline
\hline
\end{tabular}\\
\end{center}

\vspace{32pt}
You can make this even shorter if you do carries while working out the partial products, and you don't really need to add the trailing zero to the multiplier as long as the decimal points line up.

\begin{center}
\begin{tabular}{c@{\,}c@{\,}c@{\,}c@{\,}c@{\,}c@{\,}c@{\,}c@{\,}c@{\,}c@{\,}}
       & & & & & &2&.&3&4\\
\times & &_1& &_2& &5&.&2& \\
\hline
       & & & &4& &6& &8& \\
  +&1&^{1}1& &7& &0& &0& \\
\hline
       &1&2&.&1& &6& &8& \\
\hline
\hline
\end{tabular}\\
\end{center}

\newpage

\subsection*{Multiplying from Left to Right}

Multiplication is usually done from right to left, but mental multiplication can be more easily done working from left to right.

\begin{figure}[ht]
\begin{minipage}[b]{0.5\linewidth} \centering 
\begin{center}
\begin{tabular}{c@{\,}c@{\,}c}
       &8&3\\
\times & &9\\
\hline
       &2&7\\
      7&2&0\\
\hline
      7&4&7\\
\hline
\hline
\end{tabular}\\
\vspace{8pt}\text{Left to Right}\\
            \text{(usual method)}}
\end{center}
\end{minipage}
\begin{minipage}[b]{0.5\linewidth} \centering 
\begin{center}
\begin{tabular}{c@{\,}c@{\,}c}
       &8&3\\
\times & &9\\
\hline
      7&2&0\\
       &2&7\\
\hline
      7&4&7\\
\hline
\hline
\end{tabular}\\
\vspace{8pt}\text{Right to Left}\\
            \text{(better mentally)}
\end{center}
\end{minipage}
\end{figure}

\section{Checking Multiplication}

\paragraph{Reversing Order}
You can check the result of your multiplication by reversing the multiplicand and multiplier to verify that they both give the same product.\\

\begin{figure}[ht]
\begin{minipage}[b]{0.5\linewidth} \centering 
\begin{center}
\begin{tabular}{c@{\,}c@{\,}c@{\,}c@{\,}c}
       &&&2&3\\
\times &&&_{1}4&2\\
\hline
       &&&4&6\\
     + &&9&2&0\\
\hline
       &&9&6&6\\
\hline
\hline
\end{tabular}\\
\end{center}
\end{minipage}
\begin{minipage}[b]{0.5\linewidth} \centering 
\begin{center}
\begin{tabular}{c@{\,}c@{\,}c@{\,}c@{\,}c}
       &&&4&2\\
\times &&&2&3\\
\hline
       &&1&2&6\\
     + &&8&4&0\\
\hline
       &&9&6&6\ \checkmark\\
\hline
\hline
\end{tabular}\\
\end{center}
\end{minipage}
\end{figure}
\begin{center}
$23 \times 42$ should equal $42 \times 23$.\\
\end{center}

\newpage

\paragraph{Casting Out 9s}
The digit sums of the factors, multiplied, should equal the digit sum of product.

\begin{figure}[ht]
\begin{minipage}[b]{0.5\linewidth} \centering 
\begin{center}
\begin{tabular}{c@{\,}c@{\,}c@{\,}c@{\,}c}
       &&&2&3\\
\times &&&_{1}4&2\\
\hline
       &&&4&6\\
     + &&9&2&0\\
\hline
       &&9&6&6\\
\hline
\hline
\end{tabular}\\
\end{center}
\end{minipage}
\begin{minipage}[b]{0.5\linewidth} \centering 
\begin{center}
\begin{align*}
2 + 3 &= 5\\
4 + 2 &= 6\\
\cancel{9} + 6 + 6 &= 12; 1 + 2 = 3\\\\
5 \times 6 &= 30; 3 + 0 = 3\\
\end{align*}
\end{center}
\end{minipage}
\end{figure}

\section{Other Methods of Multiplication}

\subsection*{Sum of Differences\\
and Product of Differences}

$$97 \times 96 = 9,312 :$$

\begin{tikzpicture} 
\path ( 0, 0) node (a1) {$97$};
\path ( 2, 0) node (b1) {$\times$};
\path ( 4, 0) node (c1) {$96$};
\path ( 6, 0) node (d1) {$=$};
\path ( 8, 0) node (e1) {$93$};
\path ( 10, 0) node (f1) {$12$};
\path ( 0, -1) node (a2) {$\overbrace{100-97}$};
\path ( 4, -1) node (c2) {$\overbrace{100-96}$};
\path ( 8, -1) node (e2) {$\overbrace{100-7}$};
\path ( 0, -2.5) node (a3) {$3$};
\path ( 2, -2.5) node (b3) {$+$};
\path ( 4, -2.5) node (c3) {$4$};
\path ( 8, -2.5) node (e3) {$7$};
\path ( 2, -3.7) node (b5) {$\times$};
\path ( 10, -3.5) node (f5) {};
\draw [->] (a2) -- (a3);
\draw [->] (c2) -- (c3);
\draw [->] (e3) -- (e2);
\draw [->] (c3) -- (e3);
\draw (a3) -- (0,-3.1) -- (4,-3.1) -- (c3);
\draw [->] (b5) -- (10,-3.7) -- (f1);
\end{tikzpicture}

\pagebreak
\subsection*{Approximation Method}
\begin{itemize}
\item Approximate each factor\\to the nearest multiple of 10 or 100.
\item These must be multiples of each other.\\Note the multiple.
\item Note under each factor the difference between it and its approximation.
\item Note under the smaller factor the product of its difference and the multiplier.
\item Add the other factor.
\item Multiply by the approximation of this factor.
\item Add the product of the differences.
\end{itemize}

\begin{tabular}{c@{\,}c@{\,}c@{\,}c@{\,}c@{\,}c@{\,}c@{\,}c@{\,}r}
        & &2&0&            &8&0&${(\times 4)$&            (approximations)\\
        & &1&8&\ $\times\ $&7&7& &                               (factors)\\
        & &-&2&            &-&3& &                           (differences)\\
        & &-&8&            & & & &          (difference$\times$\ multiple)\\
        &+&7&7&            & & & &                        (+ other factor)\\ \cline{2-4}
        & &6&9&            & & & &\\
 &$\times$&2&0&            & & & &                ($\times$ approximation)\\ \cline{1-4}
       1&3&8&0&            & & & &\\
        & &+&6&            & & & &              (+ product of differences)\\ \cline{1-4}
       1&3&8&6&            & & & &\\ \cline{1-4}
\end{tabular}

\subsection*{Base Method}

\begin{itemize}
\item Select some multiple of a power of 10 that is near both factors.
\item Subtract this base number from each factor and note the differences.
\item To one factor, add the difference of the other factor minus the base number.
\item Multiply that by the base number.
\item Add the product of the differences.
\end{itemize}

\begin{tabular}{c@{\,}c@{\,}c@{\,}c@{\,}c@{\,}c@{\,}c@{\,}c@{\,}c@{\,}}
        &1&4&$\times$&1&2\ = \ &1&6&8\\
        & & &\ 10\   & &       & & &\\
        & &+4&       &+2&      & & &\\\\
        & 1&4&&&&&&\\
        & +&2&&&&&&\\\cline{2-3}
        & 1&6&&&&&&\\
        $\times$&1&0&&&&&&\\\cline{2-4}
        & 1&6&0&&&&&\\
        +&4&$\times$&2&&&&&\\\cline{2-4}
        & 1&6&8&&&&&\\\cline{2-4}
\end{tabular}

\pagebreak

\subsection*{\=Urdhva Biryag Bhy\=am\\
(upwards-sideways) Method}

This method is from the Vedas which are ancient writings from India. It is also known as cross or vertical multiplication. Each partial product is written from left to right with tens carried to the last units digit. Here is an example showing the order of multiplications.\\

\noindent
\begin{minipage}{.5\textwidth}
\begin{tikzpicture}
\path (1,1) node {5};
\path (2,1) node {4};
\path (3,1) node {3};
\path (4,1) node {2};
\path (0,-0.7) node {$\times$};
\path (1,-0.7) node {3};
\path (2,-0.7) node {1};
\path (3,-0.7) node {2};
\path (4,-0.7) node {4};
\draw [->] (1,-0.2) -- (1,0.5);
\draw (0.5,-1.2) -- (4.5,-1.2);
\path (1,-1.6) node [align=left] {15};
\end{tikzpicture}
$$3\times5=15$$
\end{minipage}
\begin{minipage}{.5\textwidth}
\begin{tikzpicture}
\path (1,1) node {5};
\path (2,1) node {4};
\path (3,1) node {3};
\path (4,1) node {2};
\path (0,-0.7) node {$\times$};
\path (1,-0.7) node {3};
\path (2,-0.7) node {1};
\path (3,-0.7) node {2};
\path (4,-0.7) node {4};
\draw [->] (1,-0.2) -- (2,0.5);
\draw [->] (2,-0.2) -- (1,0.5);
\draw (0.5,-1.2) -- (4.5,-1.2);
\path (1.3,-1.6) node [align=left] {$15^{1}7$};
\end{tikzpicture}
$$3\times4+1\times5=17$$
\end{minipage}

\vspace{32pt}
\noindent
\begin{minipage}{.5\textwidth}
\begin{tikzpicture}
\path (1,1) node {5};
\path (2,1) node {4};
\path (3,1) node {3};
\path (4,1) node {2};
\path (0,-0.7) node {$\times$};
\path (1,-0.7) node {3};
\path (2,-0.7) node {1};
\path (3,-0.7) node {2};
\path (4,-0.7) node {4};
\draw [->] (1,-0.3) -- (3,0.5);
\draw [->] (3,-0.3) -- (1,0.5);
\draw [->] (2,-0.3) -- (2,0.5);
\draw (0.5,-1.2) -- (4.5,-1.2);
\path (1.6,-1.6) node {$15^{1}7^{2}3$};
\end{tikzpicture}
$$3\times3+2\times5+1\times4=23$$
\end{minipage}
\begin{minipage}{.5\textwidth}
\begin{tikzpicture}
\path (1,1) node {5};
\path (2,1) node {4};
\path (3,1) node {3};
\path (4,1) node {2};
\path (0,-0.7) node {$\times$};
\path (1,-0.7) node {3};
\path (2,-0.7) node {1};
\path (3,-0.7) node {2};
\path (4,-0.7) node {4};
\draw [->] (1,-0.3) -- (4,0.5);
\draw [->] (2,-0.3) -- (3,0.5);
\draw [->] (3,-0.3) -- (2,0.5);
\draw [->] (4,-0.3) -- (1,0.5);
\draw (0.5,-1.2) -- (4.5,-1.2);
\path (1.9,-1.6) node {$15^{1}7^{2}3^{3}7$};
\path (3, -2.6) node {$3\times2+1\times3$};
\path (3.5, -3.3) node {$+\ 2\times4+4\times5=37$};
\end{tikzpicture}
\end{minipage}

\vspace{32pt}

\noindent
\begin{minipage}{.5\textwidth}
\begin{tikzpicture}
\path (1,1) node {5};
\path (2,1) node {4};
\path (3,1) node {3};
\path (4,1) node {2};
\path (0,-0.7) node {$\times$};
\path (1,-0.7) node {3};
\path (2,-0.7) node {1};
\path (3,-0.7) node {2};
\path (4,-0.7) node {4};
\draw [->] (2,-0.3) -- (4,0.5);
\draw [->] (3,-0.3) -- (3,0.5);
\draw [->] (4,-0.3) -- (2,0.5);
\draw (0.5,-1.2) -- (4.5,-1.2);
\path (2.1,-1.6) node {$15^{1}7^{2}3^{3}7^{2}4$};
\end{tikzpicture}
$$1\times2+4\times4+2\times3=24$$
\end{minipage}
\begin{minipage}{.5\textwidth}
\begin{tikzpicture}
\path (1,1) node {5};
\path (2,1) node {4};
\path (3,1) node {3};
\path (4,1) node {2};
\path (0,-0.7) node {$\times$};
\path (1,-0.7) node {3};
\path (2,-0.7) node {1};
\path (3,-0.7) node {2};
\path (4,-0.7) node {4};
\draw [->] (3,-0.3) -- (4,0.5);
\draw [->] (4,-0.3) -- (3,0.5);
\draw (0.5,-1.2) -- (4.5,-1.2);
\path (2.3,-1.6) node {$15^{1}7^{2}3^{3}7^{2}4^{1}6$};
\end{tikzpicture}
$$2\times2+4\times3=16$$
\end{minipage}

\vspace{32pt}
\noindent
\begin{minipage}{.5\textwidth}
\begin{tikzpicture}
\path (1,1) node {5};
\path (2,1) node {4};
\path (3,1) node {3};
\path (4,1) node {2};
\path (0,-0.7) node {$\times$};
\path (1,-0.7) node {3};
\path (2,-0.7) node {1};
\path (3,-0.7) node {2};
\path (4,-0.7) node {4};
\draw [->] (4,-0.3) -- (4,0.5);
\draw (0.5,-1.2) -- (4.5,-1.2);
\path (2.3,-1.6) node {$15^{1}7^{2}3^{3}7^{2}4^{1}68$};
\end{tikzpicture}
$$4\times2=8$$
\end{minipage}
\begin{minipage}{.5\textwidth}
Adding up the carries:\\
\begin{tikzpicture}
\path (1,1) node {5};
\path (2,1) node {4};
\path (3,1) node {3};
\path (4,1) node {2};
\path (0,-0.7) node {$\times$};
\path (1,-0.7) node {3};
\path (2,-0.7) node {1};
\path (3,-0.7) node {2};
\path (4,-0.7) node {4};
\draw (0.5,-1.2) -- (4.5,-1.2);
\path (2.3,-1.6) node {$15^{1}7^{2}3^{3}7^{2}4^{1}68$};
\path (2.3,-2.4) node {$=16,969,568$.};
\draw (0.5,-2.8) -- (4.5,-2.8);
\draw (0.5,-2.9) -- (4.5,-2.9);
\end{tikzpicture}
\end{minipage}

\newpage
\
\newpage

\doublespacing

\begin{center}

Enquiries

\textbf{Applied Scholastics Ferndale}

Principal: Paula McLennan

mobile phone: 0431 683 306

email address: apsferndale@gmail.com

website: apsferndale.webs.com

\end{center}

\end{spacing}

\end{document}