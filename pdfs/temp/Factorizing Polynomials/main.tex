\documentclass[12pt]{article}
\usepackage{amsmath}
\usepackage{amssymb}
\usepackage{multicol}
\usepackage{graphicx}

\title{\textbf{Factorizing Polynomials}}\\
\author{Tutoring Centre Ferndale\\
\includegraphics[width=4em]{ApS_logo.png}}
\date{}

\begin{document}

\maketitle

Factorizing polynomials simplifies expressions and solves equations.

\section*{Perfect Squares}
\begin{itemize}
\item A trinomial of the form $a^2 + 2ab + b^2$ factors into $(a+b)^2$.
\item A trinomial of the form $a^2 - 2ab + b^2$ factors into $(a-b)^2$.
\end{itemize}

\textbf{Examples}
\begin{align*}
  x^2 + 6x + 9 &= (x+3)^2 \\
4x^2 - 12x + 9 &= (2x-3)^2 \\
y^2 + 10y + 25 &= (y+5)^2
\end{align*}

\subsection*{Exercises}
Factor the following trinomials:
\begin{multicols}{2}
\begin{enumerate}
    \item $x^2 + 8x + 16$
    \item $9x^2 - 24x + 16$
    \item $4y^2 + 12y + 9$
    \item $z^2 - 14z + 49$
\end{enumerate}
\end{multicols}

\newpage

\subsection*{Answers}
\begin{multicols}{2}
\begin{enumerate}
\item $x^2 + 8x + 16 = (x+4)^2$
\item $9x^2 - 24x + 16 = (3x-4)^2$
\item $4y^2 + 12y + 9 = (2y+3)^2$
\item $z^2 - 14z + 49 = (z-7)^2$
\end{enumerate}
\end{multicols}

\section*{Difference of Squares}
A binomial of the form $a^2 - b^2$ factors into $(a+b)(a-b)$.

\subsection*{Examples}
\begin{align*}
1.\ & x^2 - 9 = (x+3)(x-3) \\
2.\ & 4x^2 - 25 = (2x+5)(2x-5) \\
3.\ & y^2 - 16 = (y+4)(y-4)
\end{align*}

\subsection*{Exercises}
Factor the following expressions:
\begin{multicols}{2}
\begin{enumerate}
    \item $x^2 - 36$
    \item $49x^2 - 1$
    \item $25y^2 - 64$
    \item $z^2 - 81$
\end{enumerate}
\end{multicols}

\subsection*{Answers}
\begin{multicols}{2}
\begin{enumerate}
\item $x^2 - 36 = (x+6)(x-6)$
\item $49x^2 - 1 = (7x+1)(7x-1)$
\item $25y^2 - 64 = (5y+8)(5y-8)$
\item $z^2 - 81 = (z+9)(z-9)$
\end{enumerate}
\end{multicols}

\newpage
\section*{General Factorization Techniques}
There are several general methods for factorizing polynomials. These include:
\begin{itemize}
    \item Factoring out the Greatest Common Factor (GCF)
    \item Factoring by Grouping
    \item Factoring Trinomials
    \item Special Cases (Cubic Polynomials)
\end{itemize}

\subsection*{Factoring out the Greatest Common Factor (GCF)}
The first step in many factoring problems is to factor out the GCF, which is the largest factor common to all terms in the polynomial.

\textbf{Examples}
\begin{align*}
1.\ & 6x^2 + 9x = 3x(2x + 3) \\
2.\ & 8x^3 - 4x^2 + 2x = 2x(4x^2 - 2x + 1) \\
3.\ & 15x^4y - 25x^2y^2 = 5x^2y(3x^2 - 5y)
\end{align*}

\subsubsection*{Exercises}
Factor out the GCF:
\begin{multicols}{2}
\begin{enumerate}
    \item $12x^3 + 18x^2$
    \item $10x^2y + 15xy^2$
    \item $24x^4y^2 - 16x^3y$
    \item $14x^3 - 21x^2 + 28x$
\end{enumerate}
\end{multicols}

\subsubsection*{Answers}
\begin{multicols}{2}
\begin{enumerate}
\item $12x^3 + 18x^2 = 6x^2(2x + 3)$
\item $10x^2y + 15xy^2 = 5xy(2x + 3y)$
\item $24x^4y^2 - 16x^3y = 8x^3y(3x - 2)$
\item $14x^3 - 21x^2 + 28x\\ = 7x(2x^2 - 3x + 4)$
\end{enumerate}
\end{multicols}

\newpage

\subsection*{Factoring by Grouping}
When a polynomial has four or more terms, grouping terms to factor out a common factor can be effective.

\textbf{Examples}
\begin{align*}
x^3 + 3x^2 + x + 3 &= (x^3 + 3x^2) + (x + 3)\\
                   &= x^2(x + 3) + 1(x + 3)\\
                   &= (x^2 + 1)(x + 3)
\end{align*}

\begin{align*}
2x^3 + 4x^2 + 3x + 6 &= (2x^3 + 4x^2) + (3x + 6)\\
                     &= 2x^2(x + 2) + 3(x + 2)\\
                     &= (2x^2 + 3)(x + 2)
\end{align*}

\begin{align*}
3x^2y - 6xy + 2x - 4 &= (3x^2y - 6xy) + (2x - 4)\\
                     &= 3xy(x - 2) + 2(x - 2)\\
                     &= (3xy + 2)(x - 2)
\end{align*}

\subsubsection*{Exercises}
Factor the following polynomials by grouping:
\begin{multicols}{2}
\begin{enumerate}
    \item $x^3 - 2x^2 + x - 2$
    \item $2x^2 + 6x + x + 3$
    \item $4xy + 8y - x - 2$
    \item $3x^2 + 5x + 6x + 10$
\end{enumerate}
\end{multicols}

\subsubsection*{Answers}
\begin{align*}
1.\ x^3 - 2x^2 + x - 2  &= (x^3 - 2x^2) + (x - 2)\\
                        &= x^2(x - 2) + 1(x - 2)\\
                        &= (x^2 + 1)(x - 2)
\end{align*}
\begin{align*}
2.\ 2x^2 + 6x + x + 3   &= (2x^2 + 6x) + (x + 3)\\
                        &= 2x(x + 3) + 1(x + 3)\\
                        &= (2x + 1)(x + 3)
\end{align*}
\begin{align*}
3.\ 4xy + 8y - x - 2    &= (4xy + 8y) - (x + 2)\\
                        &= 4y(x + 2) - 1(x + 2)\\
                        &= (4y - 1)(x + 2)
\end{align*}
\begin{align*}
4.\ 3x^2 + 5x + 6x + 10 &= (3x^2 + 5x) + (6x + 10)\\
                        &= x(3x + 5) + 2(3x + 5)\\
                        &= (x + 2)(3x + 5)
\end{align*}
\end{align*}

\subsection*{Factoring Trinomials}
Trinomials of the form $ax^2 + bx + c$ can often be factored into the product of two binomials.

\subsubsection*{Examples}
\begin{align*}
1.\ & x^2 + 5x + 6 = (x + 2)(x + 3) \\
2.\ & 2x^2 + 7x + 3 = (2x + 1)(x + 3) \\
3.\ & 3x^2 - x - 4 = (3x + 4)(x - 1)
\end{align*}

\subsubsection*{Exercises}
Factor the following trinomials:
\begin{multicols}{2}
\begin{enumerate}
    \item $x^2 + 7x + 12$
    \item $2x^2 + 3x - 2$
    \item $3x^2 + 8x + 4$
    \item $4x^2 + 11x + 6$
\end{enumerate}
\end{multicols}

\subsubsection*{Answers}
\begin{multicols}{2}
\begin{align*}
1.\ & x^2 + 7x + 12 = (x + 3)(x + 4) \\
2.\ & 2x^2 + 3x - 2 = (2x - 1)(x + 2) \\
3.\ & 3x^2 + 8x + 4 = (3x + 2)(x + 2) \\
4.\ & 4x^2 + 11x + 6 = (4x + 3)(x + 2)
\end{align*}
\end{multicols}

\newpage

\section*{Sum and Difference of Cubes}

\subsection*{Sum of Cubes}
The sum of cubes is a polynomial of the form $a^3 + b^3$. It factors into:
\[
a^3 + b^3 = (a + b)(a^2 - ab + b^2)
\]

\subsubsection*{ Examples}
\begin{align*}
1.\ & x^3 + 8 = x^3 + 2^3 = (x + 2)(x^2 - 2x + 4) \\
2.\ & 27x^3 + 1 = (3x)^3 + 1^3 = (3x + 1)(9x^2 - 3x + 1) \\
3.\ & y^3 + 125 = y^3 + 5^3 = (y + 5)(y^2 - 5y + 25)
\end{align*}

\subsubsection*{Exercises}
Factor the following sums of cubes:
\begin{multicols}{2}
\begin{enumerate}
    \item $x^3 + 27$
    \item $8x^3 + 1$
    \item $y^3 + 64$
    \item $125x^3 + 8$
\end{enumerate}
\end{multicols}

\subsubsection*{Answers}
\begin{align*}
1.\ & x^3 + 27 = (x + 3)(x^2 - 3x + 9) \\
2.\ & 8x^3 + 1 = (2x + 1)(4x^2 - 2x + 1) \\
3.\ & y^3 + 64 = (y + 4)(y^2 - 4y + 16) \\
4.\ & 125x^3 + 8 = (5x + 2)(25x^2 - 10x + 4)
\end{align*}

\newpage
\subsection*{Difference of Cubes}
The difference of cubes is a polynomial of the form $a^3 - b^3$. It factors into:
\[
a^3 - b^3 = (a - b)(a^2 + ab + b^2)
\]
The difference of cubes is similar to the sum of cubes, except that the sign between the two cubes is negative.

\subsubsection*{ Examples}
\begin{align*}
1.\ & x^3 - 8 = x^3 - 2^3 = (x - 2)(x^2 + 2x + 4) \\
2.\ & 27x^3 - 1 = (3x)^3 - 1^3 = (3x - 1)(9x^2 + 3x + 1) \\
3.\ & y^3 - 125 = y^3 - 5^3 = (y - 5)(y^2 + 5y + 25)
\end{align*}

\subsubsection*{Exercises}
Factor the following differences of cubes:
\begin{multicols}{2}
\begin{enumerate}
    \item $x^3 - 27$
    \item $8x^3 - 1$
    \item $y^3 - 64$
    \item $125x^3 - 8$
\end{enumerate}
\end{multicols}

\subsubsection*{Answers}
\begin{align*}
1.\ & x^3 - 27 = (x - 3)(x^2 + 3x + 9) \\
2.\ & 8x^3 - 1 = (2x - 1)(4x^2 + 2x + 1) \\
3.\ & y^3 - 64 = (y - 4)(y^2 + 4y + 16) \\
4.\ & 125x^3 - 8 = (5x - 2)(25x^2 + 10x + 4)
\end{align*}

\end{document}
