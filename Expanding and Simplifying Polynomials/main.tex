\documentclass[12pt]{article}
\usepackage{amsmath, amssymb, amsthm}
\usepackage{hyperref}
\usepackage{enumitem}
\usepackage{graphicx}

\title{\textbf{Expanding \& Simplifying Polynomials}}
\author{Tutoring Centre Ferndale\\
\includegraphics[width=4em]{ApS_logo.png}}
\date{}

\maketitle

\begin{document}

Polynomials are algebraic expressions consisting of variables and coefficients, constructed using only addition, subtraction, multiplication, and non-negative integer exponents of variables. Expanding and simplifying polynomials are fundamental skills in algebra that are essential for solving equations, analyzing functions, and more.

\section*{Definitions}
\subsection*{Polynomials}
A \textbf{polynomial} in one variable \(x\) is an expression of the form:
\[
P(x) = a_nx^n + a_{n-1}x^{n-1} + \dots + a_1x + a_0
\]
where \(a_n, a_{n-1}, \dots, a_0\) are coefficients and \(n\) is a non-negative integer.

\subsection*{Degree of a Polynomial}
The \textbf{degree} of a polynomial is the highest power of the variable in the expression.

\subsection*{Leading Coefficient}
The \textbf{leading coefficient} is the coefficient of the term with the highest power.

\newpage

\section*{Methods for Expanding Polynomials}
\subsection*{The Distributive Property}
The distributive property states that:
\[
a(b + c) = ab + ac
\]
This property is fundamental in expanding polynomials.

\subsubsection*{Example}
Expand \( 3(x + 4) \).

\[
3(x + 4) = 3 \cdot x + 3 \cdot 4 = 3x + 12
\]

\subsubsection*{Example}
Expand \( 2(3x - 5) \).

\[
2(3x - 5) = 2 \cdot 3x + 2 \cdot (-5) = 6x - 10
\]

\subsection*{FOIL Method}
The FOIL method is used to expand the product of two binomials:
\[
(a + b)(c + d) = ac + ad + bc + bd
\]
FOIL stands for First, Outer, Inner, Last, which is the order in which the terms are multiplied.

\subsubsection*{Example}
Expand \( (x + 2)(x + 3) \) using FOIL.

\[
(x + 2)(x + 3) = x \cdot x + x \cdot 3 + 2 \cdot x + 2 \cdot 3 = x^2 + 3x + 2x + 6 = x^2 + 5x + 6
\]

\newpage

\section*{Simplifying Polynomials}
Simplifying polynomials involves combining like terms and performing arithmetic operations to reduce the expression to its simplest form.

\subsection*{Combining Like Terms}
Like terms are terms that have the same variable raised to the same power. To combine like terms, add or subtract their coefficients.

\subsubsection*{Example}
Simplify \( 3x^2 + 5x - 2x^2 + 7x + 4 \).

\[
3x^2 - 2x^2 + 5x + 7x + 4 = (3 - 2)x^2 + (5 + 7)x + 4 = x^2 + 12x + 4
\]

\vfill

\subsection*{Factoring Polynomials}
Factoring is the reverse process of expanding. It involves writing the polynomial as a product of its factors.

\subsubsection*{Example}
Factor \( x^2 + 5x + 6 \).

Find two numbers that multiply to 6 and add to 5. These numbers are 2 and 3.
\[
x^2 + 5x + 6 = (x + 2)(x + 3)
\]

\vfill

\newpage

\section*{Special Polynomials}

\vfill

\subsection*{Difference of Squares}
\[
a^2 - b^2 = (a - b)(a + b)
\]
\subsubsection*{Example}
Expand \( x^2 - 9 \).

\[
x^2 - 9 = (x - 3)(x + 3)
\]

\vfill

\subsection*{Perfect Square Trinomial}
\[
a^2 + 2ab + b^2 = (a + b)^2 \\
a^2 - 2ab + b^2 = (a - b)^2
\]
\subsubsection*{Example}
Expand \( (x + 4)^2 \).

\[
(x + 4)^2 = x^2 + 2 \cdot x \cdot 4 + 4^2 = x^2 + 8x + 16
\]

\vfill

\newpage

\section*{Examples}

\vfill

\subsection*{Example}
Expand and simplify \( 2(x + 3)(x - 2) \).

First, expand \( (x + 3)(x - 2) \) using FOIL:
\[
(x + 3)(x - 2) = x^2 - 2x + 3x - 6 = x^2 + x - 6
\]
Now, multiply by 2:
\[
2(x^2 + x - 6) = 2x^2 + 2x - 12
\]

\vfill

\subsection*{Example}
Simplify \( 4x(x - 5) + 3(x^2 - x + 2) \).

First, distribute:
\[
4x(x - 5) = 4x^2 - 20x \\
3(x^2 - x + 2) = 3x^2 - 3x + 6
\]
Combine like terms:
\[
4x^2 - 20x + 3x^2 - 3x + 6 = (4x^2 + 3x^2) + (-20x - 3x) + 6 = 7x^2 - 23x + 6
\]

\vfill

\newpage

\section*{Practice Questions}
\subsection*{Question 1}
Expand \( 5(x - 4) \).\\

\textbf{Answer:} \ \(5(x - 4) = 5x - 20\)

\subsection*{Question 2}
Expand \( (2x + 3)(x - 5) \) using the FOIL method.\\

\textbf{Answer:}\\
\(
(2x + 3)(x - 5) = 2x \cdot x + 2x \cdot (-5) + 3 \cdot x + 3 \cdot (-5) = 2x^2 - 10x + 3x - 15 = 2x^2 - 7x - 15
\)

\subsection*{Question 3}
Simplify \( 6x^2 - 2x + 4x^2 + x - 7 \).\\

\textbf{Answer:} \ \(6x^2 + 4x^2 - 2x + x - 7 = 10x^2 - x - 7\)

\subsection*{Question 4}
Factor \( x^2 - 16 \).\\

\textbf{Answer:} \ \(x^2 - 16 = (x - 4)(x + 4)\)

\section*{Answers to Practice Questions}
\begin{enumerate}
    \item \textbf{Question 1 Answer:} \(5x - 20\)
    \item \textbf{Question 2 Answer:} \(2x^2 - 7x - 15\)
    \item \textbf{Question 3 Answer:} \(10x^2 - x - 7\)
    \item \textbf{Question 4 Answer:} \((x - 4)(x + 4)\)
\end{enumerate}

\end{document}
