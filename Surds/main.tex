\documentclass{article}
\usepackage{amsmath}
\usepackage[fontsize=16pt]{fontsize}
\usepackage{setspace}

\author{}
\date{}
\title{Surds\\
\vspace{28pt}
\begin{normalsize}Applied Scholastics, Ferndale WA \end{normalsize}}

\begin{document}
\maketitle
\pagebreak

\section*{Surds}

\setstretch{1.1}

A rational number is a number that can be written as a ratio of two whole numbers. $1, \frac{2}{3}, -4, \frac{273}{8}$ are all rational numbers.\\

Numbers that can’t be written as the ratio of two whole numbers are called irrational numbers.\\

For example, the relationship between the circumference of a circle and its diameter, known by the Greek letter $\pi$ (pi), which stands for ‘perimeter,’ cannot be written exactly as a ratio of any two whole numbers so it is called an irrational number.\\

Surds are numbers that can only be written as the root of an integer because they are not a rational number. ‘Surd’ is a Latin word meaning ‘deaf, mute’ because a surd is a value that cannot be expressed as a number.\\

$\sqrt{9}$ is not a surd because $\sqrt{9}=3$ exactly.\\

\indent $\sqrt{2}$ is a surd because $\sqrt{2}\approx 1.41421\ldots$ and can only be written exactly as $\sqrt{2}$.

\newpage

\subsection*{Surd Laws}
Surds follow laws similar to the Power Laws:
\begin{Large}
\setstretch{2.1}
\begin{align*}
x^{\frac{m}{n}}&=\sqrt[n]{x^m}=(\sqrt[n]{x})^m\\
x^{\frac{1}{n}}&=\sqrt[n]{a}\\
x^{\frac{m}{n}}&=(a^{\frac{1}{n}})^m\\
\sqrt{ mn}&=\sqrt{m} \cdot \sqrt{n}\\
\sqrt{\frac{m}{n}}&=\frac{\sqrt{m}}{\sqrt{n}}\\
(\sqrt{n})^2&=n
\end{align*}
\end{Large}
\singlespacing

\newpage

\subsection*{Simplification of Surds}

\doublespacing
Simplifying a surd to its simplest possible form when it appears in some problem makes the rest of the problem easier.\\

To simplify a surd expression, list the factors of the radicand, choose two of them, one of which must be a square number, and then use the rule $\sqrt{mn}=\sqrt{m}\sqrt{n}$.\\

(Remember that ‘radical’ means ‘root,’ and is the name of the $\surd$ symbol and ‘radicand’ is the word for the value in the radical symbol .)\\

\newpage

\doublespacing
\begin{align*}
\text{e.g. }
\sqrt{18}
&=\sqrt{2 \times 9}\\
&=\sqrt{2} \times \sqrt{9}\\
&=\sqrt{2} \times 3\\
&=3\sqrt{2}\\
\end{align*}

\begin{align*}
\text{e.g. }
\sqrt{147}-2\sqrt{12}
&=\sqrt{49 \cdot 3}-2\sqrt{4 \cdot 3}\\
&=\sqrt{49} \cdot \sqrt{3}-2\sqrt{4} \cdot \sqrt{3}\\
&=7\sqrt{3}-2 \cdot 2\sqrt{3}\\
&=7\sqrt{3}-4\sqrt{3}\\
&=3\sqrt{3}
\end{align*}
\singlespacing

\newpage

\subsection*{Rationalizing the Denominator}
\setstretch{1.1}
Rationalizing means turning something into a ratio, such as a fraction. The denominator means the number on the bottom of a fraction which says what sort of fraction it is. So rationalizing the denominator just means getting rid of the irrational surd on the bottom of a fraction.\\

That makes it easier to calculate a decimal number value for a surd, and it is easier generally to have the surds in the numerator rather than in the denominator of expressions.\\

Working out $\frac{3}{\sqrt{2}}$ requires long division of $\frac{3}{1.4142\ldots}$.\\

Instead, multiply the numerator and denominator by $\sqrt{2}$:

$$\frac{3}{\sqrt{2}}\cdot
\frac{\sqrt{2}}{\sqrt{2}}
=\frac{3\sqrt{2}}{2}\\
\approx2.1213$$

\newpage

To rationalize \Large$\frac{a}{\sqrt{b}}$\normalsize multiply the numerator\\ and denominator by $\sqrt{b}$.

\begin{align*}
\text{e.g. }
\frac{1}{\sqrt{3}}
&=\frac{1}{\sqrt{3}}\cdot \frac{\sqrt{3}}{\sqrt{3}}\\
&=\frac{1\sqrt{3}}{\sqrt{3}\sqrt{3}}=\frac{\sqrt{3}}{3}
\end{align*}

\begin{align*}
\text{e.g. }
\frac{1}{\sqrt{3}}
&=\frac{1}{\sqrt{3}}\cdot\frac{\sqrt{3}}{\sqrt{3}}\\
&=\frac{1\sqrt{3}}{\sqrt{3}\sqrt{3}}=\frac{\sqrt{3}}{3}
\end{align*}

\begin{align*}
\text{e.g. }
\frac{\sqrt{3}}{\sqrt{2}}+\frac{2}{\sqrt{6}}
&=\frac{\sqrt{3}}{\sqrt{2}}\cdot
\frac{\sqrt{3}}{\sqrt{3}}
+\frac{2}{\sqrt{6}}\\
&=\frac{3}{\sqrt{6}}+\frac{2}{\sqrt{6}}
=\frac{5}{\sqrt{6}}\\
&=\frac{5}{\sqrt{6}}\cdot\frac{\sqrt{6}}{\sqrt{6}}\\
&=\frac{5\sqrt{6}}{6}
\end{align*}

\newpage

\subsubsection*{Using the Difference of Squares Formula}
\setstretch{1.2}

Quadratic means ‘to do with a square’ and is from the Latin word for four. A quadratic surd is a surd or an expression containing a surd that has an index of 2, or is a ‘square root.’\\

$\sqrt{2}$, $\sqrt{5}$, and $3\sqrt{10}$ are quadratic surds since the indices of their roots are 2.\\

It is possible to rationalize the denominator of a quadratic surd using the difference of squares formula.\\

The difference of squares formula is that the product of the sum and difference of two values is equal to the difference of their squares:\\

$$(a+b)(a-b)=a^2-b^2.$$

\newpage

A conjugate is a pair of joined or related opposites, such as the sum and difference $(a+b)(a-b)$.\\

For surds, the sum and difference of $n\sqrt{a}$ and $n\sqrt{b}$ are $(n\sqrt{a}+m\sqrt{b})$ and $(n\sqrt{a}-m\sqrt{b})$. These are said to be conjugate to each other.\\

Conjugates like these are useful because if you multiply a quadratic surd with its conjugate you get a rational number.\\

\begin{equation*}
\begin{split}
\textrm{e.g. }(3+\sqrt{2})(3-\sqrt{2})
&=3^2-{\sqrt{2}}^2\\
&=9-2=7\\
\end{split}

\vspace{32pt}
\begin{split}
\textrm{e.g. }(\sqrt{2}+\sqrt{3})(\sqrt{2}-\sqrt{3})
&={\sqrt{2}}^2-{\sqrt{3} }^2\\
&=2-3=-1
\end{split}
\end{equation*}

\newpage

To rationalize \Large$\frac{a}{b+\sqrt{c}}$\normalsize, multiply the numerator and denominator by the conjugate of the denominator, $b-\sqrt{c}$.

\setstretch{3}
\begin{equation*}
\begin{split}
\textrm{e.g. }
\frac{1}{\sqrt{7}+\sqrt{5}}
&=\frac{1}{\sqrt{7}+\sqrt{5}}
\cdot\frac{\sqrt{7}-\sqrt{5}}{\sqrt{7}-\sqrt{5}}\\
&=\frac{\sqrt{7}-\sqrt{5}}{(\sqrt{7})^2-(\sqrt{5})^2}\\
&=\frac{\sqrt{7}-\sqrt{5}}{7-5}\\
&=\frac{\sqrt{7}-\sqrt{5}}{-2}
\end{split}
\end{equation*}

\newpage

\begin{align*}
\text{e.g. }
\frac{3}{2+\sqrt{5}}
&=\frac{3}{2+\sqrt5}\cdot\frac{2-\sqrt{5}}{2-\sqrt{5}}\\
&=\frac{3(2-\sqrt{5})}{(2+\sqrt{5})(2-\sqrt{5})}\\
&=\frac{6-3\sqrt{5}}{4+2\sqrt{5}-2\sqrt{5}-5}\\
&=\frac{6-3\sqrt{5}}{-1}\\
&=3\sqrt{5}-6
\end{align*}

\newpage

\setstretch{4}
\begin{align*}
\text{e.g. }
\frac{\sqrt{3}+\sqrt{2}}
{3\sqrt{2}+2\sqrt{3}}
&=\frac{\sqrt{3}+\sqrt{2}}
{3\sqrt{2}+2\sqrt{3}}\cdot
\frac{3\sqrt{2}-2\sqrt{3}}
{3\sqrt{2}-2\sqrt{3}}\\
&=\frac{3\sqrt{2}\sqrt{3}
-2{\sqrt{3}}^2
+3{\sqrt{2}}^2
-2{\sqrt{3}}^2}
{(3\sqrt{2})^2-(2\sqrt{3})^2}\\
&=\frac{2\sqrt{3}-2{\sqrt{3}}^2+3{\sqrt{2}}^2}
{3^2{\sqrt{2}}^2-2^2{\sqrt{3}}^2}\\
&=\frac{\sqrt{6}-(2\cdot3)+(3\cdot2)}{(9\cdot2)-(4\cdot3)}\\
&=\frac{\sqrt{6}}{6}
\end{align*}

\newpage
\
\newpage
\
\newpage
\
\newpage
\

\begin{center}
\doublespacing
\large

Enquiries

\textbf{Applied Scholastics Ferndale}

Principal: Paula McLennan

mobile phone: 0431 683 306

email address: apsferndale@gmail.com

website: apsferndale.webs.com
\end{center}

\end{document}
