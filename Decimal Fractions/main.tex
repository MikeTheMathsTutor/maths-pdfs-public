\documentclass{article}
\usepackage{amsmath}
\usepackage{caption}
\usepackage{cancel}
\usepackage[fontsize=16pt]{fontsize}

\newcommand\mylongdiv[2]{%
$\strut#1$\kern.25em\smash{\raise.3ex\hbox{$\big)$}}$\mkern-8mu
        \overline{\quad\strut#2}$}

\author{}
\date{}
\title{Decimal\\Fractions\\
\vspace{28pt}
\begin{normalsize}Applied Scholastics, Ferndale WA \end{normalsize}}

\begin{document}
\maketitle
\newpage
\tableofcontents
\newpage

\section{Decimal Numbers}
The numbers that we usually use are called decimal numbers. Decimal means having to do with the number ten, and in decimal numbers the value of each digit is ten times the value of the digit to its left.\\

1234 &= (1 \times 1000) + (2 \times 100) + (3 \times 10) + 4.\\

\section{Decimal Points}

In decimal numbers, just as the value of each digit to the left is multiplied by ten, the value of each digit to the right is divided by ten. This can continue to the right past the units digit to show fractions.
\begin{align*}
12.34 &= (1 \times 10) + (2 \times 1)\\
      &+ (3 \div 10) + (4 \div100)\\
      &= 10 + 2 + \frac{3}{10} + \frac{4}{100}\\
\end{align*}

The decimal point is a full stop that marks where the whole number part of a number ends and where the fractional part starts.\\

\newpage

\subsection*{Radix Points}

Generally, the symbol used to separate the integer part of a number from the fractional part is called a radix point. Radix is the Latin word for root, meaning here the base of the number system being used. For base 10 the radix point is called the decimal point, but for base 2, for example, it is called a binary point.

\begin{equation*}
\begin{split}
\text{e.g. }$&101.01_2$\text{ (binary)}\\
            $&= 1 \times 4 + 0 \times 2 + 1 \times 1$\\
            $&\hspace*{1em}+ 0 \times \frac{1}{2} + 1 \times \frac{1}{4}$\\
            $&= 5.75_{10}$\text{ (decimal.)}
\end{split}
\end{equation*}

\newpage

\section{Decimal Fractions}

Fractions written as decimal numbers are called decimal fractions. They are usually the answers given by calculators, and they are the answers that you get when doing division by hand.\\

Decimal numbers are sometimes just called decimals, and decimal fractions are often just called decimals.\\

\paragraph{Leading and Trailing Zeroes}
Any number can be considered to have infinite leading zeroes before and infinite zeroes trailing after, but they are only used or written when needed.\\

For a number less than 1, the zero in the units position is still written to make it clear that the number is a fraction.\\

You write 0.25, not just .25.\\

Unnecessary trailing zeroes that come up in answers are removed.\\

Trailing zeroes in decimal fractions can be specifically included to show what level of accuracy is intended.\\

$0.2$ means accurate to $\frac{1}{10}^{\textrm{th}}$ of a unit but $0.200$ means accurate to $\frac{1}{1000}^{\textrm{th}}$ of a unit.

\newpage

\section{Remainders in Division}
When you get a remainder in division, you can either write it just as itself as a remainder, you can write it as a fraction with the remainder as the numerator and the divisor as the denominator, or you can write it as a decimal fraction.\\

You can do
\hspace*{4.1ex}1\hspace{1.3ex}1\hspace{0.9ex}2\\
\hspace*{7em}\mylongdiv{6}{6\hspace{1.1ex}7\hspace{0.4ex}{^1}5}\hspace{1em}remainder 3\\

Or
\hspace*{12.1ex}1\hspace{1.2ex}1\hspace{1.1ex}2\hspace{1ex}$\frac{3}{6} = 112 \frac{1}{2}$\\
\hspace*{7em}\mylongdiv{6}{6\hspace{1.1ex}7\hspace{0.4ex}{^1}5}\\

Or
\hspace*{12.1ex}1\hspace{1.3ex}1\hspace{0.9ex}2\hspace{0.7ex}.\hspace{0.8ex}5\\
\hspace*{7em}\mylongdiv{6}{6\hspace{1.1ex}7\hspace{0.4ex}{^1}5\hspace{0.3ex}.^30}\\

To make a decimal fraction when doing division like this, when you reach the last digit of the dividend and there is still a remainder, add a decimal point to both the dividend and the quotient and simply continue the procedure.\\

\newpage

\section{Repeating\\Decimal Fractions}

A decimal fraction  will either end with no further remainder, or it will start to repeat itself, or it may continue forever without repeating.\\

That is why it is sometimes better to write a remainder as itself or as a normal fraction rather than as a decimal fraction.\\

When a decimal fraction starts to repeat, draw a line or make a dot above the repeating part of the fraction. You don't keep working out a division past that point.\\

An endlessly repeating digit of a decimal fraction is shown by putting a dot above that digit, such as $1 \div 3 = 0.333\ldots=0.\dot{3}.$\\

An endlessly repeating series of digits such as in $22 \div 7 = 3.142857142857142857\dots$ is written as $3.\overline{142857}$ with a line drawn over the repeating part of the decimal.\\

\section{Irrational\\Decimal Fractions}

Every rational number can be written as either a repeating or terminating decimal fraction. All repeating decimal fractions can be converted to exact fractions or ratios.\\

Decimal fractions that do not repeat or terminate are irrational numbers. The digits will continue forever without repeating.\\

The value of $\pi = \frac{Circumference}{Diameter} \approx 3.14159...$ is a well known example of a non-terminating and non-repeating decimal fraction.\\

\newpage

\section{Converting a Repeating\\Decimal Fraction\\into a Fraction}

Create an equation setting the repeating decimal fraction  equal to $x$. Multiply that equation by a power of 10 so that the repeating part of the fraction is shifted to the left of the decimal point. Subtract the two equations.


\begin{align}
                            x&=0.\overdot{7}\\
\text{(1) $\times 10$ : } 10x&=7.\overdot{7}\\
      \text{(2) - (1) : }  9x&=7\\
                            x&=\frac{7}{9}
\end{align}

\setcounter{equation}{0}

\begin{align}
x&=0.\overline{24}\\
\text{(1) $\times 100$ : }
100x&=24.\overline{24}\\
\text{(2) - (1) : }
\ 99x&=24\\
x&=\frac{24}{99}=\frac{8}{33}
\end{align}

\newpage
\
\newpage
\
\newpage
\
\newpage
\
\newpage
\
\newpage
\
\newpage
\

\begin{center}
\linespread{2}\large

Enquiries

\textbf{Applied Scholastics Ferndale}

Principal: Paula McLennan

mobile phone: 0431 683 306

email address: apsferndale@gmail.com

website: apsferndale.webs.com
\end{center}

\end{document}