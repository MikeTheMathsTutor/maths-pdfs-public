\documentclass{article}

\usepackage[a4paper,margin=3cm]{geometry}
\usepackage{amsmath}
\usepackage{cancel}
\usepackage{setspace}
\usepackage[fontsize=16pt]{fontsize}

\author{}
\date{}
\title{Ratios,\\Proportions,\\and Percentages\\
\vspace{28pt}
\begin{normalsize}Applied Scholastics, Ferndale WA \end{normalsize}}

\begin{document}

\maketitle
\pagebreak
\begin{spacing}{1.25}

\section*{Ratios}
Ratio means relative size.\\

A ratio is shown by a colon between two numbers. 2:1 means that one thing is twice another thing. It is read as "to," so 2:1 is "2 to 1."\\

A ratio can be written as a fraction. $\frac{2}{1}$ means the same as 2:1.\\

Ratios can be simplified in the same way as fractions. A 6:8 ratio is the same as a 3:4 ratio.\\

The fractional form of a ratio indicates the size of one part in relation to the other. If there are 12 men and 20 women in a group, then the ratio of men to women is 12:20, which simplifies to 3:5. That means there are $\frac{3}{5}$ as many men as women, and $\frac{5}{3}$ as many women as men.\\

\newpage

For a ratio a:b, $a$ is $\frac{a}{a+b}$ of the whole, and $b$ is $\frac{b}{a+b}$ of the whole.\\

For a group of 24 members where the ratio of men to women is 5:7, the group can be divided into 12 groups of 2 where 5 of the groups, $\frac{5}{12}$, are men, and 7 of the groups, $\frac{7}{12}$ are women. $\frac{5}{12}$ of 24 is 10 and $\frac{7}{12}$ of 24 is 14, so the ratio of men to women is 10:14 which is equivalent to 5:7.\\

There can be triple ratios, or even more, where more than 2 amounts are compared.\\

\section*{Proportions}
Ratio and proportion are sometimes used to mean the same thing but they are actually different.\\

Proportion means equal ratios.\\

1:2 = 2:4 is a proportion.\\

$\frac{3}{4}=\frac{9}{12}$ is a proportion.\\

"a is to b as c is to d" is a proportion.\\

\newpage

Proportions are used in science and in business when scaling things up or down. They are solved by using algebra, calling the quantity that we want to know $x.$\\

Say 12 apples cost \$5.00. How many apples can you buy for \$3.00? Writing that out as an equation, using algebra,
$$\frac{x}{3}=\frac{12}{5}$$
Multiplying both sides by 3, and cancelling,
$$\frac{x\times\cancel{3}}{\cancel{3}}=\frac{12\times3}{5}$$
$$x=\frac{36}{5}=7 \frac{1}{5}=\$7.20$$

\newpage

\subsection*{Cross-Multiplying}
Proportions can also be solved by cross-multiplying.\\

Cross-multiplying is where you multiply the denominator of each fraction by the numerator of the other fraction, making a cross across the equals sign.\\
The general rule is that if $\frac{a}{b}=\frac{c}{d}$ then $a\times d=c\times b$.\\

Say there is a proportion with an unknown value, $x$.\\
$$\frac{5}{3}=\frac{25}{x}$$

Cross multiply to get $5x=25\times3$, then divide both sides by 5 to get $x=25\times3\div5=25$.\\

\newpage

\section*{Percentages}

Per means "for each," as in "one per customer." Cent means 100, as in there are 100 cents in a dollar or 100 years in a century.\\

Percent means "for each 100." A fraction means an amount that has been divided into some number of equal parts, and a percentage is a fraction with 100 parts.\\

5 percent just means $\frac{5}{100}$.\\

The special symbol "\%" is usually used instead of writing "percent."\\

The "\%" comes from Italian "per cento." Over the years the "per" was shortened to "p" and eventually just dropped. "Cento" was shortened to "c/o" with the slash indicating letters that were left out, and eventually "c/o" changed into the \% that we have now.\\

A useful fact about percentages is that because of the commutative property of multiplication, where the order of multiplying doesn't matter, reversing percentages does not change the result. 20\% of 50 is 40, and 50\% of 20 is 10.\\

\subsubsection*{Converting to Percentages}

A percentage is just a type of ratio where amounts are compared relative to 100. 25\% means 25:100. Ratios can be converted to percentages by using proportions.\\

Just remember
\begin{Large}
$\frac{part}{whole}=\frac{percent}{100}$.\\
\end{Large}

What is 30\% of 70?\\
The percent is 30 and the whole is 70, so $\frac{part}{70}=\frac{30}{100}$.\\
Cross-multiply, and divide, so part $= (70 \times 30) \div 100 = 2100 \div 100 = 21$..\\

\
\newpage

\doublespacing

\begin{center}

Enquiries

\textbf{Applied Scholastics Ferndale}

Principal: Paula McLennan

mobile phone: 0431 683 306

email address: apsferndale@gmail.com

website: apsferndale.webs.com

\end{center}

\end{spacing}

\end{document}