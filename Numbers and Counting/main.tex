\documentclass{article}
\usepackage[fontsize=18pt]{fontsize}
\usepackage[a4paper,margin=3cm]{geometry}
\usepackage{setspace}
\usepackage{cancel}

\author{Mike McLennan}
\date{}
\title{Numbers and Counting\\
\vspace{28pt}
\begin{normalsize}Applied Scholastics, Ferndale WA \end{normalsize}}

\begin{document}
\maketitle
\newpage

\section*{Numbers}

A number is an amount of things. It also means the word or symbol used to express the amount of things.\\

A numeral is the word or symbol used to express a number. "Four" and "4" are numerals that represent the same number.\\

Figure is another word for numeral. That's because figure means and outline or a drawing of something, and digits and numerals have different shapes and outlines. A person's outline is known as their figure, for example, or diagrams in books are sometimes called figures. The saying "figure something out" comes from working things out with numbers.\\

A digit is a single symbol that is a numeral in itself or it can stand with other digits as part of a larger numeral. "2" is a digit. "23" is a numeral made up of two digits.\\

\newpage

\section*{Counting}

Count means to note each thing in a group of things to find out how many there are.\\

To number things means to give a number to each thing in turn when counting them. Like the chapters of a book are numbered chapter 1, chapter 2, chapter 3, and so on.\\

To number things can also mean to give a group of things the result of a count. You could say that the people of a town number 1000 or that the number of students in the class is 20.\\

\section*{Tallies}

For most of our history there were no numbers – there were just marks made to keep a count of things and there were many different ways of doing that around the world. This is the simplest sort of number with each mark corresponding to one real thing that has been counted.\\

$\cancel{||||}\ \cancel{||||}\ \cancel{||||}\ |||$\\

One way of counting like this is called keeping a tally. Notches are cut or lines are drawn to record an count. To make the count easier to read, the marks are put into groups of five by making every fifth line go across the first four.

\newpage

\section*{Roman Numerals}

The Roman Empire existed for many centuries and they spread their language and their counting system with them. Roman numerals are still used today.\\

Roman numerals use I for 1, V for 5, X for 10, L for 50, C for 100, D for 500, and M for 1000. The V is said to be a shorthand for a spread hand with 5 fingers and the X is said to be two Vs on top of each other.\\

Using these special symbols makes Roman numerals much shorter than tallies.\\

A smaller amount is written either before or after the larger one, indicating that much more or less. That makes them even shorter to write.\\

XVIII in Roman numerals is easier to read and write than $\cancel{||||}\ \cancel{||||}\ \cancel{||||}\ |||$ as a tally.\\

Counting in Roman numerals is I II III IV V VI VII VIII IX X XI XII XIII XIV XV… and so on.\\

\pagebreak

Here are the Roman Numerals counting from 1 to 100:\\
\begin{center}
\begin{table}[ht]
\scriptsize
\renewcommand*{\arraystretch}{1.4}
\begin{tabular}{rlrlrlrlrl}
1  & I      & 2  & II      & 3  & III      & 4  & IV     & 5  & V    \\
6  & VI     & 7  & VII     & 8  & VIII     & 9  & IX     & 10 & X    \\
11 & XI     & 12 & XII     & 13 & XIII     & 14 & XIV    & 15 & XV   \\
16 & XVI    & 17 & XVII    & 18 & XVIII    & 19 & XIX    & 20 & XX   \\
21 & XXI    & 22 & XXII    & 23 & XXIII    & 24 & XXIV   & 25 & XXV  \\
26 & XXVI   & 27 & XXVII   & 28 & XXVIII   & 29 & XXIX   & 30 & XXX  \\
31 & XXXI   & 32 & XXXII   & 33 & XXXIII   & 34 & XXXIV  & 35 & XXXV \\
36 & XXXVI  & 37 & XXXVII  & 38 & XXXVIII  & 39 & XXXIX  & 40 & XL   \\
41 & XLI    & 42 & XLII    & 43 & XLIII    & 44 & XLIV   & 45 & XLV  \\
46 & XLVI   & 47 & XLVII   & 48 & XLVIII   & 49 & XLIX   & 50 & L    \\
51 & LI     & 52 & LII     & 53 & LIII     & 54 & LIV    & 55 & LV   \\
56 & LVI    & 57 & LVII    & 58 & LVIII    & 59 & LIX    & 60 & LX   \\
61 & LXI    & 62 & LXII    & 63 & LXIII    & 64 & LXIV   & 65 & LXV  \\
66 & LXVI   & 67 & LXVII   & 68 & LXVIII   & 69 & LXIX   & 70 & LXX  \\
71 & LXXI   & 72 & LXXII   & 73 & LXXIII   & 74 & LXXIV  & 75 & LXXV \\
76 & LXXVI  & 77 & LXXVII  & 78 & LXXVIII  & 79 & LXXIX  & 80 & LXXX \\
81 & LXXXI  & 82 & LXXXII  & 83 & LXXXIII  & 84 & LXXXIV & 85 & LXXXV\\
86 & LXXXVI & 87 & LXXXVII & 88 & LXXXVIII & 89 & LXXXIX & 90 & XC   \\
91 & XCI    & 92 & XCII    & 93 & XCIII    & 94 & XCIV   & 95 & XCV  \\
96 & XCVI   & 97 & XCVII   & 98 & XCVIII   & 99 & XCIX   & 100& C    \\
\end{tabular}
\end{table}
\end{center}

I = 1, V = 5, X = 10, L = 50, C = 100, D = 500, M = 1000\\

\newpage

\section*{Natural Numbers}

Natural means things that are in the natural world, which is to say the real or physical world. Natural numbers are the numbers that are used for counting things, as in "I have six ducks," or for ordering things, as in "the third duck quacked."

\paragraph{Whole Numbers}
Natural numbers are also called whole numbers, in that they are counting or ordering whole things rather than parts of things. Whole numbers are all numbers greater than zero and, depending on who you talk to, may or may not include zero as a whole number.

\paragraph{Cardinal Numbers}
Cardinal means main or the thing that other things depend on. Numbers that are used for counting things are called counting numbers or cardinal numbers.

\paragraph{Ordinal Numbers}
Numbers that are used for ordering things are called ordinal numbers. Pointing to the "$3{^{rd}}$" duck in a line of ducks uses an ordinal number.

\newpage

\

\newpage

\large
\doublespacing

\begin{center}

Enquiries

\textbf{Applied Scholastics Ferndale}

Principal: Paula McLennan

mobile phone: 0431 683 306

email address: apsferndale@gmail.com

website: apsferndale.webs.com
\end{center}

\end{spacing}

\end{document}
