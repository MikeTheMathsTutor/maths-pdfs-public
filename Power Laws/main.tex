\documentclass{article}
\usepackage{amsmath}
\usepackage{cancel}
\usepackage[fontsize=16pt]{fontsize}

\author{}
\date{}
\title{Power Laws\\
\vspace{28pt}
\begin{normalsize}Applied Scholastics, Ferndale WA \end{normalsize}}

\begin{document}
\maketitle
\pagebreak
\tableofcontents
\pagebreak

\section{Powers}
Powers are numbers multiplied by themselves a given number of times. The number being multiplied is called the \textit{base}, and the number of times that it is multiplied by itself is called the \textit{power} (or the \textit{index}, or the \textit{exponent}.)\\

The power or index or exponent is written as a small raised number to the right of the base.\\

The word \textit{power} is from Latin \textit{potentias} which is a mistranslation of the Greek word \textit{dynamis} meaning \textit{amplification} used by the mathematician Euclid around 300BC for the square of a line.\\

The word \textit{exponent} was coined in a book from 1544 called \textit{Arithmetica Integra}. It is from Latin \textit{expo}, out of, and \textit{ponere}, place, with the idea of laying something out to view it's parts.\\

\textit{Exponent} is the preferred term in the US but the rest of the world prefers the terms \textit{index} or \textit{power}.\\

\textit{Index} is a Latin word meaning pointer, and the plural of \textit{index} is \textit{indices}. (Pronounced "indisees.") The use of the term index derived from 'pointing' to which of the powers of a number was meant. For example, the powers of 3 are $3^1=3, 3^2=9, 3^3=27, 3^4=81 and 3^5=273$ so you could say that the $4^{th}$ power of 3 is 81, with 4 being the pointer to that particular power.\\

The word power is also used to mean the result of raising a number to a power, such as when we say that 8 is a power of 2 because $2^3=8$.\\

If we use \textit{a} as the base and \textit{n} as the exponent, it is read as the \textit{$n^{th}$ power of a}, or as \textit{a to the power of n}, or as \textit{a the the $n^{th}$ power}.\\

e.g.	a to the power of n is a times a, n times.
	${a^n = a\cdot a\cdot a \cdot ... \cdot a}$ \{for n factors\}
	${3^5 = 3 \cdot 3 \cdot 3 \cdot 3 \cdot 3 = 243}$\\

In writing computer programs, and in some maths programs, because they use only standard text characters and can’t write the small raised n, the “caret” symbol, \^\, is used to write the power of a number.\\

e.g.	$3 \string^ 5 = 3 \cdot 3 \cdot 3 \cdot 3 \cdot 3 = 243$

\newpage

\section{Roots}

The \textit{base} can also be called the \textit{root}, particularly when given some number and you are looking for the number which when multiplied by itself a given number of times will result in that number.\\

The Latin word for \textit{root} is \textit{radical} and that word is also sometimes used to mean the root. It is also thought to be where the radical symbol \textsurd \enspace comes from, being a sort of stretched out letter \textit{r}.\\

The number to the left of the radical symbol that indicates which root is to be taken and it is called the \textit{index} or the \textit{order} or the \textit{degree} of the root.\\

The value inside the radical symbol is known as the \textit{radicand}.\\

e.g.	$\sqrt[n]{a}$ is read as ‘the $n^{th}$ root of a’\\

e.g.	$\sqrt[5]{243}=3$ is read as ‘the $5^{th}$ root of 243 equals 3.’ (notice that $3^5 = 243$)\\

\newpage

The second root of a number is also called its ‘square’ root, because the area of a square is given by the product of the lengths of its sides. The number is usually not written. If a radical symbol has no number given, then it is implied to be a square root that is meant.\\

e.g.	$\sqrt{a}$ is read as ‘the square root of a.’\\

The third root of a number is also called its ‘cube’ root, because the volume of a cube is given by the product of the lengths of its three dimensions.\\

e.g.	$\sqrt[3]{a}$ is read as ‘the cube root of a.’

\newpage

\section{Power Laws}

The powers of numbers follow some useful laws:\\

\subsection{Unit Power Law}
\begin{Large}
$$a^1=a$$
\end{Large}
$$\text{e.g. }3^1=1$$

\vspace{16pt}
\subsection{Product Law}
\begin{Large}
$$a^m\times a^n=a^{m+n}$$
\end{Large}

\begin{equation*}
\begin{split}
\text{e.g. }{3^2 \times 3^3} = {3^{2+3} &= 3^5}\\
&= {3 \cdot 3 \times 3 \cdot 3 \cdot 3}\\
&= {3 \cdot 3 \cdot 3 \cdot 3 \cdot 3} = 3^5
\end{split}
\end{equation*}

\newpage

\subsection{Quotient Law}
\begin{Large}
$$\frac{a^m}{a^n}=a^{m-n}$$
\end{Large}
\begin{align*}
\text{e.g. }\frac{3^5}{3^3}&=3^{5-3}=3^2\\
&=(3 \times 3 \times 3 \times 3 \times 3) \div (3\times 3 \times 3 )\\
&=\frac{3 \cdot 3 \cdot \cancel{3 \cdot 3 \cdot 3}}{\cancel{3 \cdot 3 \cdot 3}}=3 \cdot 3=3^2
\end{align*}

\vspace{32pt}
\subsection{Power of a Power Law}
\begin{Large}
$$(a^m)^n=a^{m \times n}$$
\end{Large}
\begin{align*}
\text{e.g. }(3^2)^3&=3^{2 \times 3}=3^6\\
&=(3 \cdot 3) \times (3 \cdot 3) \times (3\cdot 3)\\
&= 3 \cdot 3 \cdot 3 \cdot 3 \cdot 3 \cdot 3=3^6
\end{align*}

\newpage

\subsection{Power of a Product Law}

\begin{Large}
$$(a \times b)^m=a^m \times b^m$$
\end{Large}
\\

\begin{center}
\begin{large}
\text{e.g. }$(3 \cdot 3)^2=3^2 \cdot 3^2=3^{2+2}=3^4$
\end{large}
\end{center}

\vspace{32pt}
\subsection{Power of a Quotient Law}
\begin{Large}
$$\left(\frac{a}{b}\right)^m=\frac{a^m}{b^m}$$
\end{Large}
\\

\begin{center}
\begin{large}
\text{e.g. }$\left(\frac{3}{9}\right)^2=\frac{3^2}{9^2}=\frac{9}{81}=\frac{1}{9}$
\end{large}
\end{center}

\newpage

\subsection{Zero Power Law}
\begin{Large}
$$a^0=1$$
\end{Large}
\begin{center}
\text{e.g. }$3^0=1$
\end{center}

\vspace{32pt}
Some explanations of the Zero Power Law:
\\

\subsubsection*{by the Quotient Law}
\begin{align*}
\frac{n^x}{n^x}&=\frac{\cancel{n^x}}{\cancel{n^x}}  =1\\
&=n^{x-x}=n^0
\end{align*}
Anything divided by itself equals 1 but, by the quotient law, a number raised to a power and divided by itself is equal to that number raised to the power 0. (For any number other than $0^0$.)

\newpage

\subsubsection*{by the Multiplicative Identity}
\begin{large}
\begin{align*}
n^x&=1 \times \underbrace{n \times \ldots \times n}_{\text{n times}}\\
\end{align*}
\end{large}

\paragraph{Identities}
The number 1 is known as the ‘multiplicative identity’ because anything multiplied by 1 results in the same (identical) number. Zero is the ‘additive identity’ because 0 added to anything results in the same number. The ‘1×’ or ‘0+’ is always there but not usually written.\\

If you apply the fact that 1 times any number equals the same number then you see that $n^0$ is just 1 multiplied by n, zero times, which is 1.\\

\newpage

\subsubsection*{by Dividing by the Base}
A number to a power, divided by its base, equals that number to one less power:\\

\begin{equation*}
\begin{split}
x^{n-1}        &= \frac{x^n}{x}\\
\\
\text{e.g. }2^3&= \frac{2^4}{2}
= \frac{2 \times 2 \times 2 \times \cancel{2}}{\cancel{2}} = 2^3\\
\\
\text{So, }x^0 &= x^{1-1} = \frac{x^1}{x} = \frac{x}{x} = 1
\end{split}
\end{equation*}

\newpage

\subsection{Negative Power Law}

\vspace{16pt}
\begin{Large}
$$a^{-n}=\frac{1}{a^n}$$

$$\text{also}$$

$$\frac{1}{a^{-n}}=a^n$$
\end{Large}

\vspace{16pt}
\begin{large}
$$\text{e.g. }3^{-2}=\frac{1}{3^2}=\frac{1}{9}$$
\end{large}

\vspace{32pt}
Some explanations of the Negative Power Law:

\subsubsection*{by the Quotient Law}
\begin{align*}
a^{-n}=a^{0-n}=\frac{a^0}{a^n}=\frac{1}{a^n}
\end{align*}

\subsubsection*{by Arithmetic}
\begin{large}
\begin{center}
\begin{tabular}{rr}
&\text{e.g. }$\frac{a^2}{a^5}=a^{2-5}=a^3$\\\\
&$\frac{a \cdot a}{a \cdot a \cdot a \cdot a \cdot a}
=\frac{\cancel{a \cdot a}}{\cancel{a \cdot a} \cdot a \cdot a \cdot a}
=\frac{1}{a \cdot a \cdot a}=\frac{1}{a^3}$
\end{tabular}
\end{center}
\end{large}

\vspace{16pt}

\subsubsection*{by the Multiplicative Identity}
A positive exponent is how many times to multiply by a number. The multiplicative identity, 1×, is always there but not usually written.
\begin{align*}
a^n&=1 \underbrace{\times a \times \ldots \times a}_{\text{n times}}\\

\text{e.g. }2^3&=1 \underbrace{\times 2 \times 2 \times 2}_{\text{3 times}}
\end{align*}

If you include the multiplicative identity then you can see that a negative exponent is just how many times to divide by a number, the opposite of multiplication, starting at 1.

\begin{align*}
a^{-n}&=1 \underbrace{\div a \div \ldots \div a}_{\text{n times}}\\
\text{e.g. }2^{-3}&=1 \underbrace{\div 2 \div 2 \div 2}_{\text{3 times}}=\frac{1}{2^3}=\frac{1}{8}
\end{align*}

\newpage

\subsection{Reciprocal Power law}
\begin{Large}
$$a^{\frac{1}{n}}=\sqrt[n]{a}$$
\end{Large}
\begin{center}
\begin{align*}
\text{e.g. }4^{\frac{1}{2}}&=\sqrt[2]{4}=2\\
\text{e.g. }32^{\frac{1}{5}}&=\sqrt[5]{32}=2
\end{align*}
\end{center}

\vspace{16pt}
\subsubsection*{An explanation of the Reciprocal Power Law by using the Power of a Power Law}

A reciprocal means the inverse of a fraction or a fraction that has been flipped the other way around. For example, $\frac{3}{4}$ and $\frac{4}{3}$ are reciprocal fractions. A fraction multiplied by it’s reciprocal equals 1. For example, $\frac{1}{2} \times \frac{2}{1}=1$.

\begin{large}
\begin{align*}
(a^{\frac{1}{n}})^{\frac{n}{1}}=a^{\frac{1}{n} \times \frac{n}{1}}=a^{\frac{n}{n}}=a^1=a
\end{align*}
\end{large}

A number raised to the reciprocal of a power, that is then raised to that power, gives the original number.

\newpage

Expanding this out, you can see that $a^{\frac{1}{n}}$ must be multiplied by itself $n$ times to equal $a$:\\
\begin{equation*}
\begin{split}
(a^{\frac{1}{n}})^n&= \underbrace{a^{\frac{1}{n}} \times a^{\frac{1}{n}} \times \ldots \times a^{\frac{1}{n}}}_{\text{n times}}\\
&=a^{\frac{n}{n}}=a^1=a\\
\end{split}
\end{equation*}

and that $a^{\frac{1}{n}}$ is actually the $n^{th}$ root of $a$.

\newpage
\subsection{Fractional Power Law}
(Also known as the ‘rational’ power law because ‘rational’ means to do with ratios, i.e. fractions.)

\begin{Large}
$$a^{\frac{m}{n}}=\sqrt[n]{a^m}=(\sqrt[n]{a})^m$$
\end{Large}

\vspace{16pt}
\begin{center}
\text{e.g. }$3^{\frac{2}{3}}=\sqrt[3]{3^2}\text{  }(=\sqrt[3]{9} ) =(\sqrt[3]{3})^2$
\end{center}

\vspace{16pt}
\subsubsection*{An explanation by using the Power of a Power Law and Reciprocal Power Law}
Separating the fraction into numerator and denominator, so that $\frac{m}{n}=m \times \frac{1}{n}$, you can see:

\begin{large}
\begin{align*}
a^{\frac{m}{n}}&=a^{m \times \frac{1}{n}}=(a^m)^{\frac{1}{n}}=\sqrt[n]{a^m}\\
               &\text{and }\\
a^{\frac{m}{n}}&=a^{\frac{1}{n} \times m}=(a^{\frac{1}{n}})^m=(\sqrt[n]{a})^m\\
\end{align*}
\end{large}

\newpage

\subsubsection*{Negative Fractional Powers}
\begin{Large}
$$a^{\frac{-m}{n}}=1/a^{\frac{m}{n}}=\frac{a}{(\sqrt[m]{a})^n}$$
\end{Large}
\begin{center}
\text{e.g. } $3^{\frac{-2}{3}}=1/3^{\frac{2}{3}}=\frac{1}{(\sqrt[2]{3})^3}\approx \frac{1}{1.73^3}\approx\frac{1}{5.20}\approx0.19$
\end{center}

\vspace{16pt}
\subsubsection*{Fractions with Negative Powers}
\begin{Large}
$$(\frac{a}{b})^{-n}=\frac{1}{({\frac{a}{b}})^n}=\frac{1}{\frac{a^n}{b^n}}=\frac{b^n}{a^n}$$
\end{Large}
\begin{center}
\text{e.g. }
$(\frac{2}{3})^{-2}=\frac{1}{({\frac{2}{3}})^2}=\frac{1}{\frac{2^2}{3^2}}=\frac{3^2}{2^2}=\frac{9}{4}$
\end{center}

\newpage
\subsubsection*{Multiplication of Negative Powers}
different powers:
\begin{Large}
$$a^{-m} \times a^{-n}=a^{-(m+n)}=\frac{1}{a^{m+n}}$$
\end{Large}
\begin{center}
\text{e.g. }
$2^{-2} \times 2^{-3}=2^{-(2+3)}=2^{-5}=\frac{1}{2^5}=\frac{1}{32}$
\end{center}

different bases:
\begin{Large}
$$a^{-n}\times b^{-n}=(a\times b)^{-n}$$
\end{Large}
\begin{center}
\text{e.g. }
$2^{-3}\times 3^{-3}=(2\times 3)^{-3}=6^{-3}= \frac{1}{6^3}=\frac{1}{216}$
\end{center}

different bases and powers:
\begin{Large}
$$a^{-m} \times b^{-n}\text{\normalsize (calculate separately)}$$
\end{Large}
\begin{center}
\text{e.g. }
$2^{-3} \times 3^{-2}=\frac{1}{2^3} \times \frac{1}{3^2}=\frac{1}{8} \times \frac{1}{9}=\frac{1}{72}$
\end{center}

division of negative powers:
\begin{Large}
$$a^{-m} \div a^{-n}=a^{-m-(-n)}=a^{-m+n}$$
\end{Large}
\begin{center}
\text{e.g. }
$3^{-3} \div 3^{-2}=3^{-3-(-2)}=3^{-3+2}=3^{-1}=\frac{1}{3^1}=\frac{1}{3}$
\end{center}

\newpage

\subsection{Powers of a Negative\\ Number}
A negative number multiplied by a negative number results in a positive product. When that positive product is multiplied by a negative number, the result is a negative product. And so the sign alternates depending on whether the power is an odd or an even number.\\

A negative number taken to an even power gives a positive result.\\

$\text{e.g. }(-4)^4 = -4 \times -4 \times -4 \times -4 = 256$

A negative number taken to an odd power gives a negative result.\\

$\text{e.g. }(-4)^5 = -4 \times -4 \times -4 \times -4 \times -4 = -1024$\\

\newpage

\subsubsection*{Roots of an Even Power of a Negative Number}

There is no $n^{th}$ root of an even power of a negative number.

e.g.	No number can be multiplied by itself to find $\sqrt{-16}$.

\subsubsection*{Roots of an Odd Power of a Negative Number}
You can, however, find the $n^{th}$ root of an odd power.

e.g.	$-3 \times -3 \times -3 = -27, so \sqrt[3]{-27}= -3$.

\paragraph{Brackets}
Be careful with brackets,\\
e.g. $-3^2 = -(3 \times 3)=-1 \times 3 \times 3 = -9$,\\
but $(-3)^2 = -3 \times -3 = 9.$

\newpage

\section{Scientific Notation}
(also called Exponential Notation, or Standard Form)\\

Values in science can range from very large to very small. To make these numbers shorter they are usually expressed as multiples of some power of ten.\\

For example, the speed of light is approximately 300,000,000 meters per second, but it is usually written more briefly as $3x10^8$ m/s, or 3E+8 m/s.\\

Similarly for very small values, the weight of an electron has been measured as\\0.0000000000000000000000000009109 kilograms but that is much more briefly written as $9.109x10^{-31}$ kg, or 9.109E-31 kg.

\newpage
\
\newpage
\
\newpage
\
\newpage
\
\newpage
\
\newpage
\
\newpage
\
\newpage
\
\newpage
\
\newpage
\

\begin{center}
\linespread{2}\large

Enquiries

\textbf{Applied Scholastics Ferndale}

Principal: Paula McLennan

mobile phone: 0431 683 306

email address: apsferndale@gmail.com

website: apsferndale.webs.com
\end{center}


\end{document}