\documentclass[12pt]{article}
\usepackage{amsmath}
\usepackage{pgfplots}
\usepackage{enumitem}
\usepackage{tikz}
\usetikzlibrary{arrows.meta}

\title{The Equation of a Circle}\\
\author{Tutoring Centre Ferndale\\
\includegraphics[width=4em]{ApS_logo.png}}
\date{}

\begin{document}

\maketitle

\section*{What is the Equation of a Circle?}

The equation of a circle is a mathematical expression that describes all the points that are a fixed distance (the radius) from a central point (the center). The standard form of the equation of a circle is derived from the Pythagorean theorem.

\section*{Key Terms}

\begin{itemize}
    \item \textbf{Circle:} A set of all points in a plane that are a fixed distance (the radius) from a central point (the center).
    \item \textbf{Center:} The fixed point from which all points on the circle are equidistant, denoted as \( (h, k) \).
    \item \textbf{Radius:} The fixed distance from the center to any point on the circle, denoted as \( r \).
\end{itemize}

\section*{Equation of a Circle}

The standard form of the equation of a circle with center at \( (h, k) \) and radius \( r \) is:
\boldmath{\[
(x - h)^2 + (y - k)^2 = r^2
\]}

\section*{The Pythagorean Theorem}

The equation of a circle is derived from the Pythagorean theorem. In a right triangle with legs parallel to the coordinate axes, the hypotenuse is the distance between a point \( (x, y) \) on the circle and the center \( (h, k) \). Thus, the distance formula:
\[
r = \sqrt{(x - h)^2 + (y - k)^2}
\]
Squaring both sides gives the standard form of the equation of a circle.

\section*{Effects of Each Variable}

\begin{itemize}
    \item \textbf{Center \((h, k)\):} Changing \( h \) or \( k \) moves the circle horizontally or vertically, respectively.
    \item \textbf{Radius \( r \):} Changing \( r \) changes the size of the circle. A larger \( r \) makes the circle larger, and a smaller \( r \) makes the circle smaller.
\end{itemize}

\begin{center}
\begin{tikzpicture}
    \begin{axis}[width=\textwidth,
        axis lines = middle,
        xlabel = {$x$},
        ylabel = {$y$},
        ymin=-2, ymax=5.5,
        xmin=-2, xmax=5.5,
        domain=-2:6,
        samples=100,
        ytick={-1,0,1,2,3,4,5},
        xtick={-1,0,1,2,3,4,5},
        grid=both,
        grid style={line width=.3pt, draw=gray!50}]
        \addplot [thick, samples=200, domain=-0.5:4.5] {sqrt(2.5^2 - (x-2)^2) + 1};
        \addplot [thick, samples=200, domain=-0.5:4.5] {-sqrt(2.5^2 - (x-2)^2) + 1};
        \node at (axis cs:4,4.2) [anchor=east] {\boldmath{$(x-2)^2 + (y-1)^2 = 2\frac{1}{2}^2$}};
        \addplot [only marks, mark=*] coordinates {(2,1)};
        \node at (axis cs:2,1) [anchor=east] {(2,1)};
        \node at (axis cs:2,0.9) [below] {centre};
        \draw[->, thick] (axis cs:2,1) -- (axis cs:{2+2.5*cos(20)},{1+2.5*sin(20)}) node[midway, above, sloped] {radius $2\frac{1}{2}$};
        \draw[dashed] (axis cs:2,1) -- (axis cs:{2+2.5*cos(20)},1) -- (axis cs:{2+2.5*cos(20)},{1+2.5*sin(20)});
    \end{axis}
\end{tikzpicture}
\end{center}

\section*{Examples}

\subsection*{Example 1}

Find the equation of a circle with center at \( (3, -2) \) and radius 5.\\

\textbf{Solution:}

Using the standard form:
\[
(x - h)^2 + (y - k)^2 = r^2
\]

Substitute \( h = 3 \), \( k = -2 \), and \( r = 5 \):
\[
(x - 3)^2 + (y + 2)^2 = 5^2
\]

Simplify:
\[
(x - 3)^2 + (y + 2)^2 = 25
\]

\textbf{Equation of the circle:} \( (x - 3)^2 + (y + 2)^2 = 25 \)

\subsection*{Example 2}

Find the center and radius of the circle given by the equation \( (x + 1)^2 + (y - 4)^2 = 16 \).\\

\textbf{Solution:}

Compare with the standard form:
\[
(x - h)^2 + (y - k)^2 = r^2
\]

We have:
\[
(x - (-1))^2 + (y - 4)^2 = 4^2
\]

Thus, the center is \( (-1, 4) \) and the radius is 4.\\

\textbf{Center:} \( (-1, 4) \)

\textbf{Radius:} 4

\newpage

\section*{Practice Problems}

Solve the following problems related to the equation of a circle:

\subsection*{Problem 1}

Find the equation of a circle with center at \( (2, 3) \) and radius 7.\\

\textbf{Solution:}

Using the standard form:
\[
(x - 2)^2 + (y - 3)^2 = 7^2
\]
\[
(x - 2)^2 + (y - 3)^2 = 49
\]\\

\textbf{Equation of the circle:} \( (x - 2)^2 + (y - 3)^2 = 49 \)

\subsection*{Problem 2}

Find the center and radius of the circle given by the equation \( x^2 + y^2 = 9 \).\\

\textbf{Solution:}

Compare with the standard form:
\[
(x - 0)^2 + (y - 0)^2 = 3^2
\]

Thus, the center is \( (0, 0) \) and the radius is 3.\\

\textbf{Center:} \( (0, 0) \)

\textbf{Radius:} 3

\section*{Conclusion}

The equation of a circle is a crucial concept in geometry that helps us understand the properties and relationships of circles. By learning how to derive and manipulate this equation, students can solve a variety of geometric problems and deepen their understanding of mathematical relationships. Practice with these problems to strengthen your understanding of the equation of a circle and its applications!

\end{document}
